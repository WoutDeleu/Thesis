%%%%%%%%%%%%%%%%%%%%%%%%%%%%%%%%%%%%%%%%%%%%%%%%%%%%%%%%%%%%%%%%%%% 
%                                                                 %
%                            CHAPTER                              %
%                                                                 %
%%%%%%%%%%%%%%%%%%%%%%%%%%%%%%%%%%%%%%%%%%%%%%%%%%%%%%%%%%%%%%%%%%% 

\chapter{Inleiding}

\section{Situering}
Sociale media is zo goed als niet meer weg te denken uit het huidige moderne
leven. Over de jaren heen zijn er verschillende definities aan het concep
sociale media gegeven. In het werk van Howard en Park wordt sociale media
gedefinieerd als de infrastructuur en tools om content te maken en te
verspreiden~\cite{PhilipsAndParks}. Deze definitie is erg ruim, en vertakt zich
dus in heel wat facetten, waaronder sociale netwerken, media sharing networks,
etc, maar ook de tak van de fitnesstrackers. Deze opkomst van nieuwe media
brengt echter ook onbedoelde maar significante privacy bezorgdheden met zich
mee.

De focus in deze dissertatie ligt op privacy binnen fitnesstrackers, meer
specifiek platformen die \ac{gps}-locaties gebruiken, zoals Strava, Nike Run
Club, etc. Dit zijn platformen waar personen sportactiviteiten zoals lopen,
fietsen, wandelen,\ldots kunnen delen met elkaar. Het algemene concept is
hierbij dat wanneer je een sportactiviteit uitvoert, je deze voor je volgers en
vrienden beschikbaar maakt. De sportactiviteiten zullen natuurlijk bepaalde
gegevens bevatten die zichtbaar zijn voor die andere gebruikers.
Figuur~\ref{fig:activityExample} geeft bijvoorbeeld weer hoe Strava de afstand,
bewegingstijd, en natuurlijk de \ac{gps}-locaties deelt. Vele van deze gegevens
hebben direct of indirect een negatieve impact op de privacy van de user. Deze
negatieve gevolgen komen dan vooral in de vorm van het onbedoeld vrijgeven van
\textit{gevoelige locaties}. Onder het concept van een gevoelige locatie vallen
heel wat beschrijvingen. Een algemene beschrijving kan zijn, een locatie die
geografische informatie deelt die negatieve gevolgen kan hebben, en die je dus
liever niet deelt. In het kader van dit onderzoek zal dit gaan over start en
eindlocaties van activiteiten. Dit kan gaan over woonplaatsen, wat kan leiden
tot o.a.\ stalking, alsook locaties waar sportmateriaal wordt opgeborgen, wat
eventueel zou kunnen leiden tot diefstal. Er zijn gevallen bekend van
fietsdieven die Strava gebruiken om fietsen te kunnen
lokaliseren~\cite{Sportapp72:online}~\cite{Cyclistw89:online}. Grootschaligere
voorbeelden die zeker het vermelden waard zijn, zijn de gevallen waarbij
geheime militaire basissen ontdekt worden door het bestuderen van de
heatmap\footnote{Een heatmap is een visuele weergave van gegevens waarbij
    verschillende kleuren worden gebruikt om de intensiteit van waarden in een
    matrixachtige structuur weer te geven. In de context van GPS-locaties kan een
    heatmap worden gebruikt om de concentratie of frequentie van locaties op een
    kaart weer te geven~\cite{Whatishe21:online}. Het doel is om gebieden met een
    hoge dichtheid of veelvuldige locaties te identificeren en visueel te
    markeren.} vrijgegeven door Strava~\cite{Fitnesst33:online}.
Figuur~\ref{fig:heatmap} toont een voorbeeld van de heatmap van Strava.
\begin{figure}
    \centering
    \includegraphics[width=\textwidth]{fig/Heatmap_strava.png}
    \caption{Voorbeeld Heatmap, vrijgegeven door Strava~\cite{StravaGl10:online}}\label{fig:heatmap}
\end{figure}
\begin{figure}
    \centering
    \includegraphics[width=0.5\linewidth]{fig/VoorbeeldActiviteiten/VoorbeeldActiviteit_Cropped.png}
    \caption{Voorbeeldactiviteit Strava}\label{fig:activityExample}
\end{figure}

Fitnesstrackers implementeren elk manieren om de privacy van de users te
verbeteren. De meest eenvoudige te bedenken manier is misschien wel de
mogelijkheid om activiteiten te verbergen voor een selectie van personen (bv.\
iedereen die geen volger is). Zo kunnen enkel de mensen die de gebruiker
expliciet toelaat activiteiten bekijken. Een complexer alternatief is het
gebruik van \acp{EPZ}. Hierbij wordt de weergegeven route voor de persoon die
meekijkt gedeeltelijk verborgen. Er wordt als het ware een deel van de route
afgekapt. De echte begin- en eindpunten zullen binnenin het afgekapte deel
liggen. Er zullen nieuwe punten worden gegenereerd, op de rand van de cirkel,
die voor de externe waarnemer het begin en einde zullen voorstellen. Het begin-
en eind-deel van de route wordt dus onzichtbaar voor de andere gebruikers. Door
de aanwezigheid van al deze pogingen tot privacyverbeteringen valt op dat de
ontwikkelaars van de platformen erg bewust zijn van de mogelijke gevaren.
Echter is er een afweging te maken bij de implementatie tussen de bruikbaarheid
van het platform, en de privacy van de eindgebruiker. Hoe meer data wordt
vrijgegeven, hoe groter de kans op mogelijk gevoelige info wordt meegegeven.
Aan de andere kant, bij het weglaten van informatie gaat de
gebruiksvriendelijkheid en de aanwezigheid van nuttige data van het platform er
sterk op achteruit.

\section{Doelstelling}
In dit onderzoek bekijken we of er een mogelijkheid bestaat om private locaties
(verborgen start- en eindlocaties) van een activiteiten te achterhalen, ondanks
het gebruik van de \ac{EPZ} beschreven in Sectie~\ref{sec:EPZ} als privacy
beveiligingsmechanisme. In het verleden werden enkele manieren beschreven om
a.d.h.v.\ andere metadata zoals hoogtedata en afstanden de \ac{EPZ} te omzeilen
(\textit{~\cite{Dhondt},~\cite{Verdonck_2022}}).
In deze thesis wordt meer in detail gegaan op het gebruik van snelheidsdata.
Als basis voor deze aanval gebruiken we de inferentie aanval op de EPZ
van~\citeauthor{Dhondt}. We
onderzoeken of deze aanval nog steeds mogelijk is bij het weglaten van bepaalde
gegevens, en dus door het gebruik van andere gegevens. De focus ligt in deze
studie voornamelijk op snelheidsdata.

Om deze doelstelling te behalen is eerst een berekeningsmechanisme nodig voor
de afstanden die nodig zijn om de inferentie-aanval te kunnen uitvoeren. Daarna
voeren we een analyse uit op het verschil tussen de berekende afstanden, en de
waarden afgeleid volgens de berekeningen
van~\citeauthor{Dhondt}. Zo kunnen we
de effectiviteit van de aanval a priori schatten. Er is een analyse van de
beschikbare data, en een bespreking en reflectie over de resultaten van de
aanval.