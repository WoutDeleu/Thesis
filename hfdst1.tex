%%%%%%%%%%%%%%%%%%%%%%%%%%%%%%%%%%%%%%%%%%%%%%%%%%%%%%%%%%%%%%%%%%% 
%                                                                 %
%                            CHAPTER                              %
%                                                                 %
%%%%%%%%%%%%%%%%%%%%%%%%%%%%%%%%%%%%%%%%%%%%%%%%%%%%%%%%%%%%%%%%%%% 

\chapter{Inleiding}

\section{Situering}
Sociale media is zo goed als niet meer weg te denken uit het huidige moderne
leven. Over de jaren heen zijn er dan ook verschillende definities gegeven. in
In het werk van Howard en Park wordt sociale media gedefinieerd als de
infrastructuur en tools om content te maken en te
verspreiden\cite{PhilipsAndParks}. Deze definitie is erg ruim, en vertakt zich
dus in heel wat facetten, waaronder sociale netwerken, media sharing
networks,\ldots Maar ook de fitnesstrackers. Deze opkomst van nieuwe media
brengt echter ook onbedoelde maar significante privacy bezorgdheden met zich
mee.

De focus in deze dissertatie ligt op privacy binnen fitnesstrackers, meer
specifiek platformen die gps-locaties gebruiken, zoals Strava, Nike Run Club,
etc. Dit zijn platformen waar sportactiviteiten zoals lopen, fietsen,
wandelen,\ldots kunnen worden gedeeld met andere personen. Het algemene concept
is hierbij dat wanneer je een sportactiviteit uitvoert, je deze voor je volgers
en vrienden beschikbaar maakt. De sportactiviteiten zullen natuurlijk bepaalde
gegevens bevatten die zichtbaar zijn voor die andere gebruikers,
Figuur~\ref{fig:activityExample} geeft bijvoorbeeld weer hoe Strava de afstand,
bewegingstijd, en natuurlijk de gps-locaties deelt. Vele van deze gegevens
hebben direct of indirect een negatieve impact op de privacy van de user. Deze
negatieve gevolgen komen dan vooral in de vorm van het onbedoeld vrijgeven
\textit{gevoelige locaties}. Onder het concept van een gevoelige locatie vallen
heel wat beschrijvingen. Een algemene beschrijving kan zijn, een locatie die
geografische informatie deelt die negatieve gevolgen kan hebben, en die je dus
liever niet deel. In het kader van dit onderzoek, zal dit dan gaan over start
en eindlocaties van activiteiten. Dit kan gaan over woonplaatsen, wat kan
leiden tot o.a.\ stalking. Alsook locaties waar sportmateriaal wordt
opgeborgen. Er zijn gevallen bekend van fietsdieven die Strava gebruiken om
fietsen te kunnen lokaliseren\cite{Sportapp72:online}\cite{Cyclistw89:online}.
Grootschaligere voorbeelden die zeker het vermelden waard zijn, zijn de
gevallen waarbij geheime militaire basissen ontdekt worden door het bestuderen
van de heatmap\cite{Fitnesst33:online}.
\begin{figure}
    \centering
    \includegraphics[width=0.5\linewidth]{fig/VoorbeeldActiviteiten/VoorbeeldActiviteit_Cropped.png}
    \caption{Voorbeeldactiviteit Strava}\label{fig:activityExample}
\end{figure}

Deze platformen implementeren elk manieren om de privacy van de users te
verbeteren. Hiervoor zijn verschillende manieren mogelijk. De meest eenvoudige
te bedenken is misschien wel de mogelijkheid om activiteiten te verbergen voor
een selectie van personen (bv.\ iedereen die geen volger is). Zo kunnen enkel
de mensen die de gebruiker expliciet toelaat activiteiten bekijken. Een
complexer alternatief is het gebruik van \textit{endpoint privacy zones} (EPZ).
Hierbij wordt de weergegeven route voor de persoon die meekijkt gedeeltelijk
verborgen. Er wordt als het ware een deel van de route afgekapt, en het eind-
en startpunt worden verschoven. Het begin- en eind-deel van de route wordt dus
onzichtbaar voor de andere gebruikers. Door de aanwezigheid van al deze
pogingen tot privacyverbeteringen valt op dat de ontwikkelaars van de
platformen erg bewust zijn van de mogelijke gevaren. Echter is er een afweging
te maken bij de implementatie tussen de bruikbaarheid van het platform, en de
privacy van de eindgebruiker. Hoe meer info wordt vrijgegeven, hoe groter de
kans om mogelijk gevoelige info wordt meegegeven. Aan de andere kant, bij het
weglaten van informatie gaat de gebruiksvriendelijkheid en de aanwezigheid van
nuttige info van het platform serieus achteruit.

\section{Doelstelling}
In dit onderzoek bekijken we of er een mogelijkheid bestaat om private locaties
(verborgen start- en eindlocaties) van een activiteiten te achterhalen, ondanks
het gebruik van de EPZ~\ref{EPZ} als privacy beveiligingsmechanisme. In het
verleden werden enkele manieren beschreven om a.d.h.v.\ andere metadata zoals
hoogtedata en afstanden de EPZ te omzeilen
(\textit{\cite{Dhondt_Pochat_Voulimeneas_Joosen_Volckaert_2022},\cite{Verdonck_2022}}).
Gedurende deze thesis wordt meer in detail gegaan op het gebruik van
snelheidsdata. Als basis voor deze aanval wordt de inferentie aanval op de EPZ
van~\citeauthor{Dhondt_Pochat_Voulimeneas_Joosen_Volckaert_2022} genomen. Er
wordt dan onderzocht of deze aanval nog succesvol kan worden uitgevoerd bij het
weglaten van bepaalde gegevens, en dus door het gebruik van andere gegevens. De
focus ligt in deze studie voornamelijk op snelheidsdata.

Om deze doelstelling te bekomen is eerst een analyse nodig op de afwijkingen
van tussen de berekende afstanden nodig om de inferentie aanval uit te voeren,
en de waarden afgeleid volgens de berekeningen
van~\citeauthor{Dhondt_Pochat_Voulimeneas_Joosen_Volckaert_2022}. Er is een
analyse van de beschikbare data, en natuurlijk ook een studie gebeuren van de
effectiviteit van deze aanval op basis van de nieuw bekomen afstanden.

% Uitschrijven Attack Model