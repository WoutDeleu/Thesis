%%%%%%%%%%%%%%%%%%%%%%%%%%%%%%%%%%%%%%%%%%%%%%%%%%%%%%%%%%%%%%%%%%% 
%                                                                 %
%                            CHAPTER                              %
%                                                                 %
%%%%%%%%%%%%%%%%%%%%%%%%%%%%%%%%%%%%%%%%%%%%%%%%%%%%%%%%%%%%%%%%%%% 

\chapter{Setting aanval}\label{sec:inferentieaanval}
Gedurende dit hoofdstuk de setting alsook de werking van de aanval beschreven.
De aanval is sterk gebaseerd op de aanvallen
van~\citeauthor{Dhondt_Pochat_Voulimeneas_Joosen_Volckaert_2022}\cite{Dhondt_Pochat_Voulimeneas_Joosen_Volckaert_2022}
en~\citeauthor{Verdonck_2022}\cite{Verdonck_2022}. Deze aanvallen worden
inferentie-aanvallen genoemd, vanwege het feit dat uit metadata essentiële
gegevens kunnen worden geïnfereerd. In het geval
van~\citeauthor{Dhondt_Pochat_Voulimeneas_Joosen_Volckaert_2022} gaat dit over
afgelegde afstand binnenin de \ac{EPZ}. In het geval
van~\citeauthor{Verdonck_2022} gaat dit dan weer over geïnduceerde
hoogteverschillen binnen de privacy zone. Als eerst zal kort de mogelijkheden
van een mogelijke aanvaller beschouwd in deze thesis worden besproken. Daarna
wordt de inferentie aanval
van~\citeauthor{Dhondt_Pochat_Voulimeneas_Joosen_Volckaert_2022}, die de basis
vormt voor de aanval in deze thesis, besproken volgens een opdeling in drie
stappen.

\section{Definitie aanvaller}
Deze thesis voert een onderzoek naar de mogelijkheid om een \ac{EPZ} te
omzeilen. De studie wordt dus gevoerd vanuit het opzicht van een aanvaller.
Vooraleer de werking van een aanval wordt beschreven, is het belangrijk om een
zicht te hebben op het doel, en de capaciteiten van een aanvaller.

Hier is een aanvaller een gebruiker van het platform, die geen eigenaar is van
een activiteit. Hij heeft echter wel zicht op alle metadata die publiekelijk
gedeeld is. Dit is data zoals afgelegde afstand, snelheid, tempo, \ldots
Aangezien de aanval gaat over het omzeilen \acp{EPZ} worden activiteiten
beschouwd die gecloaked zijn. De aanvaller heeft dus geen zicht op de reële
start- en/of eindlocatie, zijn doel is dan ook om de ondanks de aanwezigheid
van cloaking deze gevoelige locatie te achterhalen.

Vanuit het oogpunt van de inferentie-aanval beschreven
door~\citeauthor{Dhondt_Pochat_Voulimeneas_Joosen_Volckaert_2022} heeft de
aanvaller toegang tot alle data die publiek beschikbaar is. Deze gebruikt dan
voornamelijk afgelegde weg als basis.

De aanvaller die in deze thesis wordt beschreven, heeft echter geen toegang tot
deze afstandsdata. Hij heeft wel nog toegang tot de ruwe GPS-data, maar ook de
snelheid, het tempo enzovoort. Het onderzoek bestaat er dus uit om te
onderzoeken in hoeverre een aanval nog mogelijk is wanneer de afstandsdata
onbruikbaar zou zijn. Een alternatieve aanpak wordt dus onderzocht om de
inferentie-aanval alsnog succesvol te kunnen uitvoeren.

\subsection{Assumpties}
Om de aanval te kunnen uitvoeren, moeten enkele assumpties worden gemaakt.
\citeauthor{Dhondt_Pochat_Voulimeneas_Joosen_Volckaert_2022} maakte al enkele
assumpties om de inferentie aanval succesvol uit te voeren. Voor dit onderzoek
moeten deze dus ook gelden. De eerste bestaat eruit opdat dat de zichtbare
begin -en eindpunten op de cirkel moeten liggen. Ten tweede moet de beschermde
locatie op de roadgraph liggen, hij kan niet buiten het voor ons te mappen
gebied liggen, bijvoorbeeld in een bos waar geen pad ligt. Er wordt dieper
ingegaan op de roadgraph in Sectie~\ref{sec:roadgraph}. Als laatste, maar
desalniettemin belangrijk punt moet de gebruiker binnenin de \ac{EPZ} de
kortste route volgen.~\cite{Dhondt_Pochat_Voulimeneas_Joosen_Volckaert_2022}

Dhondt et al.\ maakt nog een laatste assumptie over start- en eindpunten, die
hetzelfde moeten zijn. Dit is echter niet van toepassing op dit onderzoek. Het
onderzoek focust zich op activiteiten waar slechts één deel van het traject
gecloaked is. Dit wil dan ook zeggen dat de gebruiker ofwel vertrekt op de
gevoelige locatie, of er eindigt, maar niet beide. Op
Figuur~\ref{fig:totalDistanceAttack} zijn de 2 mogelijke scenarios van een
total distance attack terug te vinden, namelijk waarbij gestart wordt binnenin
de zone en wanneer erbinnen geëindigd wordt. Dit wordt ook een \textit{total
    distance attack} genoemd, omdat enkel de totale afstand en de afstand buiten de
\ac{EPZ} nodig is. Een andere aanval is de \textit{inner distance attack},
hierbij zullen zowel de start als het einde van een activiteit binnenin het te
verbergen gebied liggen. De kennis van de afzonderlijke afstand die de
gebruiker aflegt van de start tot de rand van de \ac{EPZ} en van de rand van de
\ac{EPZ} tot de eindlocatie is dan ook een vereiste. In
Sectie~\ref{sec:berekeningen} wordt dieper ingegaan op de reden waarom een
\textit{inner distance attack} niet mogelijk is. In deze thesis worden dus alle
activiteiten enkel een verhulde start- of eindlocatie behouden, de rest wordt
gefilterd in deze context.
\begin{figure}[h]
    \centering
    \begin{subfigure}[b]{.5\textwidth}
        \centering
        \caption{Start binnenin de \ac{EPZ}}
        \includegraphics[width=1\textwidth]{fig/TotalDistanceAttacks/start.png}
    \end{subfigure}\hfill
    \begin{subfigure}[b]{.5\textwidth}
        \centering
        \caption{Einde binnenin de \ac{EPZ}}
        \includegraphics[width=1\textwidth]{fig/TotalDistanceAttacks/end.png}
    \end{subfigure}
    \caption{Voorbeeld van de mogelijke scenarios bij een total distance attack}\label{fig:totalDistanceAttack}
\end{figure}

Deze thesis baseert zich ook voor een stuk op gemiddelde snelheden en tempo's.
Hierdoor stellen we volgende bijkomende assumptie voor: Een gebruiker mag niet
stilstaan binnenin de \ac{EPZ}. Platformen zoals Strava hebben namelijk een
ingebouwde functie die bij het uploaden van een activiteit tijden waarbij een
gebruiker stilstaat aan bijvoorbeeld een rood licht filtert. Zo kunnen ze een
meer representatieve gemiddelde snelheid en tempo berekenen en weergeven. Dit
wil wel zeggen dat de totale bewegingstijd waarop de gemiddelde snelheid en
tempo gebaseerd zijn, niet overeenkomt met de totale tijd van de activiteit.
Bij een berekening gebaseerd op totale verstreken tijd zou een significante
fout kunnen optreden.

\section{Identificeren van de EPZ}
De eerste hiervan is het identificeren van de \ac{EPZ}. Alhoewel deze stap niet
noodzakelijk is, vernauwt deze de zoekruimte drastisch. Hierbij worden van alle
activiteiten die van een gebruiker ter beschikking zijn gesteld, de zichtbare
begin- en eindpunten genomen. Deze zullen dan via k-means clustering worden
gegroepeerd opdat ze de zogenaamde \textit{entry gates} zullen aantonen.

K-means clustering is een unsupervised machine learning techniek die veel wordt
gebruikt bij het clusteren van data. Het is een iteratief proces waarbij het
algoritme $k$ clusters tracht te creëren waarbij de datapunten in elke cluster
zo dicht mogelijk bij het gemiddelde van die cluster
liggen~\cite{Understa24:online}. Dit algoritme kiest willekeurig initiële
middelpunten voor de verschillende clusters. Daarna worden alle punten in de
data toegekend aan de cluster met de laagste Euclidische afstand tot het
centrum van deze cluster. Daarna worden de gemiddeldes van deze clusters
herberekend, en worden deze gezien als nieuwe centrums. Opnieuw worden alle
punten aan de correcte cluster toegekend, en het proces wordt verschillende
iteraties herhaald tot een ietwat stabiele cirkel bekomen wordt. In de
implementatie van~\citeauthor{Dhondt_Pochat_Voulimeneas_Joosen_Volckaert_2022}
waarop deze thesis gebaseerd wordt, is een cirkel stabiel wanneer het verschil
in afstand tussen twee opeenvolgende gevonden cirkels kleiner is dan een
drempelwaarde, in dit geval 10
meter\cite{Dhondt_Pochat_Voulimeneas_Joosen_Volckaert_2022, Verdonck_2022}. Op
Figuur~\ref{fig:kmeans} is te zien hoe de clustering bij elke iteratie beter
wordt. In de context van het identificeren van de \ac{EPZ} zal het gebruikt
worden om \ac{gps}-punten te groeperen op basis van hun locaties. Punten die
dezelfde entry gate representeren, zullen in dezelfde cluster terecht komen.

\begin{figure}[h]
    \centering
    \begin{subfigure}[b]{.33\textwidth}
        \centering
        \includegraphics[width=1\textwidth]{fig/kmeans/1.png}
    \end{subfigure}\hfill
    \begin{subfigure}[b]{.33\textwidth}
        \centering
        \includegraphics[width=1\textwidth]{fig/kmeans/2.png}
    \end{subfigure}
    \begin{subfigure}[b]{.33\textwidth}
        \centering
        \includegraphics[width=1\textwidth]{fig/kmeans/3.png}
    \end{subfigure}
    \caption{Voorbeeld werking k-means clustering~\cite{InDepthk59:online}}\label{fig:kmeans}
\end{figure}

De besproken entry gates zijn zoals de naam al doet vermoeden de
`toegangspoorten' tot de cirkel. Dit is waar de gebruiker de \ac{EPZ} betreedt
en/of verlaat. Deze punten zouden dus in theorie de \ac{EPZ} perfect moeten
definiëren. Maar door de mogelijke fouten die standaard met het meten van
gps-punten komen\footnote{Gps-metingen bevatten standaard onnauwkeurigheden, er
    kan bouncing of signal loss voor een bepaalde interval optreden. Ook kan
    slechts op bepaalde tijdsintervallen de locatie worden genomen, perfect op de
    cirkel kan dus nooit gemeten worden.}. Op Figuur~\ref{fig:entrygate} is te zien
meerdere eindpunten van activiteiten geclusterd worden tot één \ac{E.G.} op de
figuur, voorgesteld door een kruis. Een cirkel kan worden gedefinieerd door
drie punten, bijgevolg moeten er dus ten minste drie \ac{E.G.} gevonden worden.
\begin{figure}
    \centering
    \includegraphics[width=0.5\linewidth]{fig/EPZ-mechanisme/Entry_Gate.png}
    \caption{Voorbeeld van entry gates gevonden door k-means clustering en identificatie van de \ac{EPZ}}\label{fig:entrygate}
\end{figure}

Het algoritme zal na de identificatie van de \ac{EPZ} ook nog nakijken of er
niet meer dan één \ac{EPZ} te vinden is. Er wordt onderzocht of punten die
meegenomen zijn in de beschouwing van de huidige \ac{EPZ}, toch niet horen bij
een mogelijke andere \ac{EPZ} van de user. Als controle wordt van elk eind- of
beginpunt de Euclidische afstand berekent tot de rand van de bijhorende
gevonden \ac{EPZ}. Indien deze kleiner is dan de grootst mogelijke radius, dan
wordt verondersteld dat het punt bij deze zone hoort. Indien dit voor alle
punten geldt, dan stopt de het algoritme hier. In het andere geval waarbij de
berekende afstand groter is, worden meer clusters toegevoegd aan het algoritme
van k-means clustering. Dit zal dus een nieuwe privacy zone aanwijzen.

Deze stap is niet noodzakelijk in het globale verhaal van de thesis, maar is
wel een stap die de zoekruimte erg kan verkleinen. Indien het algoritme één of
meerdere \acp{EPZ} vindt, dan zullen er enkel voorspellingen gebeuren in de
regio binnenin. Indien dit niet het geval is en er geen \ac{EPZ} gevonden is,
bestaat de kans dat voorspellingen van locaties gebeuren buiten de verhullende
zone. Ook is in dit geval een groter stuk van het stratenplan nodig om de
locatie te achterhalen.

\section{Voorspellen locatie}
Na de bepaling van de \acp{EPZ} voor de gebruiker wordt overgegaan tot het
voorspellen van kandidaten die de gevoelige locatie zouden kunnen zijn.
Hiervoor wordt verder ingegaan op de methodiek in de
paper~\citeauthor{Dhondt_Pochat_Voulimeneas_Joosen_Volckaert_2022}, maar er
worden enkele gegevens op een andere manier benadert volgens de huidige
definitie van de aanvaller.

\subsection{Roadgraph}\label{sec:roadgraph}
Voor elke gevonden EPZ is het noodzakelijk om een graafvoorstelling van de
omgeving op te stellen. Op Figuur~\ref{fig:graph_generation} is een voorbeeld
terug te vinden van hoe een graaf kan worden geëxtraheerd. Er worden punten
geplaatst op de straten op een vaste afstand van elkaar, en deze kunnen dan
worden verbonden. Indien geen \acp{EPZ} geïdentificeerd zijn, dan wordt de
omgeving die moet worden omgezet naar een graaf een stuk ruimer genomen. De
graafvoorstelling bestaat uit een serie van nodes, die zich allemaal op een
gekende straat bevinden. De bogen waarmee de nodes verbonden zijn volgen het
straatplan, opdat een node een mogelijks te volgen weg
is~\cite{neira2022graph}. Aan de hand van de `Chaining Distance' wordt bepaald
hoeveel afstand tussen de nodes zal zitten, en zo dus impliciet ook welke
densiteit het netwerk zal hebben. Hoe lager de densiteit, hoe meer nodes, en
dus ook hoe preciezer. Om voorspellingen te maken is wel een bepaalde precisie
vereist, dus mag deze waarde niet te hoog zijn. Empirisch werd gekozen voor een
waarde van $3.0m$.
\begin{figure}[h]
    \caption{Voorbeeld van het genereren van een roadgraph}\label{fig:graph_generation}
    \centering
    \begin{subfigure}[b]{.4\textwidth}
        \centering
        \includegraphics[width=1\textwidth]{fig/RoadGraph/RoadMap.png}
        \caption{Voorbeeld stratenplan}
    \end{subfigure}\hfill
    \begin{subfigure}[b]{.4\textwidth}
        \centering
        \includegraphics[width=1\textwidth]{fig/RoadGraph/Graph_Over_Map.png}
        \caption{Nodes en bogen geplot op het stratenplan}
    \end{subfigure}
    \begin{subfigure}[b]{.4\textwidth}
        \centering
        \includegraphics[width=1\textwidth]{fig/RoadGraph/Graph.png}
        \caption{Resulterende graafvoorstelling van het stratenplan}
    \end{subfigure}
\end{figure}

\subsection{Begin- en eindnodes}
Voor elke activiteit is het volledige traject buiten de \ac{EPZ} gegeven. Dit
omvat alle \ac{gps}-punten die niet verborgen zijn. De begin- en eindnodes van
het traject zijn hier van belang, voor de duidelijkheid en de vlotheid van de
tekst zullen we naar deze punten refereren als het zichtbare beginpunt en het
zichtbare eindpunt. Volgens één van de voorafgaand gemaakt assumptie vertrekt
of eindigt de sporter in de \ac{EPZ}. Dit betekent dat ofwel het reële
eindpunt, ofwel het reële beginpunt zal overeenstemmen met de gevoelige
locatie. In geval dat een gebruiker aankomt binnenin de \ac{EPZ}, en dus ook
vertrekt erbuiten, dan starten de berekeningen vanaf het zichtbare eindpunt. En
omgekeerd, indien hij vertrekt binnenin de \ac{EPZ}, dan worden de berekeningen
gestart vanaf het zichtbare beginpunt. Deze punten zullen in het vervolg
\textit{randpunten} genoemd worden, refererend naar de rand van de \ac{EPZ}.
Deze \textit{randpunten} zullen de basis vormen voor de volgende berekeningen.

Bijhorend zijn bij de randpunten ook bepaalde extra gegevens beschikbaar. De
belangrijkste zijn de cumulatieve afstand tot dit punt\footnote{De totale
    afstand afgelegd vanaf het begin van de activiteit tot en met het punt in
    kwestie.}, en de cumulatieve tijd tot dit punt\footnote{De totale afstand
    afgelegd vanaf het begin van de activiteit tot en met het punt in kwestie.}.
Bij de aanval van Dhondt et al.\ wordt de afstand gebruikt om predicties te
doen, dit wil dus zeggen dat deze afstand dus aan de basis zal liggen. Maar in
deze thesis wordt ervan uitgegaan dat afstanden verborgen worden. Onder het
verbergen van afstanden wordt een onderscheid gemaakt tussen 2 scenarios: het
eerste gaat ervan uit dat de totale afstand verborgen wordt, maar de
cumulatieve afstand gegeven is. Het tweede scenario gaat ervan uit dat alle
afstandsgegevens verborgen worden. Het alternatieve type data waar dus mee zal
moeten gewerkt worden is dus \ac{gps}-data.

\subsection{Berekeningen distance binnenin de EPZ}\label{sec:Berekeningen}
Om voorspellingen te kunnen doen zullen volgens de inferentie aanval die hier
besproken wordt twee belangrijke gegevens ter beschikking moeten zijn. Met name
het straatnetwerk met de mogelijks gevolgde routes, wat werd besproken in
Sectie~\ref{sec:roadgraph}, en de afstand die wordt afgelegd binnenin de
\ac{EPZ}. Deze afstand benoemen we ook als de \textit{inner distance}.

In de implementatie
van~\citeauthor{Dhondt_Pochat_Voulimeneas_Joosen_Volckaert_2022} kan de
\textit{inner distance} simpelweg berekent worden door het verschil te nemen
tussen de afgelegde afstand buiten de verhulde zone (deze noemen we de
\textit{outer distance}), en de totale afstand: \[inner\ distance = total\
    distance - outer\ distance \]

In deze thesis moet dit echter gebeuren met een tussenstap. In het eerste
scenario waarbij de cumulatieve afstand gegeven is, maar de totale afstand
niet, moet de totale afstand berekend worden. Maar door de aanwezigheid van
snelheids- en tijdsgegevens kan dit via basisformules gebeuren. Gebruik makend
van het gemiddelde tempo kan de voorgaande formule worden omgevormd tot: \[inner\ distance = total\ time \times average\ pace - outer\ distance \]

% Aanvullen referentie naar waarom geen inner distance attack mogelijk

% Basseren op totale verstreken tijd of niet?

% Welke gps afwijkingne zijn er\

\section{Selecteren van de beste voorspelling}

%  Beschrijven hoe zo'n aanval werkt - Karels methodiek uitwerken
% Afstanden wijken af
% Fouten in GPS data
% Fouten bij mappen GPS op het Strava routeplan
% Bij het aan en uitzetten van Strava - route verspringt!
% Afwijkende punten - zie grafiek

% Afstanden mogelijks berekenen op 2 manieren 
% 1. Coordinaten 
% 2. Cumulatieve afstanden van de punten

% Werkwijze
%     1. Bereken tijd in de EPZ
%           Wanneer ge stil staat... in de EPZ
%           Eerste - laatste tijd
%           Snelheid
%               Zelf berekenen (Maar komt overeen met degene gegeven door Strava - Strava geeft sommige weer als NULL, dus kan er niet mee werken)
%               Strava: m/h
%               Zelf: m/s
%     2. Berkenen outerdistance op verschillende manieren
%     2. Berekenen inner distance
%           Voor complete dataset => visualize verschillen tss de berekende en de gegeven inner distance
%           opm: Wanneer je stil staat in den EPZ => aanval niet mogelijk (Gemiddelde snelheid obv moving time <-> Mijn methode gebruikt elapsed_time)
%           haversine => Vectorized
%       Methodiek volledig uitschrijven + analyse van deze waarden

% Bepalen bij welke omstandigheden het werkt/niet werkt