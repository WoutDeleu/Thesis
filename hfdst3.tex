%%%%%%%%%%%%%%%%%%%%%%%%%%%%%%%%%%%%%%%%%%%%%%%%%%%%%%%%%%%%%%%%%%% 
%                                                                 %
%                            CHAPTER                              %
%                                                                 %
%%%%%%%%%%%%%%%%%%%%%%%%%%%%%%%%%%%%%%%%%%%%%%%%%%%%%%%%%%%%%%%%%%% 

\chapter{Setting aanval}
Gedurende dit hoofdstuk de setting alsook de werking van de aanval beschreven.
De aanval is sterk gebaseerd op de aanvallen
van~\citeauthor{Dhondt_Pochat_Voulimeneas_Joosen_Volckaert_2022}\cite{Dhondt_Pochat_Voulimeneas_Joosen_Volckaert_2022}
en~\citeauthor{Verdonck_2022}\cite{Verdonck_2022}. Deze aanvallen worden
inferentie-aanvallen genoemd, vanwege het feit dat uit metadata essentiële
gegevens kunnen worden geïnfereerd. In het geval
van~\citeauthor{Dhondt_Pochat_Voulimeneas_Joosen_Volckaert_2022} gaat dit over
afgelegde afstand binnenin de \ac{EPZ}. In het geval
van~\citeauthor{Verdonck_2022} gaat dit dan weer over geïnduceerde
hoogteverschillen binnen de privacy zone.

\section{Aanvaller}
Deze thesis voert een onderzoek naar de mogelijkheid om een \ac{EPZ} te
omzeilen. De studie wordt dus gevoerd vanuit het opzicht van een aanvaller.
Vooraleer de werking van een aanval wordt beschreven, is het belangrijk om een
zicht te hebben op het doel, en de capaciteiten van een aanvaller.

Hier is een aanvaller een gebruiker van het platform, die geen eigenaar is van
een activiteit. Hij heeft echter wel zicht op alle metadata die publiekelijk
gedeeld is. Dit is data zoals afgelegde afstand, snelheid, tempo, \ldots
Aangezien de aanval gaat over het omzeilen \acp{EPZ} worden activiteiten
beschouwd die gecloaked zijn. De aanvaller heeft dus geen zicht op de reële
start- en/of eindlocatie, zijn doel is dan ook om de ondanks de aanwezigheid
van cloaking deze gevoelige locatie te achterhalen.

Vanuit het oogpunt van de inferentie-aanval beschreven
door~\citeauthor{Dhondt_Pochat_Voulimeneas_Joosen_Volckaert_2022} heeft de
aanvaller toegang tot alle data die publiek beschikbaar is. Deze gebruikt dan
voornamelijk afgelegde weg als basis.

De aanvaller die in deze thesis wordt beschreven, heeft echter geen toegang tot
deze afstandsdata. Hij heeft wel nog toegang tot de ruwe GPS-data, maar ook de
snelheid, het tempo enzovoort. Het onderzoek bestaat er dus uit om te
onderzoeken in hoeverre een aanval nog mogelijk is wanneer de afstandsdata
onbruikbaar zou zijn. Een alternatieve aanpak wordt dus onderzocht om de
inferentie-aanval alsnog succesvol te kunnen uitvoeren.

\section{Inferentie aanval}\label{sec:inferentieaanval}
De inferentie aanval
van~\citeauthor{Dhondt_Pochat_Voulimeneas_Joosen_Volckaert_2022}, die de basis
vormt voor de aanval in deze thesis, kan worden in opgedeeld in drie stappen.
De eerste hiervan is het identificeren van de \ac{EPZ}. Alhoewel deze stap niet
noodzakelijk is, vernauwt deze de zoekruimte drastisch. Als inb
% Beschrijven : Uitgang van de aanval - Geen afstanden gegeven

%  Beschrijven hoe zo'n aanval werkt - Karels methodiek uitwerken
% Afstanden wijken af
% Fouten in GPS data
% Fouten bij mappen GPS op het Strava routeplan
% Bij het aan en uitzetten van Strava - route verspringt!
% Afwijkende punten - zie grafiek

% Afstanden mogelijks berekenen op 2 manieren 
% 1. Coordinaten 
% 2. Cumulatieve afstanden van de punten

% Werkwijze
%     1. Bereken tijd in de EPZ
%           Wanneer ge stil staat... in de EPZ
%           Eerste - laatste tijd
%           Snelheid
%               Zelf berekenen (Maar komt overeen met degene gegeven door Strava - Strava geeft sommige weer als NULL, dus kan er niet mee werken)
%               Strava: m/h
%               Zelf: m/s
%     2. Berkenen outerdistance op verschillende manieren
%     2. Berekenen inner distance
%           Voor complete dataset => visualize verschillen tss de berekende en de gegeven inner distance
%           opm: Wanneer je stil staat in den EPZ => aanval niet mogelijk (Gemiddelde snelheid obv moving time <-> Mijn methode gebruikt elapsed_time)
%           haversine => Vectorized
%       Methodiek volledig uitschrijven + analyse van deze waarden

% Bepalen bij welke omstandigheden het werkt/niet werkt