%%%%%%%%%%%%%%%%%%%%%%%%%%%%%%%%%%%%%%%%%%%%%%%%%%%%%%%%%%%%%%%%%%%%%%%%
%                                                                      %
% LaTeX, FIIW thesis template                                          %
% 
% Voor Nederlandse versie gebruik volgende lijnen code: 15,46,79,124   %
% For English Version use the following lines of code: 16,47,80,125    %
%%%%%%%%%%%%%%%%%%%%%%%%%%%%%%%%%%%%%%%%%%%%%%%%%%%%%%%%%%%%%%%%%%%%%%%%
\documentclass[11pt,a4paper]{report}
%% For english version
% Indien je je thesis recto-verso wil afdrukken gebruik je onderstaande opties i.p.v. bovenstaande
%\documentclass[11pt,a4paper,twoside,openright]{report}

\usepackage[a4paper,left=3.5cm, right=2.5cm, top=3.5cm, bottom=3.5cm]{geometry}
\usepackage{filecontents}
\usepackage{amsmath}                    % wiskundige formules
\usepackage{tikz}
\usepackage{booktabs}
\usepackage[dutch]{babel}
\usepackage{graphicx}                   % om niet ascii karakters rechtstreeks te kunnen inputten
\usepackage[utf8]{inputenc}             % commentarieer deze regel uit als je utf8 encoded files gebruikt in plaats van latin1
\usepackage[square, numbers]{natbib}    % Referentie opties [square, numbers] 
\usepackage{listings}             		% voor het weergeven van broncode
\usepackage{verbatim}					% weergeven van code, commando's, ...
\usepackage{hyperref}					% maak PDF van de thesis navigeerbaar
\usepackage{url}						% URL's invoegen in tekst met behulp van \url{http://}

\usepackage[small,bf,hang]{caption}     % om de captions wat te verbeteren
\usepackage{pdfpages}                   % gebruikt voor het invoegen van het artikel in pdf-formaat
\usepackage{pslatex}					% andere lettertype's dan de standaard types
\usepackage{lipsum}
\usepackage{sectsty}		            % aanpassen van de fonts van sections en chapters
\usepackage{subcaption}                 % subfiguren
%\usepackage[nottoc,numbib]{tocbibind}	% Bibliography mee in de ToC

\allsectionsfont{\sffamily}
\chapterfont{\raggedleft\sffamily}

\usepackage[graphicx]{realboxes}
\interfootnotelinepenalty=10000
\usepackage{float}                      % De optie H voor de plaatsing van figuren op de plaats waar je ze invoegt. bvb. \begin{figure}[H]
%\usepackage{longtable}					% tabellen die over meerdere pagina's gespreid worden
%\usepackage[times]{quotchap}           % indien je fancy hoofdstuktitels wil
%\usepackage[none]{hyphenat}
%\usepackage{latexsym}
%\usepackage{amsmath}
%\usepackage{amssymb}

\usepackage{acronym}

% MFA: zet zoekpad voor figure
\graphicspath{{fig/}}
\usepackage{fiiw} % Voor de nedelandse versie
%door onderstaande regels in commentaar te zetten, of op false, kan je pagina's weglaten
%bijvoorbeeld het weglaten van een voorwoord, lijst met symbolen, ...
%%%%%%%%%%%%%%%%%%%%%%%%%%%%%%%%%%%%%%%%%%%%%%%%%%%%%%%%%%%%%%%%%%%%%%%%%%%%%%%%%%%%%%%%
%voorwoord toevoegen?

\acknowledgementspagetrue{}
\acknowledgements{voorwoord}			%.tex file met daarin het voorwoord

%samenvatting toevoegen
\summarypagetrue{}
\summary{samenvatting}					%.tex met daarin de samenvatting

%abstract toevoegen?
\abstractpagetrue{}
\abstracts{abstract}					%.tex file met daarin het abstract
%lijst van figuren toevoegen?
\listoffigurespagetrue{}
%lijst van tabellen toevoegen?
\listoftablespagetrue{}
%lijst van symbolen toevoegen?
% \listofsymbolspagetrue{}
\listofsymbols{symbolen}				%.tex file met daarin de lijst van symbolen
%lijst van afkortingen toevoegen?
\listofabbrevspagetrue{}
\listofabbrevs{afkortingen}				%.tex file met daarin de lijst van symbolen

%informatie over het eindwerk, de promotor, ...
%%%%%%%%%%%%%%%%%%%%%%%%%%%%%%%%%%%%%%%%%%%%%%%
\opleiding{Elektronica/ICT}
\afdeling{Optie Smart Applications}
\campus{gent} % Voor de Nederlandse versie van campus de nayer. Voor andere campussen gebuirk geel,gent,groept of brugge.

\title{Time Is Running Out}
\subtitle{Assessing Temporal Privacy of Privacy Zones in Fitness Tracking Social Networks}
% \author{naam student}
\forenameA{Wout}
\surnameA{Deleu}

\forenameB{}
\surnameB{}

\academicyear{2022 - 2023}

\promotorA[Promotor]{Prof.\ dr.\ ir.\ Stijn Volckaert}
\promotorB[Begeleiders]{Ing. Karel Dhondt, \\Ing. Alicia Andries \\Ing. Jonas Vinck}

% URL's bibliography overflow prevention
\def\UrlBreaks{\do\/\do-}

\setlength{\parskip}{16pt}
\begin{document}
\selectlanguage{dutch}
\preface{}

%%%%%%%%%%%%%%%%%%%%%%%%%%%%%%%%%%%%%%%%%%%%%%%%%%%%%%%%%%%%%%%%%%% 
%                                                                 %
%                            CHAPTER                              %
%                                                                 %
%%%%%%%%%%%%%%%%%%%%%%%%%%%%%%%%%%%%%%%%%%%%%%%%%%%%%%%%%%%%%%%%%%% 

\chapter{Inleiding}

\section{Situering}
Sociale media is zo goed als niet meer weg te denken uit het huidige moderne
leven. Over de jaren heen zijn er verschillende definities aan het concep
sociale media gegeven. In het werk van Howard en Park wordt sociale media
gedefinieerd als de infrastructuur en tools om content te maken en te
verspreiden~\cite{PhilipsAndParks}. Deze definitie is erg ruim, en vertakt zich
dus in heel wat facetten, waaronder sociale netwerken, media sharing networks,
etc, maar ook de tak van de fitnesstrackers. Deze opkomst van nieuwe media
brengt echter ook onbedoelde maar significante privacy bezorgdheden met zich
mee.

De focus in deze dissertatie ligt op privacy binnen fitnesstrackers, meer
specifiek platformen die \ac{gps}-locaties gebruiken, zoals Strava, Nike Run
Club, etc. Dit zijn platformen waar personen sportactiviteiten zoals lopen,
fietsen, wandelen,\ldots kunnen delen met elkaar. Het algemene concept is
hierbij dat wanneer je een sportactiviteit uitvoert, je deze voor je volgers en
vrienden beschikbaar maakt. De sportactiviteiten zullen natuurlijk bepaalde
gegevens bevatten die zichtbaar zijn voor die andere gebruikers.
Figuur~\ref{fig:activityExample} geeft bijvoorbeeld weer hoe Strava de afstand,
bewegingstijd, en natuurlijk de \ac{gps}-locaties deelt. Vele van deze gegevens
hebben direct of indirect een negatieve impact op de privacy van de user. Deze
negatieve gevolgen komen dan vooral in de vorm van het onbedoeld vrijgeven van
\textit{gevoelige locaties}. Onder het concept van een gevoelige locatie vallen
heel wat beschrijvingen. Een algemene beschrijving kan zijn, een locatie die
geografische informatie deelt die negatieve gevolgen kan hebben, en die je dus
liever niet deelt. In het kader van dit onderzoek zal dit gaan over start en
eindlocaties van activiteiten. Dit kan gaan over woonplaatsen, wat kan leiden
tot o.a.\ stalking, alsook locaties waar sportmateriaal wordt opgeborgen, wat
eventueel zou kunnen leiden tot diefstal. Er zijn gevallen bekend van
fietsdieven die Strava gebruiken om fietsen te kunnen
lokaliseren~\cite{Sportapp72:online}~\cite{Cyclistw89:online}. Grootschaligere
voorbeelden die zeker het vermelden waard zijn, zijn de gevallen waarbij
geheime militaire basissen ontdekt worden door het bestuderen van de
heatmap\footnote{Een heatmap is een visuele weergave van gegevens waarbij
    verschillende kleuren worden gebruikt om de intensiteit van waarden in een
    matrixachtige structuur weer te geven. In de context van GPS-locaties kan een
    heatmap worden gebruikt om de concentratie of frequentie van locaties op een
    kaart weer te geven~\cite{Whatishe21:online}. Het doel is om gebieden met een
    hoge dichtheid of veelvuldige locaties te identificeren en visueel te
    markeren.} vrijgegeven door Strava~\cite{Fitnesst33:online}.
Figuur~\ref{fig:heatmap} toont een voorbeeld van de heatmap van Strava.
\begin{figure}
    \centering
    \includegraphics[width=\textwidth]{fig/Heatmap_strava.png}
    \caption{Voorbeeld Heatmap, vrijgegeven door Strava~\cite{StravaGl10:online}}\label{fig:heatmap}
\end{figure}
\begin{figure}
    \centering
    \includegraphics[width=0.5\linewidth]{fig/VoorbeeldActiviteiten/VoorbeeldActiviteit_Cropped.png}
    \caption{Voorbeeldactiviteit Strava}\label{fig:activityExample}
\end{figure}

Fitnesstrackers implementeren elk manieren om de privacy van de users te
verbeteren. De meest eenvoudige te bedenken manier is misschien wel de
mogelijkheid om activiteiten te verbergen voor een selectie van personen (bv.\
iedereen die geen volger is). Zo kunnen enkel de mensen die de gebruiker
expliciet toelaat activiteiten bekijken. Een complexer alternatief is het
gebruik van \acp{EPZ}. Hierbij wordt de weergegeven route voor de persoon die
meekijkt gedeeltelijk verborgen. Er wordt als het ware een deel van de route
afgekapt. De echte begin- en eindpunten zullen binnenin het afgekapte deel
liggen. Er zullen nieuwe punten worden gegenereerd, op de rand van de cirkel,
die voor de externe waarnemer het begin en einde zullen voorstellen. Het begin-
en eind-deel van de route wordt dus onzichtbaar voor de andere gebruikers. Door
de aanwezigheid van al deze pogingen tot privacyverbeteringen valt op dat de
ontwikkelaars van de platformen erg bewust zijn van de mogelijke gevaren.
Echter is er een afweging te maken bij de implementatie tussen de bruikbaarheid
van het platform, en de privacy van de eindgebruiker. Hoe meer data wordt
vrijgegeven, hoe groter de kans op mogelijk gevoelige info wordt meegegeven.
Aan de andere kant, bij het weglaten van informatie gaat de
gebruiksvriendelijkheid en de aanwezigheid van nuttige data van het platform er
sterk op achteruit.

\section{Doelstelling}
In dit onderzoek bekijken we of er een mogelijkheid bestaat om private locaties
(verborgen start- en eindlocaties) van een activiteiten te achterhalen, ondanks
het gebruik van de \ac{EPZ} beschreven in Sectie~\ref{sec:EPZ} als privacy
beveiligingsmechanisme. In het verleden werden enkele manieren beschreven om
a.d.h.v.\ andere metadata zoals hoogtedata en afstanden de \ac{EPZ} te omzeilen
(\textit{~\cite{Dhondt},~\cite{Verdonck_2022}}).
In deze thesis wordt meer in detail gegaan op het gebruik van snelheidsdata.
Als basis voor deze aanval gebruiken we de inferentie aanval op de EPZ
van~\citeauthor{Dhondt}. We
onderzoeken of deze aanval nog steeds mogelijk is bij het weglaten van bepaalde
gegevens, en dus door het gebruik van andere gegevens. De focus ligt in deze
studie voornamelijk op snelheidsdata.

Om deze doelstelling te behalen is eerst een berekeningsmechanisme nodig voor
de afstanden die nodig zijn om de inferentie-aanval te kunnen uitvoeren. Daarna
voeren we een analyse uit op het verschil tussen de berekende afstanden, en de
waarden afgeleid volgens de berekeningen
van~\citeauthor{Dhondt}. Zo kunnen we
de effectiviteit van de aanval a priori schatten. Er is een analyse van de
beschikbare data, en een bespreking en reflectie over de resultaten van de
aanval.
%%%%%%%%%%%%%%%%%%%%%%%%%%%%%%%%%%%%%%%%%%%%%%%%%%%%%%%%%%%%%%%%%%% 
%                                                                 %
%                            CHAPTER                              %
%                                                                 %
%%%%%%%%%%%%%%%%%%%%%%%%%%%%%%%%%%%%%%%%%%%%%%%%%%%%%%%%%%%%%%%%%%% 

\chapter{Achtergrond}

\section{Fitnesstrackers}
Zoals al enkele malen werd aangehaald, ligt de focus van deze scriptie op
mogelijke tekortkomingen/vulnerabilities betreffende privacybeleid in
fitnesstrackers. Maar voordat een aanval op basis van deze kwetsbaarheden kan
opgezet worden, is het noodzakelijk om een te vat te krijgen op welke manier
een fitnesstracker info verzamelt en weergeeft. En meer precies, hoe de
mechanismen die de privacy voorzien voor de gebruikers in detail werken.

De data waarmee de aanval wordt opgezet en waarmee wordt geëxperimenteerd, is
afkomstig van de populaire fitnesstracker
\textit{Strava\footnote{\url{https://www.strava.com/}}}. Dit is een sociaal
netwerk, waarbij alle soorten sporters hun activiteiten kunnen delen. Dit gaat
over lopen, wandelen, fietsen, zwemmen, \ldots, maar ook sporten als fitnessen,
voetballen, \ldots De verzamelde data wordt volgens het perspectief van een
mogelijke aanval gefilterd. Enkel data die gevoelige informatie met betrekking
tot woonplaats zou kunnen vrijgeven wordt behouden. Dit zal er dus op neerkomen
dat enkel activiteiten die relevante gps-informatie bevatten in beschouwing
worden genomen. Dit gaat dan meer specifiek over \textit{runs, hikes, walks,
    and rides}.

\subsection{Activiteiten}\label{data}
Een Strava activiteit bevat erg veel informatie. Echter is niet alles even
bruikbaar. Een correcte abstractie van de onnodige data is dus nodig.
Figuur~\ref{fig:activityData} geeft een voorbeeld van een gedetailleerde
activiteit weer. Een gebruiker is in staat om de activiteit een titel te geven,
en er een korte beschrijving aan toe te voegen. Ook een foto kan optioneel
toegevoegd worden. De exacte datum en tijd van de start van de activiteit wordt
hierbij ook weergegeven.

Rechts daarvan zijn de algemene basisstatistieken te zien. Deze zijn de totale
afgelegde afstand, de totale bewegingstijd, de gemiddelde snelheid, het totale
hoogteverschil, de totale verstreken tijden, en het aantal calorieën verbrand.
Als extra kunnen hier enkele statistieken m.b.t.\ het gebruikte materiaal,
zoals type fiets, loopschoenen, hartslagmeter, enzovoort worden weergegeven.
Een belangrijk onderscheid in deze context is het verschil tussen de beweegtijd
en de verstreken tijd. Deze twee lijken in definitie gelijk, maar dit zijn ze
niet. Strava, en vaak fitnessplatformen in het algemeen werken met twee
verschillende soorten tijdsberekeningen voor het bekomen van een accuratere
gemiddelde snelheid. De verstreken tijd is simpelweg het tijdsinterval tussen
het vertrek van de activiteit en de aankomsttijd ervan. De bewegingstijd is de
tijd waarbij de gebruiker zich effectief bewoog. Met andere woorden worden de
tijden waarbij de gebruiker stilstond uit de verstreken tijd gefilterd. Dit kan
gaan over bijvoorbeeld een pauze, of het wachten voor een verkeerslicht. De
snelheid wordt berekend aan de hand van de bewegingstijd. Dit kan simpel worden
geverifieerd via een manuele berekening volgens de formule voor het berekenen
van gemiddelde snelheid\footnote{$ \quad v(\frac{min}{km}) =
        \frac{t(\min)}{d(km)}$}, met de data die terug te vinden is op
Figuur~\ref{fig:activityData} ($\frac{(39:17)\min}{7.44 km} = 5:16
\frac{\min}{km}$). Een kanttekening hierbij is dat dit enkel geldt voor
activiteiten die niet gelabeld zijn als \textit{race}, dan wordt de snelheid
berekend in functie van de totaal verstreken tijd~\cite{MovingTi80:online}.

Onder de basisstatistieken zijn de \textit{Strava-segmenten} te zien. Een
Strava-segment is een specifiek deel van een bepaalde route dat door gebruikers
van de sport-app kan worden gemarkeerd, gedeeld en vergeleken met andere
gebruikers. Het segment is een bepaalde afstand en route, bijvoorbeeld een klim
of afdaling, die vaak wordt beschouwd als een uitdagende of iconische sectie
van een bepaalde fiets- of hardlooproute. Gebruikers van Strava kunnen een
segment maken door de begin- en eindpunten op een kaart aan te geven en een
naam en beschrijving toe te voegen. Zodra het segment is gemaakt, kunnen andere
gebruikers het segment vinden en deelnemen aan een leaderboard, waarop de
snelste tijden worden bijgehouden en vergeleken met andere gebruikers.
Segmenten worden vaak gebruikt om prestaties te meten en te vergelijken.

Centraal op de figuur is ook de kaart duidelijk zichtbaar. Daarbij horen ook de
tussentijden en de grafiek van snelheid. Optioneel kan hierbij ook nog een
visualisatie van de afgelegde hoogte en de hartslag worden weergegeven, indien
de gebruiker hiervoor met de juiste meetinstrumenten zijn sportactiviteit
opneemt. De tussentijden en de grafiek van snelheid zijn qua inhoud
gelijkaardig, met als verschil dat deze erg precies kan worden bestudeerd. Op
de grafiek is voor elk afstandspunt de ogenblikkelijke snelheid zichtbaar. Bij
de tussentijden wordt de gemiddelde snelheid over een kilometer weergegeven. De
kaart die de route weergeeft is zeker ook belangrijk om even te bestuderen.
Deze bevat namelijk alle gps-geregistreerde punten, en verbindt deze ook om zo
één aaneensluitende route te vormen. Wanneer deze echter in detail bestudeerd
wordt, samen met de legende die aanwezig is, is te zien dat de route uit twee
delen bestaat, een zichtbaar deel en een onzichtbaar deel. Een andere gebruiker
zal enkel zicht hebben op de het zichtbare deel, het onzichtbare deel zal dus
voor een andere gebruiker niet zichtbaar zijn. Anders geformuleerd, de
activiteit zal voor deze persoon dus als het ware afgekapt zijn, en zal in zijn
zichtbare versie op een andere plek starten en eindigen. In de volgende
Secties~\ref{Algemene Privacy} \&~\ref{EPZ} wordt meer in detail ingegaan op de
werking van deze methodiek.

Een laatste kanttekening die hierbij gemaakt moet worden, is dat voor een
gebruiker verschillende eenheden mogelijk zijn om uit te kiezen. Er is keuze
mogelijk tussen de mijl en pond, en kilometer en kilogram. Gebruikers kiezen in
welke eenheid ze de applicatie wensen te gebruiken. Voor de gebruiker in
kwestie zal dus de volledige applicatie worden weergegeven in de gekozen
eenheden.
\begin{figure}
    \centering
    \includegraphics[width=0.8\textwidth]{fig/VoorbeeldActiviteiten/VoorbeeldActiviteit_Personal.png}
    \caption{Data van een activiteit}\label{fig:activityData}
\end{figure}

\subsection{Berekening Afstanden}
Fitnesstrackers krijgen vanuit de buitenwereld ruwe data binnen. Deze data moet
dus verwerkt worden vooraleer ze bruikbaar is voor de gebruiker. Er werd al
kort ingegaan in Sectie~\ref{data} op de berekening die Strava gebruikt voor de
snelheid. Echter is het ook interessant om de berekening van Strava eens onder
de loep te nemen voor de afgelegde afstand. Strava maakt gebruik van twee
verschillende methodieken voor het berekenen van deze afstand. De eerste is de
\textit{GPS-calculated Distance}. Dit bestaat eruit om de afstand tussen
opeenvolgende gps-punten te berekenen, en deze op te tellen. Precisie is hier
afhankelijk van de precisie van de gps-punten, aangezien de afstand wordt
berekend door de punten met rechte lijnen te verbinden. Dit kan gebeuren in
real time, via de gsm, smartwatch of ander toestel die gebruikt wordt om de
activiteit op te nemen. Er zal dan ook mogelijkheid zijn om real time info te
zien. Op elk punt zal de afstand vanaf het startpunt gekend zijn, en het is
deze afstand die gedeeld zal worden op het platform. Het grote nadeel hierbij
is het real-time aspect. Fouten kunnen moeilijker on the fly worden
gecorrigeerd. Een tweede aanpak is om gps-data pas bij het uploaden te
verwerken. De gps-data wordt dan geanalyseerd, en de nodige berekeningen worden
uitgevoerd.

Het alternatief voor de GPS-calculated distance is de \textit{Ground Speed
    Distance} methodiek. Deze afstand kan enkel worden bepaald in het geval van een
fietsactiviteit. Deze afstand wordt berekend door het aantal omwentelingen te
vermenigvuldigen met de omtrek van het fietswiel~\cite{HowDista47:online}.

De bovenstaande afstandsberekeningen zijn de 2 technieken die officiële support
documentatie van Strava beschrijft~\cite{HowDista47:online}. Echter blijkt
wanneer de afstand op deze manier manueel berekent worden, afwijkende
resultaten bekomen worden. Dit is zeer waarschijnlijk te wijten aan de
preprocessing van de data die gebeurd bij het uploaden van een activiteit.
Alhoewel dit niet expliciet gedocumenteerd staat doen de resultaten dit wel
sterk vermoeden. De hypothese is dat tijdens het uploaden, de afstand
herberekend wordt. De gps-punten zullen worden geanalyseerd, en er zullen
technieken worden gebruikt om de resultaten hiervan te verbeteren. De twee
meest waarschijnlijke technieken zijn \textit{Map Snapping} en
\textit{Smoothing}.

Map Snapping of Snap to Roads is een techniek waarbij gps-punten worden
verschoven naar de dichtstbijzijnde weg. Per gps-punt wordt dan gezocht naar de
dichtste node op de desbetreffende \textit{roadgraph}\footnote{De roadgraph is
    afhankelijk van welke implementatie gebruikt wordt voor het snappen. Het is een
    wegennetwerk, omgezet in een graaf, bestaande uit edges en nodes. Elke weg of
    pad, bevat een of meerdere nodes, zodat een skeletstructuur ontstaat, die een
    abstractie van het wegennetwerk voorstelt~\cite{seiler2022haul}.} (op
Figuur~\ref{fig:MapSnapping} is de werking ervan te
zien)\cite{Snapping96:online}.
\begin{figure}[h]
    \centering
    \begin{subfigure}[b]{.5\textwidth}
        \centering
        \caption{Voorbeeldroute zonder map snapping}
        \includegraphics[width=0.5\textwidth]{fig/Map Snapping/before.png}\label{fig:before_MapSnapping}
    \end{subfigure}\hfill
    \begin{subfigure}[b]{.5\textwidth}
        \centering
        \caption{Voorbeeldroute met map snapping}
        \includegraphics[width=0.5\textwidth]{fig/Map Snapping/after.png}\label{fig:after_MapSnapping}
    \end{subfigure}
    \caption{Voorbeeld van de werking van een EPZ}\label{fig:MapSnapping}
\end{figure}

Daarnaast bestaat de kans dat er gebruik gemaakt wordt van smoothening.
Smoothening is een proces dat ruwe gps-punten (of datapunten in het algemeen)
op een traject probeert te optimaliseren opdat ze een vloeiend `curve' vormen.
Dit wordt bekomen door ruis, schommelingen en onnauwkeurigheden te filteren uit
het traject. Hiervoor bestaan verschillende implementaties. Aangezien Strava
geen openbare informatie verstrekt over het gebruik van GPS-smoothing, is het
niet bekend of ze deze techniek effectief toepassen. Het is dus gissen naar,
indien ze deze zouden gebruiken, welke implementatie dan wel gebruikt wordt. De
makkelijkste en meest modulaire methode om aan smoothening te doen, is
\textit{Smoothing met Moving Average}. Deze methode bestaat eruit om van een
aantal punten in een bepaalde range (ook `window' genoemd) het gemiddelde te
nemen, en vervolgens op te schuiven. Het gemiddelde wordt berekend met volgende
formule: $\overline{y_x} = \frac{y_x + y_{x+1} + \ldots + y_{n}}{x+n}$, voor
punt x, met n als window-grootte. Zo kan voor elk punt een evenwichtige waarde
op de nieuwe grafiek bekomen worden, en krijgt de grafiek een meer vloeiende
vorm. Merk wel op dat de precisie van de route daalt wordt op deze manier. Bij
het smoothen van een traject wordt het aantal gebruikte punten namelijk
vermindert volgens de grote van de window. Afhankelijk van de grote, worden
meer (resp.\ minder) punten samengenomen, en zo minder/meer punten weergegeven
op de grafiek. Een voorbeeld is terug te vinden op
figuur~\ref{fig:SmoothingExample}
~\cite{Smoothin16:online}\cite{SmoothingandInterpolatingNoisyGPSDatawithSmoothingSplines}\cite{Smoothin86:online}.
\begin{figure}[h]
    \centering
    \includegraphics[width=0.6\linewidth]{fig/SmoothingExample.png}
    \caption{Voorbeeld Data smoothing with moving average}\label{fig:SmoothingExample}
\end{figure}
% - https://support.strava.com/hc/en-us/articles/216917707-Bad-GPS-Data

\subsection{Algemeen Privacybeleid}\label{Algemene Privacy}
Het delen van alle data die vervat zit in zo'n activitei met alle andere
gebruikers op het platform, is zeker niet altijd wenselijk. De ontwikkelaars
kiezen er dan ook voor om gebruikers de mogelijkheid te geven om hun privacy te
bewaren. In deze sectie wordt de focus gelegd op de mechanismen gebruikt door
\textit{Strava}. Als opmerking valt te melden dat in heel wat andere
sport-applicaties worden vergelijkbare, zo niet dezelfde methodieken gebruikt.
Een eerste algemeen mechanisme bestaat eruit om de gebruiker de keuze te geven
om alle activiteiten en alle gegevens over het profiel heen te laten voldoen
aan bepaalde privacy regels. Deze regels kunnen ook per activiteit worden
ingesteld. Onder de keuzes staan meestal drie opties, \textit{zichtbaar voor
    iedereen}, \textit{zichtbaar voor volgers} en \textit{zichtbaar voor niemand}.
Er kan ook zelf een keuze gemaakt worden om specifieke elementen van een
activiteit niet te delen met de buitenwereld, zoals bijvoorbeeld de
zichtbaarheid van de kaart die de route weergeeft.\cite{Activity24:online}

\section{Endpoint Privacy Zones}\label{EPZ}
Een tweede belangrijke maatregel is het gebruik van de de \textbf{Strava
    Endpoint Privacy Zones (EPZ)}. Een EPZ is een cirkelzone met een bepaalde
straal rond een gps-punt. Het punt in kwestie zal dus de betreffende
\textit{gevoelige locatie} zijn. De straal van deze cirkel\footnote{Op Strava
    heeft de EPZ de vorm van een cirkel, maar op andere platformen kunnen andere
    vormen de norm zijn, bv.\ polygonen.} kan worden gekozen door de gebruiker, en
in het geval van Strava hebben gebruikers keuze uit waarden van 0 tot 1600m, in
stappen van 200m. Wanneer een gebruiker binnen deze zone zijn activiteit
beëindigt of begint, dan zal dat deel van de route binnen de EPZ niet zichtbaar
zijn voor anderen. Vanuit het perspectief van een andere gebruiker zal de
activiteit dus starten/eindigen op de rand van deze cirkel (die natuurlijk niet
zichtbaar is). Merk op dat een sporter ook andere gevoelige locaties kan
verbergen op de kaart. Bijvoorbeeld een frequent bezocht café, of een huis van
een partner waar regelmatig een tussenstop plaatsvindt. Een tweede opmerking is
dat wanneer een gebruiker de EPZ doorkruist, maar er niet in stopt, de route
onaangepast blijft. Op Figuur~\ref{fig:EPZ_Voorbeeld} zijn de verschillende
perspectieven te zien, hoe het er als uploader uitziet, en hoe het eruit ziet
voor een andere gebruiker. Het traject die de buitenstaander te zien krijgt,
zijn alle punten die zich buiten de EPZ bevinden. Merk ook op dat de eigenaar
van de activiteit zicht heeft op de EPZ, en wat zal verborgen worden die zich
buiten de EPZ bevinden. Dit onderscheid wordt gemaakt door het verschil in
kleur (oranje voor de publiek zichtbare punten en grijs voor de onzichtbare).
\begin{figure}[h]
    \centering
    \begin{subfigure}[b]{.7\textwidth}
        \centering
        \caption{Perspectief eigenaar}
        \includegraphics[width=1\textwidth]{fig/EPZ-mechanisme/Example_EPZ_InternalView.png}\label{fig:EPZ_internal}
    \end{subfigure}\hfill
    \begin{subfigure}[b]{.7\textwidth}
        \centering
        \caption{Perspectief externe gebruiker}
        \includegraphics[width=1\textwidth]{fig/EPZ-mechanisme/Example_EPZ_ExternalView.png}\label{fig:EPZ_external}
    \end{subfigure}
    \caption{Voorbeeld van de werking van een EPZ}\label{fig:EPZ_Voorbeeld}
\end{figure}

Het opzetten van een EPZ is dus een belangrijk onderdeel bij het blootleggen
van mogelijke zwakheden van dit systeem. Bij dit proces zal de gevoelige
locatie worden genomen als beginlocatie. Hieruit zal a.d.h.v.\ de op voorhand
vastgelegde EPZ-straal een cirkel worden opgesteld. Het centrum van deze cirkel
zal hierna een translatie ondervinden in een willekeurige richting. Dit kan een
verschuiving zijn met een afstand die maximaal 70\% van de straal van de EPZ
bedraagt (Figuur~\ref{fig:translation}). Het transleren van deze cirkel wordt
ook \textit{spatial cloaking} genoemd.
\begin{figure}[h]
    \centering
    \includegraphics[width=0.4\linewidth]{fig/EPZ-mechanisme/Translation_Center.png}
    \caption{Voorbeeld translatie EPZ}\label{fig:translation}
\end{figure}

Daarna worden alle punten vertrekkende vanaf de gevoelige locatie tot aan de
rand van de EPZ, en vanaf de rand van de EPZ tot aan de gevoelige locatie
verwijdert van het zichtbare traject. Merk op dat punten die de EPZ
doorkruisen, maar niet vertrekken/aankomen bij de gevoelige locatie niet worden
gefilterd (Figuur~\ref{fig:drop points}).
\begin{figure}[h]
    \centering
    \includegraphics[width=0.7\linewidth]{fig/EPZ-mechanisme/DropEPZPoints.png}
    \caption{Voorbeeld filtering van punten binnen EPZ}\label{fig:drop points}
\end{figure}

\section{Gerelateerd werken}
In het verleden is al wat onderzoek verricht in de richting van de
doeltreffendheid van EPZ's bij fitnesstrackers.~\citeauthor{sec18has3:online}
beschreef een implementatie van EPZ's waarbij het centrum van de zone de
gevoelige locatie is. M.a.w.\ het identificeren van deze zone is dus voldoende
om de gevoelige locatie te achterhalen\cite{sec18has3:online}. In tegenstelling
tot dit onderzoek, wordt ervan uitgegaan dat het centrum geen translatie
ondervindt, en er dus geen spatial cloaking wordt toegepast. In deze paper
wordt gefocust op de reconstructie van de cirkel op basis van 3 punten op de
rand (Figuur~\ref{fig:Hassan_EPZ}). Deze 3 randpunten worden dus bekomen door
begin/eindpunten te nemen van activiteiten, volgens het perspectief van
gebruiker die geen eigenaar is. Deze begin/eindpunten zullen zich altijd op de
rand van de cirkel begeven. In deze paper wordt spatial cloaking wel aangehaald
als mogelijke countermeasure tegen dit soort aanvallen.
\begin{figure}[h]
    \centering
    \includegraphics[width=0.6\textwidth]{fig/EPZ-mechanisme/Hassan.png}
    \caption{Mechanisme EPZ beschreven door \citeauthor{sec18has3:online}}\label{fig:Hassan_EPZ}
\end{figure}

Een onderzoek door~\citeauthor{10.1145/3491102.3502136} toonde ook aan dat
intuïtief heel wat mensen in staat zijn om de gevoelige locatie te achterhalen.
Dit gebeurde op basis van enquêtes die werden afgenomen bij gebruikers van het
platform. Uit het onderzoek bleek dat 68\% van de ondervraagden bij een
EPZ-radius van 200m de beschermde locatie tot op 50m nauwkeurig konden
voorspellen. Deze resultaten op zich zijn alarmerend, en tonen aan dat EPZ's
verre van perfect zijn.

\citeauthor{Dhondt_Pochat_Voulimeneas_Joosen_Volckaert_2022} voerde ook een studie naar de lekken aanwezig in het principe van EPZ's. Er
wordt in deze paper een nadruk gelegd op de translatie van de EPZ, en hoe deze
de privacy van een gebruiker verhoogt. Een inferentie aanval wordt er
beschreven die gebruikmaakt van de totale afstand, terug te vinden bij de
activiteit. Aan de hand van deze totale afstand in combinatie met het
wegennetwerk, wordt een poging gedaan om alle mogelijke routes die de sporter binnenin de EPZ zou kunnen afgelegd hebben, voor elk traject te
reconstrueren. Wanneer dit gedaan wordt voor verschillende trajecten, kan een
locatie voorspeld worden die het meest waarschijnlijk wordt geacht om de
gevoelige locatie te zijn.

Een laatste onderzoek die zeker ook het vermelden waard is, is de thesis
van~\citeauthor{Verdonck_2022} et al. Deze thesis bouwt in grote mate verder op
de paper van \citeauthor{Dhondt_Pochat_Voulimeneas_Joosen_Volckaert_2022}, maar
er wordt alternatieve data gebruikt. Er wordt gewerkt met hoogtedata i.p.v.\
totale afstanden, en zo wordt ook een inferentie aanval geconstrueerd.
% https://labs.strava.com/slide/
% https://www.jamesrcroft.com/2015/06/snapping-gps-tracks-to-roads/

% Beter uitwerken EPZ + literatuur achtergrond
% Beter uitwerken attack model
% 2.2 heruitwerken
%%%%%%%%%%%%%%%%%%%%%%%%%%%%%%%%%%%%%%%%%%%%%%%%%%%%%%%%%%%%%%%%%%% 
%                                                                 %
%                            CHAPTER                              %
%                                                                 %
%%%%%%%%%%%%%%%%%%%%%%%%%%%%%%%%%%%%%%%%%%%%%%%%%%%%%%%%%%%%%%%%%%% 

\chapter{Setting aanval}\label{sec:inferentieaanval}
Gedurende dit hoofdstuk wordt de setting alsook de werking van de aanval
beschreven. De aanval is sterk gebaseerd op de aanvallen
van~\citeauthor{Dhondt_Pochat_Voulimeneas_Joosen_Volckaert_2022}\cite{Dhondt_Pochat_Voulimeneas_Joosen_Volckaert_2022}
en~\citeauthor{Verdonck_2022}\cite{Verdonck_2022}. Deze aanvallen worden
inferentie-aanvallen genoemd, vanwege het feit dat uit metadata essentiële
gegevens kunnen worden geïnfereerd. In het geval
van~\citeauthor{Dhondt_Pochat_Voulimeneas_Joosen_Volckaert_2022} gaat dit over
afgelegde afstand binnenin de \ac{EPZ}. In het geval
van~\citeauthor{Verdonck_2022} gaat dit dan weer over geïnduceerde
hoogteverschillen binnen de privacy zone. Allereerst zal kort de mogelijkheden
van een aanvaller in de huidige setting worden besproken. Daarna wordt de
inferentie aanval
van~\citeauthor{Dhondt_Pochat_Voulimeneas_Joosen_Volckaert_2022}, die de basis
vormt voor de aanval in deze thesis, besproken volgens een opdeling in drie
stappen.

\section{Definitie aanvaller}
Deze thesis voert een onderzoek naar de mogelijkheid om een \ac{EPZ} te
omzeilen. De studie wordt dus gevoerd vanuit het opzicht van een aanvaller.
Vooraleer de werking van een aanval wordt beschreven, is het belangrijk om een
zicht te hebben op het doel, en de capaciteiten van een aanvaller.

Hier is een aanvaller een gebruiker van het platform, die geen eigenaar is van
een activiteit. Hij heeft echter wel zicht op alle metadata die publiekelijk
gedeeld zijn. Dit is data zoals afgelegde afstand, snelheid, tempo, \ldots
Aangezien de aanval gaat over het omzeilen \acp{EPZ} worden activiteiten
beschouwd die gecloaked zijn. De aanvaller heeft dus geen zicht op de reële
start- en/of eindlocatie, zijn doel is dan ook om ondanks de aanwezigheid van
cloaking deze gevoelige locatie te achterhalen.

Vanuit het oogpunt van de inferentie-aanval beschreven
door~\citeauthor{Dhondt_Pochat_Voulimeneas_Joosen_Volckaert_2022} heeft de
aanvaller toegang tot alle data die publiek beschikbaar is. Deze gebruikt dan
voornamelijk afgelegde weg als basis.

De aanvaller die in deze thesis wordt beschreven, heeft echter geen toegang tot
deze afstandsdata. Hij heeft wel nog toegang tot de ruwe GPS-data, maar ook de
snelheid, het tempo enzovoort. Het onderzoek bestaat er dus uit om te
onderzoeken in hoeverre een aanval nog mogelijk is wanneer de afstandsdata
onbruikbaar zou zijn. Een alternatieve aanpak wordt dus onderzocht om de
inferentie-aanval alsnog succesvol te kunnen uitvoeren.

\subsection{Assumpties}
Om de aanval te kunnen uitvoeren, moeten enkele assumpties worden gemaakt.
\citeauthor{Dhondt_Pochat_Voulimeneas_Joosen_Volckaert_2022} maakte al enkele
assumpties om de inferentie aanval succesvol uit te voeren. Voor dit onderzoek
moeten deze dus ook gelden. De eerste bestaat eruit opdat de zichtbare begin
-en eindpunten op de cirkel moeten liggen. Ten tweede moet de beschermde
locatie op de roadgraph liggen, hij kan niet buiten het voor ons te mappen
gebied liggen, bijvoorbeeld in een bos waar geen pad ligt. Er wordt dieper
ingegaan op de roadgraph in Sectie~\ref{sec:roadgraph}. Als laatste, maar
desalniettemin belangrijk punt moet de gebruiker binnenin de \ac{EPZ} de
kortste route volgen.~\cite{Dhondt_Pochat_Voulimeneas_Joosen_Volckaert_2022}

Dhondt et al.\ maakt nog een laatste assumptie over start- en eindpunten, die
hetzelfde moeten zijn. Dit is echter niet van toepassing op dit onderzoek. Het
onderzoek focust zich op activiteiten waar slechts één deel van het traject
gecloaked is. Dit wil dan ook zeggen dat de gebruiker ofwel vertrekt op de
gevoelige locatie, of er eindigt, maar niet beide. Op
Figuur~\ref{fig:totalDistanceAttack} zijn de 2 mogelijke scenarios van een
total distance attack terug te vinden, namelijk waarbij zowel gestart als
geëindigd wordt binnenin de zone. Dit wordt ook een \textit{total distance
    attack} genoemd, omdat enkel de totale afstand en de afstand buiten de \ac{EPZ}
nodig is. Op deze figuur zijn de rode punten gelabeld \textit{Start} en
\textit{End} de zichtbare start- en eindpunten. Dit scenario stelt dat één van
de reële start- of eindpunten de gevoelige locatie is, aangeduid met de zwarte
markering. Een andere aanval is de \textit{inner distance attack}, hierbij
zullen zowel de start als het einde van een activiteit binnenin het te
verbergen gebied liggen, dit is te zien op Figuur~\ref{fig:innerDist}. De
kennis van de afzonderlijke afstand die de gebruiker aflegt van de start tot de
rand van de \ac{EPZ} en van de rand van de \ac{EPZ} tot de eindlocatie is dan
ook een vereiste. Op de Figuur is opnieuw de zichtbare randpunten aangeduid in
het rood. Echter zal het onzichtbare traject voor beide gevallen doorlopen en
eindigen op de gevoelige locatie, wat in dit geval de reële start- en
eindlocatie is. In Sectie~\ref{sec:berekeningen} wordt dieper ingegaan op de
reden waarom een \textit{inner distance attack} niet mogelijk is. In deze
thesis worden dus alle activiteiten enkel een verhulde start- of eindlocatie
behouden, de rest wordt gefilterd in deze context.
\begin{figure}[h]
    \centering
    \begin{subfigure}[b]{.5\textwidth}
        \centering
        \caption{Start binnenin de \ac{EPZ}}
        \includegraphics[width=1\textwidth]{fig/TotalDistanceAttacks/start.png}
    \end{subfigure}\hfill
    \begin{subfigure}[b]{.5\textwidth}
        \centering
        \caption{Einde binnenin de \ac{EPZ}}
        \includegraphics[width=1\textwidth]{fig/TotalDistanceAttacks/end.png}
    \end{subfigure}
    \caption{Voorbeeld van de mogelijke scenarios bij een total distance attack scenario}\label{fig:totalDistanceAttack}
\end{figure}
\begin{figure}[h]
    \centering
    \includegraphics[width=.5\textwidth]{fig/TotalDistanceAttacks/InnerDistanceAttack.png}
    \caption{Voorbeeld van een inner distance attack situatie}\label{fig:innerDist}
\end{figure}

Deze thesis baseert zich ook voor een stuk op gemiddelde snelheden en tempo's.
Hierdoor stellen we volgende bijkomende assumptie voor: Een gebruiker mag niet
stilstaan binnenin de \ac{EPZ}. Platformen zoals Strava hebben namelijk een
ingebouwde functie die bij het uploaden van een activiteit tijden waarbij een
gebruiker stilstaat aan bijvoorbeeld een rood licht filtert. Zo kunnen ze een
meer representatieve gemiddelde snelheid en tempo berekenen en weergeven. Dit
wil wel zeggen dat de totale bewegingstijd waarop de gemiddelde snelheid en
tempo gebaseerd zijn, niet overeenkomt met de totale tijd van de activiteit.
Bij een berekening gebaseerd op totale verstreken tijd zou een significante
fout kunnen optreden.

\section{Identificeren van de EPZ}
De eerste hiervan is het identificeren van de \ac{EPZ}. Alhoewel deze stap niet
noodzakelijk is, vernauwt deze de zoekruimte drastisch. Hierbij worden van alle
activiteiten die van een gebruiker ter beschikking zijn gesteld, de zichtbare
begin- en eindpunten genomen. Deze zullen dan via k-means clustering worden
gegroepeerd opdat ze de zogenaamde \textit{entry gates} zullen aantonen.

K-means clustering is een unsupervised machine learning techniek die veel wordt
gebruikt bij het clusteren van data. Het is een iteratief proces waarbij het
algoritme $k$ clusters tracht te creëren waarbij de datapunten in elke cluster
zo dicht mogelijk bij het gemiddelde van die cluster
liggen~\cite{Understa24:online}. Dit algoritme kiest willekeurig initiële
middelpunten voor de verschillende clusters. Daarna worden alle punten in de
data toegekend aan de cluster met de laagste Euclidische afstand tot het
centrum van deze cluster. Daarna worden de gemiddeldes van deze clusters
herberekend, en worden deze gezien als nieuwe centrums. Opnieuw worden alle
punten aan de correcte cluster toegekend, en het proces wordt verschillende
iteraties herhaald tot een ietwat stabiele cirkel bekomen wordt. In de
implementatie van~\citeauthor{Dhondt_Pochat_Voulimeneas_Joosen_Volckaert_2022}
waarop deze thesis gebaseerd wordt, is een cirkel stabiel wanneer het verschil
in afstand tussen twee opeenvolgende gevonden cirkels kleiner is dan een
drempelwaarde, in dit geval 10
meter\cite{Dhondt_Pochat_Voulimeneas_Joosen_Volckaert_2022, Verdonck_2022}. Op
Figuur~\ref{fig:kmeans} is te zien hoe de clustering bij elke iteratie beter
wordt. In de context van het identificeren van de \ac{EPZ} zal het gebruikt
worden om \ac{gps}-punten te groeperen op basis van hun locaties. Punten die
dezelfde entry gate representeren, zullen in dezelfde cluster terecht komen.

\begin{figure}[h]
    \centering
    \begin{subfigure}[b]{.33\textwidth}
        \centering
        \includegraphics[width=1\textwidth]{fig/kmeans/1.png}
    \end{subfigure}\hfill
    \begin{subfigure}[b]{.33\textwidth}
        \centering
        \includegraphics[width=1\textwidth]{fig/kmeans/2.png}
    \end{subfigure}
    \begin{subfigure}[b]{.33\textwidth}
        \centering
        \includegraphics[width=1\textwidth]{fig/kmeans/3.png}
    \end{subfigure}
    \caption{Voorbeeld werking k-means clustering~\cite{InDepthk59:online}}\label{fig:kmeans}
\end{figure}

De besproken entry gates zijn zoals de naam al doet vermoeden de
`toegangspoorten' tot de cirkel. Dit is waar de gebruiker de \ac{EPZ} betreedt
en/of verlaat. Deze punten zouden dus in theorie de \ac{EPZ} perfect moeten
definiëren. Maar door de fouten die standaard met het meten van gps-punten
komen\footnote{Gps-metingen bevatten standaard onnauwkeurigheden, er kan
    bouncing of signal loss voor een bepaalde interval optreden. Ook kan slechts op
    bepaalde tijdsintervallen de locatie worden genomen, perfect op de cirkel kan
    dus nooit gemeten worden.} is dit niet perfect. Op Figuur~\ref{fig:entrygate}
is te zien dat meerdere eindpunten van activiteiten geclusterd worden tot één
\ac{E.G.}, op de figuur voorgesteld door een kruis. Een cirkel kan worden
gedefinieerd door drie punten, bijgevolg moeten er dus ten minste drie
\ac{E.G.} gevonden worden.
\begin{figure}
    \centering
    \includegraphics[width=0.5\linewidth]{fig/EPZ-mechanisme/Entry_Gate.png}
    \caption{Voorbeeld van entry gates gevonden door k-means clustering en identificatie van de \ac{EPZ}}\label{fig:entrygate}
\end{figure}

Het algoritme zal na de identificatie van de \ac{EPZ} ook nog nakijken of er
niet meer dan één \ac{EPZ} te vinden is. Er wordt onderzocht of punten die
meegenomen zijn in de beschouwing van de huidige \ac{EPZ}, toch niet horen bij
een mogelijke andere \ac{EPZ} van de user. Als controle wordt van elk eind- of
beginpunt de Euclidische afstand berekent tot de rand van de bijhorende
gevonden \ac{EPZ}. Indien deze kleiner is dan de grootst mogelijke radius, dan
wordt verondersteld dat het punt bij deze zone hoort. Indien dit voor alle
punten geldt, dan stopt het algoritme hier. In het andere geval waarbij de
berekende afstand groter is, worden meer clusters toegevoegd aan het algoritme
van k-means clustering. Dit zal dus een nieuwe privacy zone aanwijzen.

Deze stap is niet noodzakelijk in het globale verhaal van de thesis, maar is
wel een stap die de zoekruimte erg kan verkleinen. Indien het algoritme één of
meerdere \acp{EPZ} vindt, dan zullen er enkel voorspellingen gebeuren in de
regio binnenin. Indien dit niet het geval is en er geen \ac{EPZ} gevonden is,
bestaat de kans dat voorspellingen van locaties gebeuren buiten de verhullende
zone. Ook is in dit geval een groter stuk van het stratenplan nodig om de
locatie te achterhalen.

\section{Bepalen nodige gegevens voor predictie}
Na de bepaling van de \acp{EPZ} voor de gebruiker wordt overgegaan tot het
berekenen en achterhalen van de bijhorende gegevens die nodig zijn om de
gevoelige locatie te voorspellen. Hiervoor wordt verder ingegaan op de
methodiek in de
paper~\citeauthor{Dhondt_Pochat_Voulimeneas_Joosen_Volckaert_2022}, maar er
worden enkele gegevens op een andere manier benadert volgens de huidige
definitie van de aanvaller.

\subsection{Roadgraph}\label{sec:roadgraph}
Voor elke gevonden EPZ is het noodzakelijk om een graafvoorstelling van de
omgeving op te stellen. Op Figuur~\ref{fig:graph_generation} is een voorbeeld
terug te vinden van hoe een graaf kan worden geëxtraheerd. Er worden punten
geplaatst op de straten op een vaste afstand van elkaar, en deze kunnen dan
worden verbonden. Indien geen \acp{EPZ} geïdentificeerd zijn, dan wordt de
omgeving die moet worden omgezet naar een graaf een stuk ruimer genomen. De
graafvoorstelling bestaat uit een serie van nodes, die zich allemaal op een
gekende straat bevinden. De bogen waarmee de nodes verbonden zijn, volgen het
straatplan, opdat een node een mogelijks te volgen weg
is~\cite{neira2022graph}. Aan de hand van de `Chaining Distance' wordt bepaald
hoeveel afstand tussen de nodes zal zitten, en zo dus impliciet ook welke
densiteit het netwerk zal hebben. Hoe lager de densiteit, hoe meer nodes, en
dus ook hoe preciezer. Om voorspellingen te maken is wel een bepaalde precisie
vereist, dus mag deze waarde niet te hoog zijn. Empirisch werd gekozen voor een
waarde van $3.0m$.
\begin{figure}[h]
    \caption{Voorbeeld van het genereren van een roadgraph}\label{fig:graph_generation}
    \centering
    \begin{subfigure}[b]{.4\textwidth}
        \centering
        \includegraphics[width=1\textwidth]{fig/RoadGraph/RoadMap.png}
        \caption{Voorbeeld stratenplan}
    \end{subfigure}\hfill
    \begin{subfigure}[b]{.4\textwidth}
        \centering
        \includegraphics[width=1\textwidth]{fig/RoadGraph/Graph_Over_Map.png}
        \caption{Nodes en bogen geplot op het stratenplan}
    \end{subfigure}
    \begin{subfigure}[b]{.4\textwidth}
        \centering
        \includegraphics[width=1\textwidth]{fig/RoadGraph/Graph.png}
        \caption{Resulterende graafvoorstelling van het stratenplan}
    \end{subfigure}
\end{figure}

\subsection{Begin- en eindnodes}
Voor elke activiteit is het volledige traject buiten de \ac{EPZ} gegeven. Dit
omvat alle \ac{gps}-punten die niet verborgen zijn. De begin- en eindnodes van
het traject zijn hier van belang. Voor de duidelijkheid en de vlotheid van de
tekst zullen we naar deze punten refereren als het zichtbare beginpunt en het
zichtbare eindpunt. Volgens één van de voorafgaand gemaakt assumptie vertrekt
of eindigt de sporter in de \ac{EPZ}. Dit betekent dat ofwel het reële
eindpunt, ofwel het reële beginpunt zal overeenstemmen met de gevoelige
locatie. In geval dat een gebruiker aankomt binnenin de \ac{EPZ}, en dus ook
vertrekt erbuiten, starten de berekeningen vanaf het zichtbare eindpunt. En
omgekeerd, indien hij vertrekt binnenin de \ac{EPZ}, worden de berekeningen
gestart vanaf het zichtbare beginpunt. Deze punten zullen in het vervolg
\textit{randpunten} genoemd worden, refererend naar de rand van de \ac{EPZ}.
Deze \textit{randpunten} zullen de basis vormen voor de volgende berekeningen.

Bijhorend zijn bij de randpunten ook bepaalde extra gegevens beschikbaar. De
belangrijkste zijn de cumulatieve afstand tot dit punt\footnote{De totale
    afstand afgelegd vanaf het begin van de activiteit tot en met het punt in
    kwestie.}, en de cumulatieve tijd tot dit punt\footnote{De totale afstand
    afgelegd vanaf het begin van de activiteit tot en met het punt in kwestie.}.
Bij de aanval van Dhondt et al.\ wordt de afstand gebruikt om predicties t
doen, dit wil dus zeggen dat deze afstand dus aan de basis zal liggen. Maar in
deze thesis wordt ervan uitgegaan dat afstanden verborgen worden. Onder het
verbergen van afstanden wordt een onderscheid gemaakt tussen 2 scenarios: het
eerste gaat ervan uit dat de totale afstand verborgen wordt, maar de
cumulatieve afstand gegeven is. Het tweede scenario gaat ervan uit dat alle
afstandsgegevens verborgen worden. Het alternatieve type data waar dus mee zal
moeten gewerkt worden is dus \ac{gps}-data.

\subsection{Berekeningen distance binnenin de EPZ}\label{sec:berekeningen}
Om voorspellingen te kunnen doen zullen volgens de inferentie aanval die hier
besproken wordt twee belangrijke gegevens ter beschikking moeten zijn. Met name
het straatnetwerk met de mogelijks gevolgde routes, wat werd besproken in
Sectie~\ref{sec:roadgraph}, en de afstand die wordt afgelegd binnenin de
\ac{EPZ}. Deze afstand benoemen we ook als de \textit{inner distance}.

In de implementatie
van~\citeauthor{Dhondt_Pochat_Voulimeneas_Joosen_Volckaert_2022} kan de
\textit{inner distance} simpelweg berekent worden door het verschil te nemen
tussen de afgelegde afstand buiten de verhulde zone (deze noemen we de
\textit{outer distance}), en de totale afstand: \[inner\ distance = total\
    distance - outer\ distance \]

In deze thesis moet dit echter gebeuren met een tussenstap. In het eerste
scenario waarbij de cumulatieve afstand gegeven is, maar de totale afstand
niet, moet de totale afstand berekend worden. Maar door de aanwezigheid van
snelheid- en tijdsgegevens kan dit via basisformules gebeuren. Gebruik makend
van het gemiddelde tempo kan de voorgaande formule worden omgevormd tot: \[inner\ distance = total\ time \times average\ pace - outer\ distance \]

In opzicht van het tweede scenario, waarbij alle afstandsgegevens verborgen
zijn, ontbreekt nu ook de \textit{outer distance}. We bepalen deze dan ook via
de \ac{gps}-coördinaten. Dit gebeurd door de som te nemen van de afstanden van
alle opeenvolgende punten. Let wel dat we de afstand tussen twee
\ac{gps}-punten berekenen gebruik makend van de \textit{haversine} formule.
Equation~\ref{eq:haversine} is een uitwerking van de formule. Deze berekent de
afstand tussen twee punten op een bolvormig oppervlak, in dit geval de aarde.
Om de afstand tussen twee punten op het aardoppervlak te berekenen, moeten de
breedte- en lengtegraden van elk punt worden omgezet naar radialen. Vervolgens
worden deze waarden ingevoerd in de formule, samen met de straal van de aarde
($r$), meestal genomen als $6.371 km$. De formule berekent dan de haversine van
de helft van het verschil tussen de breedtegraden en de haversine van de helft
van het verschil tussen de lengtegraden ($\lambda$), evenals de cosinus van de
breedtegraden ($\phi$) van beide punten. Deze waarden worden vervolgens
gebruikt om de afstand tussen de twee punten te berekenen tussen de punten P \&
Q op Figuur~\ref{fig:haversine}~\cite{sheppard1922practical}.

Merk op dat ook dit een benadering is van de werkelijke afstand. De aarde is
niet perfect sferisch, wat de nauwkeurigheid kan beïnvloeden. Maar voor de
doeleinden van deze thesis is dit voldoende nauwkeurig, zeker omdat de
afstanden in deze context relatief klein zijn, waardoor over het algemeen
slechts een minimale buiging is.
\begin{equation}[h]\label{eq:haversine}
    d = 2r \arcsin\left(\sqrt{\sin^2\left(\frac{\phi_2-\phi_1}{2}\right)+\cos(\phi_1)\cos(\phi_2)\sin^2\left(\frac{\lambda_2-\lambda_1}{2}\right)}\right)
\end{equation}
\begin{figure}[h]
    \centering
    \includegraphics[width=0.5\textwidth]{fig/haversine.png}
    \caption{Haversine illustratie voor het berekenen van de afstand\cite{Distance97:online}}\label{fig:haversine}
\end{figure}

Uit de voorgaande paragrafen kunnen we dus besluiten dat de inner distance af
te leiden valt uit gegeven outer distance, total time en de gemiddelde
snelheid. Om een \textit{outer distance attack} uit te voeren is de berekening
van de totale inner distance voldoende. Maar bij het uitvoeren van een
\textit{inner distance attack} zijn twee aparte inner distances nodig (degene
van start tot de \ac{EPZ} en degene van de \ac{EPZ} tot de finish). Wanneer de
cumulatieve afstand gegeven is, zouden we een deze aanval kunnen uitvoeren
doordat in dit geval de twee afstanden te achterhalen zijn. $d_{start} =
    d_{eerste\ node}$ en $d_{finish} = d_{totaal} - d_{laatste\ node}$. Maar
wanneer deze niet beschikbaar zijn, is dit niet mogelijk, in dit geval zijn
deze afstanden niet individueel te achterhalen.

% Basseren op totale verstreken tijd of niet?

% Welke gps afwijkingne zijn er\

\section{Voorspellen locatie}
Alle nodige gegevens zijn nu beschikbaar om de gevoelige locatie te
achterhalen. Hier wordt besproken hoe voor elke bruikbare activiteit een
locatie zal worden voorspeld. Doordat voor elke activiteit één of meerdere
locaties worden voorspeld, zullen deze moeten worden gebundeld tot één locatie,
voor alle activiteiten.

\subsection{Filteren activiteiten}
Voorafgaand aan het voorspellen van de locatie, is het belangrijk dat enkel
voorspellingen gebeuren met activiteiten die een nuttige voorspelling kunnen
voortbrengen. De andere activiteiten zouden enkel de accuraatheid van de
voorspelling naar beneden halen. Dit gaat dan over activiteiten waarbij niet de
kortste route binnenin de \ac{EPZ} wordt gevolgd. Al deze activiteiten proberen
we dus in de mate van het mogelijke eruit te filteren.

Het geval waarbij een gebruiker niet de kortste route volgt vanaf de rand van
de \ac{EPZ} tot de gevoelige locatie kan in zekere mate worden gefilterd door
te stellen dat een \ldots......

\subsection{Bepalen van de locatie}

\subsection{Meest voorspelde locatie}

\section{Afwijkingen}
%  Beschrijven hoe zo'n aanval werkt - Karels methodiek uitwerken
% Afstanden wijken af
% Fouten in GPS data
% Fouten bij mappen GPS op het Strava routeplan
% Bij het aan en uitzetten van Strava - route verspringt!
% Afwijkende punten - zie grafiek

% Afstanden mogelijks berekenen op 2 manieren 
% 1. Coordinaten 
% 2. Cumulatieve afstanden van de punten

% Werkwijze
%     1. Bereken tijd in de EPZ
%           Wanneer ge stil staat... in de EPZ
%           Eerste - laatste tijd
%           Snelheid
%               Zelf berekenen (Maar komt overeen met degene gegeven door Strava - Strava geeft sommige weer als NULL, dus kan er niet mee werken)
%               Strava: m/h
%               Zelf: m/s
%     2. Berkenen outerdistance op verschillende manieren
%     2. Berekenen inner distance
%           Voor complete dataset => visualize verschillen tss de berekende en de gegeven inner distance
%           opm: Wanneer je stil staat in den EPZ => aanval niet mogelijk (Gemiddelde snelheid obv moving time <-> Mijn methode gebruikt elapsed_time)
%           haversine => Vectorized
%       Methodiek volledig uitschrijven + analyse van deze waarden

% Bepalen bij welke omstandigheden het werkt/niet werkt
%%%%%%%%%%%%%%%%%%%%%%%%%%%%%%%%%%%%%%%%%%%%%%%%%%%%%%%%%%%%%%%%%%% 
%                                                                 %
%                            CHAPTER                              %
%                                                                 %
%%%%%%%%%%%%%%%%%%%%%%%%%%%%%%%%%%%%%%%%%%%%%%%%%%%%%%%%%%%%%%%%%%% 
\chapter{Analyse van de gebruikte data}
We testen de aanval beschreven in Hoofdstuk~\ref{chap:inferentieaanval} op een
dataset met activiteiten van een aantal gebruikers, om zo de vatbaarheid van
gebruikers van fitnessplatformen te evalueren. Maar het is dan ook essentieel
dat een representatieve dataset wordt gebruikt. We onderzoeken eigenschappen en
mogelijke afwijkingen of onregelmatigheden opdat een gefundeerde conclusie kan
worden gevormd, die eventueel bepaalde eigenschappen van de aard van de data
mee in rekening brengt.

Aangezien deze thesis voor een stuk verder bouwt op het onderzoek
van~\citeauthor{Dhondt}, is het handig om verder te werken op deze
dataset~\cite{Dhondt}. Dit maakt een meer directe vergelijking mogelijk.
Deze dataset werd volledig zelf gescraped door~\citeauthor{Dhondt} vanaf de
officiële \ac{API} van het platform
`Strava'\footnote{\url{https://www.strava.com/}}. De scope van deze dataset is
een periode van één week, startend op 11 juli 2021 00:00 \ac{UTC}. De site werd
chronologisch afgelopen voor alle activiteiten beschikbaar op het platform, met
sprongen van 9000 activiteiten. Let wel dat door mogelijke vertragingen door
bijvoorbeeld het uploaden van een activiteit, de activiteiten niet exact
chronologisch kunnen worden opgehaald. Indien een activiteit privaat is of
gecloaked\footnote{Een gecloakede activiteit is een activiteit waar al een EPZ
    op is aangebracht. Aangezien deze thesis uncloaked activities nodig
    heeft~\ref{sec:zelf_cloaking} zijn ze dus niet bruikbaar. Men kan zien of een
    activiteit gecloaked is indien er een verschil is tussen de zichtbare afstand
    en de totale afstand.} is, of reeds verwijderd is, dan zal deze worden
overgeslagen en de volgende worden genomen. Voor elke gevonden activiteit
beschouwen we daarna de gebruiker. Van alle bekomen gebruikers worden dan de
rest van de activiteiten afgehaald en bijgehouden in één grote dataset. De
gegevens werden ook geanonimiseerd opgeslagen, zodat de gebruikers niet meer
kunnen worden geïdentificeerd. De dataset bevat dus geen namen of andere
persoonlijke gegevens, enkel willekeurig toegekende ID's.

In totaal werd een dataset van 4000 gebruikers verzameld. Deze thesis
experimenteert echter slechts met een subset van 131 users, waarvan er 101
gebruikt worden voor analyses en conclusies, en 30 voor het testen van de
aanval.

\section{Karakteristieken van de gebruikte dataset}
Op Figuur~\ref{fig:geographic_spread} is de geografische spreiding van de
activiteiten gevisualiseerd aan de hand van een heatmap. Hierop is duidelijk te
zien dat de meeste activiteiten zich in Centraal-Europa bevinden. Daarnaast is
ook een duidelijke concentratie te zien in de Verenigde Staten. In mindere mate
zijn ook activiteiten in Australië en Zuid-Amerika. De dataset bevat dus een
relatief grote spreiding van activiteiten over de hele wereld, wat een goede
basis is voor het testen van de aanval. Let wel, de fractie van de dataset die
wij ter beschikking kregen is met 101 gebruikers wel relatief klein, wat een
vertekend beeld kan geven over de werkelijkheid.

Op Tabel~\ref{tab:stats_dataset} zijn enkele globale statistieken met
betrekking tot gebruikers en de bijhorende activiteiten van de dataset
weergegeven. Er valt op dat de dataset per gebruiker toch meestal een groot
aantal activiteiten ter beschikking zijn. De gemiddelde gebruiker bevat 411
activiteiten, de mediaan is 296. Volgens de inferentieaanval beschreven in
Hoofdstuk~\ref{chap:inferentieaanval} resulteert een gebruiker met meer
activiteiten over het algemeen in een accuratere aanval. Op
Figuur~\ref{fig:cdf_amount_activities} is de \ac{CDF} plot\footnote{Een
    Cumulative Distribution Function (CDF) plot is een grafiek die de cumulatieve
    verdeling van de waarden van een continue variabele
    weergeeft~\cite{CursusSt38:online}. De x-as van de grafiek bevat de
    verschillende waarden die de continue variabele kan aannemen, terwijl de y-as
    de kans aangeeft dat de variabele een waarde kleiner dan of gelijk aan die op
    de x-as aanneemt. De curve van de CDF laat zien hoe waarschijnlijk het is dat
    een willekeurige waarde van de continue variabele kleiner is dan een bepaalde
    drempelwaarde.} te zien die het aantal activiteiten per gebruiker weergeeft.
Hierop worden voorgaande besluiten enkel maar bevestigd. Het plot duidt ook aan
dat meer dan 20\% van de gebruikers een aantal activiteiten heeft dat groter is
dan 100.

\begin{figure}[h]
    \centering
    \includegraphics[width=\textwidth]{fig/Afwijkingen&Analyses/Heatmap.png}
    \caption{Geografische spreiding van de activiteiten in de dataset}\label{fig:geographic_spread}
\end{figure}
\begin{table}[h]
    \centering
    \begin{tabular}{|l||c|}
        \hline
                                                         & \textbf{Aantal} \\
        \hline \hline
        Totaal \# gebruikers                             & 101             \\
        \hline
        Totaal aantal activiteiten                       & 41 554          \\
        \hline
        Gemiddeld \# activiteiten per gebruiker          & 411             \\
        \hline
        Mediaan van het \# activiteiten per gebruiker    & 296             \\
        \hline
        Maximaal \# activities voor een enkele gebruiker & 2946            \\
        \hline
        Minimaal \# activities voor een enkele gebruiker & 31              \\
        \hline
    \end{tabular}
    \captionsetup{justification=centering}
    \caption{Overzicht van gebruikers en activiteiten}\label{tab:stats_dataset}
\end{table}
\begin{figure}[h]
    \centering
    \includegraphics[width=0.9\textwidth]{fig/Afwijkingen&Analyses/CDF_amountActivities.jpg}
    \caption{\ac{CDF} plot van het aantal activiteiten per gebruiker}\label{fig:cdf_amount_activities}
\end{figure}

% Let wel, alhoewel niet expliciet vermeld door~\citeauthor{Dhondt}, is er een
% vermoeden dat er bewust gezocht werd naar gebruikers met een zo groot mogelijk
% aantal activiteiten per gebruiker. Dit is een logische keuze, aangezien de
% aanval een hogere kans op slagen heeft bij users die meer activiteiten hebben.
Let wel, de dataset is net een gemiddelde van 411 activiteiten per gebruiker
niet volledig representatief voor de werkelijkheid. Wanneer we dit vergelijken
met cijfers uit een studie die Strava zelf voerde in 2020, is er toch een
mismatch terug te vinden~\cite{StravaMi72:online}. Het persbericht, waarvan
Figuur~\ref{fig:3billionUsers} is overgenomen, stelt dat Strava in 2020 iets
meer dan 50 miljoen gebruikers had, die samen in totaal drie miljard
activiteiten op het platform hebben geplaatst. Indien we deze waarden omrekenen
naar een gemiddelde, komen we uit op een ruwe geschatte 60 activiteiten per
gebruiker ($\frac{3 \cdot 10^9}{5 \cdot 10^7} = 60 $). Dit is een stuk lager
dan de gemiddelde 411 activiteiten per gebruiker in de dataset. De conclusies
die dus getrokken worden uit deze steekproef mogen niet zomaar veralgemeend
worden naar de volledige gebruikersbasis van Strava.
\begin{figure}[h]
    \centering
    \includegraphics[width=0.7\textwidth]{fig/Strava_3billion.png}
    \caption{Post op sociale media van Strava die de evolutie van het totaal aantal activiteiten weergeeft~\cite{StravaMi72:online}}\label{fig:3billionUsers}
\end{figure}

\section{Mogelijke afwijkingen binnenin de dataset}
Doordat de dataset niet expliciet werd gecheckt op onnauwkeurigheden en een
willekeurige sample is, is er een grote kans op activiteiten die afwijkingen of
fouten vertonen. Zeker door het belang van \ac{gps}-data in deze studie, die
een grote kans heeft op fouten, is het belangrijk om de dataset te analyseren
op deze mogelijke afwijkingen. Gps-data is een signaal die a.d.h.v.\ gekende
locaties van satellieten, gecombineerd met de tijd die het signaal nodig heeft
om deze satellieten te bereiken, de locatie van een gebruiker kan
bepalen~\cite{BadGPSDa19:online}. Door de snelheid van het signaal, kunnen
kleine vertragingen in het signaal al een grote invloed hebben op de
accuraatheid van de data. Andere factoren zoals hoge bomen of gebouwen, maar
ook de aanwezigheid van wolken kunnen een impact hebben op het signaal. Ook de
frequentie waarmee locatie wordt bepaald, wat afhankelijk is van het gebruikte
toestel, kan meespellen. De soorten \ac{gps}-fouten die kunnen optreden zijn
reeds besproken in Sectie~\ref{sec:gps-fouten}.

Allereerst bestuderen we de aanwezigheid van \ac{gps}-fouten in de vorm van
signal losses of pauzes. Dit gebeurt door de afstand tussen twee opeenvolgende
\ac{gps}-punten te bestuderen.
Tabel~\ref{tab:distance_between_gps_points_table} geeft een globaal overzicht
van deze verdeling, en de volledige verdeling is terug te vinden op
Figuur~\ref{fig:distance_between_gps_points_CDF}. De gemiddelde afstand tussen
twee opeenvolgende locaties is $6.41$ meter, met een standaardafwijking van
$42.53$ meter. Het gemiddelde is relatief laag, wat kan wijzen op accurate
gegevens, maar de hoge standaardafwijking wijst op grote schommelingen. Op de
grafiek en in de tabel is te zien dat de meeste afstanden onder de $20$ meter
liggen, wat opnieuw een indicatie kan zijn van een degelijke precisie. Er is
echter wel een klein deel van de \ac{gps}-punten die een grote onderlinge
afstanden vertoond. Door de omvang van het aantal \ac{gps}-punten, en een
gemiddeld aantal punten per activiteit van $2574.90$, valt dit zeker niet te
verwaarlozen. Als we empirisch stellen dat een significant verschil $50$ meter
bedraagt, dan ligt 0.9\% van de data boven deze drempel. Per activiteit zou dit
dan resulteren op gemiddeld $23$ afwijkende punten, wat zeker kan zorgen voor
een significante afwijking op de resulterende afstand.
\begin{table}[h]
    \centering
    \begin{tabular}{lr}
        \toprule
        \midrule
        Total number of gps-points                    & $1.070 \cdot 10^8$       \\
        \hline
        distance between 2 gps points 20m $>$ 10m     & $13.91\%$                \\
        distance between 2 gps points 75m $>$ 50m     & $8.80 \cdot 10^{-1}\%$   \\
        distance between 2 gps points 100m $>$ 75m    & $9.11 \cdot 10^{-3}\%$   \\
        distance between 2 gps points 150m $>$ 100m   & $4.99 \cdot 10^{-3}\%$   \\
        distance between 2 gps points 200m $>$ 150m   & $1.97 \cdot 10^{-3}\%$   \\
        distance between 2 gps points 500m $>$ 200m   & $3.47 \cdot  10^{-3} \%$ \\
        distance between 2 gps points 1000m $>$ 500m  & $1.26 \cdot 10^{-3} \%$  \\
        distance between 2 gps points 2000m $>$ 1000m & $9.39 \cdot 10^{-4}$ \%  \\
        distance between 2 gps points $>$ 2000m       & $4.056 \cdot 10^{-4} \%$ \\
        \midrule
        \bottomrule
    \end{tabular}
    \captionsetup{justification=centering}
    \caption{Verdeling van de afstanden tussen twee opeenvolgende gps-punten}\label{tab:distance_between_gps_points_table}
\end{table}
\begin{figure}[h]
    \centering
    \includegraphics[width=\textwidth]{fig/Afwijkingen&Analyses/Graphs/Afstand tussen 2 gps-punten.png}
    \caption{Verdeling van de afstanden tussen twee opeenvolgende gps-punten}\label{fig:distance_between_gps_points_CDF}
\end{figure}

Om het aantal \ac{gps}-afwijkingen in de dataset te bepalen, wordt ook het
verschil onderzocht tussen de berekende afstand afgelegd binnen de \ac{EPZ}
(het zichtbare traject afgetrokken door de totale afstand) en de theoretisch
afgelegde afstand binnen de \ac{EPZ}, die af te lezen valt uit de dataset via
de cumulatieve afstand\footnote{Er wordt gesproken van een theoretische waarde,
    maar deze is eigenlijk de berekende waarde volgens het platform. We beschouwen
    deze dus als referentie.}. Een eerste visualisatie is te zien op
Figuur~\ref{fig:difference_noCDF}. De figuur illustreert de schommelingen
tussen de handmatig berekende afstand en de theoretische afstand van één
gebruiker. De pieken duiden op duidelijk sterk afwijkende berekende afstanden,
en dus ook op grote \ac{gps}-fouten. Maar ook de schommelingen die iets minder
opvallend zijn duiden op grote inaccuraatheden tussen de berekende en
theoretische afstanden. De verschillen in de berekeningen voor de volledige
dataset worden weergegeven op Figuur 4.10. De resultaten worden weergegeven op
Figuur~\ref{fig:differences_theoretical}. Figuur~\ref{fig:differences_log}
bevat de verdeling voor alle activiteiten, gebruik makend van een logaritmische
schaal. Figuur~\ref{fig:differences_nolog} toont 95\% van de activiteiten met
de kleinste verschillen, om een beter beeld te krijgen van de grootte van de
meeste verschillen. Op de grafieken valt op dat heel wat significante
verschillen aanwezig zijn. Dit duidt op het relatief zwaar doorwegen van de
\ac{gps}-fouten in de dataset. Aangezien het gaat over bepalen van woonplaatsen
of andere gevoelige locaties, kunnen afwijkingen vanaf 50 à 100 meter al
relevant zijn. Daarnaast gaat het vaak over kleine afwijkingen op een heel wat
punten, wat kan resulteren in een grote afwijking. De grafieken tonen aan dat
er bij de ruwe ontvangen data heel wat \ac{gps}-fouten aanwezig zijn. Smoothing
zal dus zeker nodig zijn om deze te beperken.

\begin{figure}[h]
    \centering
    \includegraphics[width=0.7\textwidth]{fig/Afwijkingen&Analyses/Graphs/Verschil_Theoretische_innerDistance.png}
    \caption{Verschil tussen de berekende afstand en de theoretische afstand voor één gebruiker}\label{fig:difference_noCDF}
\end{figure}
\begin{figure}[h]
    \centering
    \begin{subfigure}{\textwidth}
        \includegraphics[width=\textwidth]{fig/Afwijkingen&Analyses/Graphs/100_Differences_tov_theoretische_BefSmoothening.png}
        \caption{100\% van de activiteiten (logaritmische schaal)}\label{fig:differences_log}
    \end{subfigure}
    \begin{subfigure}[b]{\textwidth}
        \includegraphics[width=\textwidth]{fig/Afwijkingen&Analyses/Graphs/95_Differences_tov_theoretische_BefSmoothening.png}
        \caption{95\% van de activiteiten}\label{fig:differences_nolog}
    \end{subfigure}
    \caption{Verdeling van het verschil tussen de berekende afstand en de theoretische afstand buiten de \ac{EPZ} }\label{fig:differences_theoretical}
\end{figure}

\section{Technieken om gps-data te verbeteren}
Om de accuraatheid van de \ac{gps}-data te verbeteren, en zo een betere
\textit{outer distance} te kunnen berekenen en uiteindelijk een accuratere
aanval te bekomen, worden enkele technieken toegepast. Zoals besproken in
Sectie~\ref{sec:afstandsberekeningen_strava} is de hypothese dat de
fitness-platformen gebruik maken van technieken om de \ac{gps}-data te
verbeteren. De besproken technieken waren \textit{map matching} en
\textit{\ac{gps}-smoothing}. Bij de uitvoering van de aanvallen wordt smoothing
toegepast. Er wordt dan ook geëxperimenteerd met verschillende smoothing
windows, op zoek naar het window met het beste effect op de aanval.

Daarnaast passen we ook een filtering toe die rekening houdt met
\ac{gps}-sprongen. Het idee is te stellen dat wanneer \ac{gps}-sprongen
gebeuren, en er dus een te grote afstand tussen twee opeenvolgende punten is,
deze afstand niet te laten meetellen. Dit is echter niet vanzelfsprekend,
aangezien dit wel worden meegenomen door de berekeningen van de platformen.
Stel dat wij een sprong van 50 meter laten vallen, maar de platformen laten
deze wel meetellen, dan zullen we de afwijking op het eindresultaat enkel maar
verhogen. De drempel voor de filtering moet dus hoog genoeg zijn om deze
voorvallen te vermijden. In dit onderzoek kozen we voor een drempelwaarde van
200 meter.
%%%%%%%%%%%%%%%%%%%%%%%%%%%%%%%%%%%%%%%%%%%%%%%%%%%%%%%%%%%%%%%%%%% 
%                                                                 %
%                            CHAPTER                              %
%                                                                 %
%%%%%%%%%%%%%%%%%%%%%%%%%%%%%%%%%%%%%%%%%%%%%%%%%%%%%%%%%%%%%%%%%%%
\chapter{Resultaten en Evaluatie}
De aanval werd tot hiertoe al volledig beschreven en toegelicht, en de dataset
die gebruikt wordt om de aanval werd ook al besproken. Dit hoofdstuk bespreekt
hoe de evaluatie wordt aangepakt, wat de bekomen resultaten zijn en wat ze
betekenen.

\section{Evaluatie van de aanval}
Het doel van de aanval is om een locatie te voorspellen waar een gebruiker bij
zijn activiteiten vertrekt of aankomt, ondanks het feit dat deze locatie wordt
verborgen door het gebruik van een \ac{EPZ}. Maar indien we dit zouden
uittesten op activiteiten die al een \ac{EPZ} bevatten zouden we de bekomen
resultaten niet kunnen verifiëren. Daarom zullen we de aanval uittesten en
evalueren op publieke activiteiten die geen \ac{EPZ} bevatten, en deze manueel
voorzien van een \ac{EPZ}. Zo kunnen we de bekomen resultaten vergelijken met
een referentie, namelijk de \textit{grondwaarheid} of de \ac{GT}.

\subsection{De grondwaarheid}\label{sec:groundtruth}
De grondwaarheid van een gebruiker is de effectieve woonplaats, of de plaats
waar deze persoon meestal vanuit vertrekt of aankomt. Dit is de locatie die we
beschouwen als degene waarrond de \ac{EPZ} wordt aangebracht, en dat tracht
verborgen te worden. De grondwaarheid is dus de locatie die we trachten te
achterhalen. We bepalen deze locatie door alle activiteiten van een gebruiker
te overlopen, en indien 15 of meer begin- of eindpunten binnen een straal van
50 meter liggen, dan berekenen we hiervan het gemiddelde en wordt dit
gemiddelde gemapt op de roadgraph. Dit punt gematcht op de roadgraph stelt dan
een grondwaarheid voor van deze gebruiker. Hassan et al.\ stelden dat 50 meter
vergelijkbaar is met een breedte van een gemiddeld perceel, en dat dit dus een
goede benadering is van de grondwaarheid~\cite{sec18has3:online,
    Verdonck_2022}. Let wel, het kan dat éen gebruiker meerdere grondwaarheden
bevat.

Een tweede kanttekening die we hierbij moeten maken is dat de effectieve begin-
en eindlocaties niet altijd perfect op het wegennetwerk zullen liggen. Een
gebruiker kan bijvoorbeeld starten op een parking of een oprijlaan, wat niet
vervat zit in het netwerk. Wij mappen deze dan achteraf op het straatnetwerk,
maar gedurende de upload berekent het platform in kwestie wel de totale
afgelegde afstand tot het effectieve startpunt. Dit kan dus voor een afwijking
bij de predicties zorgen.~\citeauthor{Dhondt} onderzochten deze afwijking door
deze afwijking in een \ac{CDF} te plotten, en op zoek te gaan naar het
elleboogpunt~\cite{Dhondt}. Het elbow point is een visueel punt in de curve
waar zich een knik voordoet~\cite{Introduc22:online}. Dit duidt in theorie de
optimale afweging tussen lage afwijking en hoge precisie aan. Zo
bekomen~\citeauthor{Dhondt} een drempelwaarde van 22.95 meter om van een
succesvolle aanval te kunnen spreken. Deze drempelwaarde is toepasselijk voor
92\% van de gebruikers.
\begin{figure}[h]
    \centering
    \includegraphics[width=0.8\textwidth]{fig/Afwijkingen&Analyses/OvershootMappingDistance.png}
    \caption{Dhondt et al.\ bepaalt grafisch de trend van de afwijkingen bij het snappen van locaties op het wegennetwerk~\cite{Dhondt}}\label{fig:overshootMappingDistance}
\end{figure}

\subsection{Manueel aanbrengen van een EPZ}\label{sec:zelf_cloaking}
We werken zoals al eerde vermeld met publieke activiteiten die geen \ac{EPZ},
om zo de het evaluatieproces te versimpelen. Maar om de aanval te kunnen
uitvoeren moet we dus nog manueel een \ac{EPZ} aanbrengen. Zo kunnen we een
situatie creëren die de werkelijke situatie benadert, en kunnen we ons
aanvalsmodel uitvoeren. Het is dus wel belangrijk dat de aangebrachte \ac{EPZ}
op een realistische manier wordt aangebracht, op een manier die de
werkelijkheid weerspiegelt.

Sectie~\ref{sec:EPZ} bespreekt al uitvoerig het mechanisme van een \ac{EPZ}, en
hoe deze wordt bepaald. Om het even kort te recapituleren, een \ac{EPZ} wordt
bepaald door een centraal punt (de gevoelige locatie), wat een willekeurige
translatie zal ondervinden\footnote{In de context van de beschrijving van
    Hassan et al.\ gebeurd er geen translatie, maar deze thesis gaat wel degelijk
    uit van een model waarbij spatial cloaking op toegepast
    is~\cite{sec18has3:online}.}, en een gekozen straal. Vanaf het getransleerde
punt wordt een cirkel opgezet met de desbetreffende straal. Strava als
fitnessplatform heeft de grootste keuze uit mogelijke stralen, namelijk van 200
meter tot 1600 meter in sprongen van 200 meter. Het doel is om de effectiviteit
van de aanval vast te leggen voor verschillende radiussen, dus zullen we per
gebruiker en per aanval een \ac{EPZ} opzetten met alle verschillende radiussen.
Zo kunnen we het effect van de radius zien, maar ook de types aanval onderling
onafhankelijk van de straal vergelijken. We starten dus vanaf de \ac{GT}. Deze
locatie ondervindt dan een willekeurige translatie. De verschuiving van het
punt kan gebeuren in alle richting, en kiezen we dus willekeurig. De afstand
van de translatie kan in principe ook willekeurig worden gekozen, maar moet wel
binnen bepaalde grenzen liggen, namelijk tussen 0 en 70\% van de straal van de
\ac{EPZ}. De cirkel kan dan worden opgesteld, met als middelpunt het
getransleerde punt, en de bijhorende straal. Alle punten die zich binnen deze
zone bevinden, zullen worden verwijdert uit de activiteit.
% Figuur: 3 stappen hoe een EPZ wordt opgezet

\subsection{Bootstrapping}
Bij het uittesten van de aanval wordt de aanval niet zomaar gedraaid op alle
activiteiten per gebruiker. Dit zou vertekende resultaten kunnen geven. Per
gebruiker wordt een betrouwbaarheidsinterval berekent via
bootstrapping~\cite{Dhondt, Verdonck_2022}. Ook wordt de set met manueel
ververhulde activiteiten beschouwd. Het bootstrapalgoritme kiest hieruit
willekeurig één voor één activiteiten, en plaatst deze in een nieuwe groep
totdat deze nieuwe groep activiteiten even groot is als de originele set van
activiteiten. Let wel, het algoritme kan meerdere malen dezelfde activiteit
kiezen, dus met andere woorden kan de nieuw gemaakte groep duplicaten bevatten
en bepaalde activiteiten helemaal niet bevatten. Dit gebeurt 1000 keer, en er
worden dus 1000 verschillende sets gemaakt. Voor elke set zal dan een zal een
voorspelling worden uitgevoerd. Zo bekomen we een aantal voorspelde locaties,
waarvan er een kans is dat enkele locaties meerdere malen voorspeld worden. Op
Figuur~\ref{fig:bootstrapping} is te zien hoe de distributie eruit ziet op een
kaart~\cite{Verdonck_2022}. Op de figuur zijn de individuele voorspelling
aangeduidt met een blauwe tot paarse kleur, afhankelijk van hoe frequent ze
voorspeld werden (hoe meer naar de paarse of rode kleur ligt, hoe frequenter de
node voorspeld is). Ook zichtbaar zijn de start- en eindpunten, met een
verschillend kleur per \ac{E.G.}. De locatie van de grondwaarheid is aangeduid
door de groene marker.

\begin{figure}[h]
    \centering
    \includegraphics[width=0.6\textwidth]{fig/bootstrapping.png}
    \caption{Voorbeeld van een distributie van voorspellingen bepaald door het bootstrapalgoritme~\cite{Verdonck_2022}}\label{fig:bootstrapping}
\end{figure}

\subsection{Evaluatie metrieken}
Om iets zinnigs te kunnen vertellen over de effectiviteit van de aanval,
definiëren we enkele metrieken die we kunnen gebruiken om de aanval te
evalueren. We gebruiken hiervoor de metrieken die ook gebruikt werden in de
studies van~\citeauthor{Dhondt} en~\citeauthor{Verdonck_2022}, om zo onze
resultaten er mee te kunnen vergelijken. In totaal gebruiken we acht
verschillende evaluatiemetrieken.

De eerste evaluatie metriek is de \textit{Success Rate}~\cite{Dhondt}. Dit is
het percentage van de uitgevoerde aanvallen waar de gevoelige locatie succesvol
is achterhaald. Rekening houdend met de overshoots die komen met het snappen
van locaties op het wegennetwerk, is een correcte locatie een locatie die zich
binnen een straal van 22.95 meter van de \ac{GT} bevindt. De succes rate is dus
het percentage van de aanvallen waar de correcte locatie zich binnen deze
straal bevindt. Hoe hoger het percentage, hoe succesvoller de aanval.

De \textit{Correctness} van een aanval is de som van de Euclidische afstanden
tussen de \ac{GT} en de voorspelde locatie gedeeld door het aantal keer deze
locatie werd voorspeld~\cite{Dhondt, Verdonck_2022}. Dit geeft een indicatie
van de gemiddelde afwijking in afstand van de voorspelde locaties ten opzichte
van de \ac{GT}. Hoe lager deze waarde, hoe preciezer de aanval. Let wel, een
succes rate kan hoog zijn, maar de correctheid kan nog steeds hoog zijn. Dit
duidt op een aanval die veel overshoots heeft, maar waar de correcte locatie
zich wel binnen de straal van 22.95 meter bevindt. De probabiliteitsdistributie
wordt gegeven door $\widehat{\operatorname{Pr}}(v \mid a)$, waarbij $v$ de
beschermde locaties voor activiteit $a$ zijn in
Vergelijking~\ref{eq:correctness}.
\begin{equation}
    \sum_{v \in V} \widehat{\operatorname{Pr}}(v \mid a) \operatorname{dist}\left(v, v_{G T}\right)\label{eq:correctness}
\end{equation}

De \textit{Accuracy} definiëren we als de breedte van het
betrouwbaarheidsinterval~\cite{Dhondt, Verdonck_2022}. Met de breedte van het
betrouwbaarheidsinterval doelen we op het aantal unieke voorspellingen, het
aantal nodes dat precies eenmalig worden voorspeld. Hoe meer unieke nodes, hoe
hoger de accuracy en ook hoe minder `zeker' onze voorspelling is.

De \textit{Reduction of the k-anonymity set} kwantificeert de de afname in de
set van alle mogelijke eindlocaties voor en na de effectieve predicties bij een
aanval~\cite{Dhondt, Verdonck_2022}. De mogelijke eindlocaties voor de aanval
zijn simpelweg alle nodes in de graafvoorstelling, eventueel begrensd door de
\ac{EPZ}. Degene na de aanval zijn al degene die voorspelt worden. De reduction
is dus een percentage die het verschil tussen de twee sets aangeeft. Hoe hoger
de reduction, hoe meer nodes verdwijnen uit de set van mogelijke eindlocaties
na de aanval. Dit zegt dus iets over de hoeveelheid kandidaten het algoritme
voorspelt, ten opzichte van hoeveel kandidaten er mogelijk zijn.
\begin{equation}
    \frac{k-\left|V_{\text {pred }, \text { ext }}\right|}{k}\label{eq:reduction}
\end{equation}

De \textit{Uncertainty Region ($m^2$)} is de som van oppervlaktes van de
unie\footnote{De unie van de oppervlakte van twee cirkels is de som van de twee
    oppervlakten, min één maal het overlappende deel.} van de onzekerheidsregio's
rond de voorspelde nodes~\cite{Dhondt,Verdonck_2022}. De chaining distance, die
in ons geval drie meter aanneemt, veroorzaakt deze onzekerheidsregio's.
Aangezien pas om de drie meter een node bestaat, zal voor elk punt die op deze
node gemapt is een mogelijkheid bestaan dat deze eigenlijk ergens anders in de
een zone van drie meter rond de node ligt. Hoe groter deze waarde, hoe groter
de onzekerheid van de aanval, aangezien dit wijst op weinig overlap.
\begin{equation}
    \operatorname{Area}\left(\bigcup_{v_p \in V_{\text {pred }}} C_{v_p}, d_{\text {chain }}\right)\label{eq:uncertainty}
\end{equation}

De laatste evaluatiefactor is de \textit{Degree of
    Anonymity}~\cite{Dhondt,Verdonck_2022}. Dit is de genormaliseerde entropie van
de verwachte distributie. Deze wordt genormaliseerd op basis van de maximale
mogelijke entropie, en wordt bepaald op basis van percentage $p_v$. Dit is een
percentage die aangeeft bij hoeveel percent van de voorspellingen node $p$
voorspelt wordt. Dit zal voor vele nodes gelijk zijn aan 0. De maximale
entropie komt voor indien elke node exactly evenveel voorspelt wordt, en is dus
gelijk aan $\frac{1}{\# nodes}$.
\begin{equation}
    \frac{-\sum_{v \in V} \widehat{\operatorname{Pr}}(v \mid a) \log _2(\widehat{\operatorname{Pr}}(v \mid a))}{H_0(V)}\label{eq:degree_of_anonymity}
\end{equation}

\textit{Certainty} \ldots
\begin{equation}
    -\sum_{v \in V} \widehat{\operatorname{Pr}}(v \mid a) \log (\widehat{\operatorname{Pr}}(v \mid a))\label{eq:certainty}
\end{equation}

\textit{Spatial Certainty} \ldots
\begin{equation}
    -\sum_{v \in V} \widehat{\operatorname{Pr}}(v \mid a) \log \left(\widehat{\operatorname{Pr}_n}(v \mid a)\right)\label{eq:spatial_certainty}
\end{equation}

\section{Resultaten}
Nu we alle evaluatiemetrieken besproken hebben, kunnen we overgaan naar de
evaluatie van de geteste scenarios. Elk getest scenario zal afzonderlijk worden
besproken, en in het laatste deel zal een vergelijking gemaakt worden tussen de
verschillende scenario's. De resultaten van de aanvallen worden weergegeven in
tabellen, die telkens voor elke metriek een score weergeven. Ook draaien we
voor (zo goed als alle) scenario's een aanval voor enkele radiussen.

Daarnaast worden de alle resultaten ook grafisch weergegeven op
Figuur~\ref{fig:attack_comparison}. Dit geeft een mooi globaal overzicht van
alle modellen ten opzichte van elkaar, en maakt de onderlinge verschillen
zichtbaar. Bij de bespreking van de resultaten zullen we dan ook zowel
refereren naar de tabellen als naar de grafieken.

\begin{figure}[h]
    \centering
    \includegraphics[width=\textwidth]{fig/result_graphs/all_results.png}
    \caption{Vergelijking van de verschillende aanvallen}\label{fig:attack_comparison}
\end{figure}

Over het algemeen is een gelijklopende trend merkbaar bij het veranderen van de
\acp{EPZ}. Bij een toenemende radius zakt de succesratio doordat een grotere
radius meer nodes met zich meebrengt. En meer nodes zorgt voor meer mogelijke
verwarring in de \ac{LAD} regressie~\cite{Verdonck_2022}. Dit brengt ook een
grotere degree of anonymity en uncertainty region met zich mee. Het aantal
voorspellingen neemt niet evenredig toe met het aantal nodes in de graaf
naarmate de omvang toeneemt, wat resulteert in een verhoogde reduction. Als
laatste valt ook op dat de correctness ook stijgt bij een grotere radius. Dit
komt door een grotere kans op schending van één van de gestelde assumpties uit
Sectie~\ref{sec:assumpties}. Deze sectie stelt onder andere dat een gebruiker
het kortste pad moet volgen binnenin de \ac{EPZ}. Echter hoe groter de \ac{EPZ}
van omvang is, hoe groter de kans dat hieraan niet voldaan wordt, en dus ook
resulteren in een hogere correctness.

\subsection{Model volgens Dhondt et al.}
Het eerste model dat we testen is het model van
\citeauthor{Dhondt}~\cite{Dhondt}, Tabel~\ref{tab:aanval_karel} toont de
betreffende resultaten. We gebruiken deze resultaten om de rest van de ermee
resultaten te vergelijken. Het model van~\citeauthor{Dhondt} heeft geen
restricties betreffende de beschikbare data, het kan dus alle data gebruiken.
Het is dan ook logisch dat dit resulteert in goeie scores. Het doel van de
andere aanvallen was dan ook deze waardes te benaderen, maar met alternatieve
data. Het aanvalsmodel is geïmplementeerd en uitgevoerd op ons eigen systeem op
de beschikbare fractie van de dataset, wat de lichte verschillen ten opzichte
van de resultaten beschreven in de desbetreffende paper verklaart. We zien hier
dan ook een hoge succes rate, die relatief weinig afneemt bij hogere radiussen.
De rest van de statistieken wijzen ook op een goeie aanval, wat naar de
verwachtingen is.

\begin{table}[h]
    \centering
    \scalebox{0.55}{
        \begin{tabular}{lrrrrrrrr}
            \toprule
            {}         & Success Rate (\%) & Correctness (m) & Accuracy & Reduction (\%) & Uncertainty Region ($m^2$) & Certainty & Spatial Certainty & Degree of Anonymity (\%) \\
            Radius (m) &                   &                 &          &                &                            &           &                   &                          \\
            \midrule
            200        & 89.86             & 28.61           & 17       & 86.06          & 352.08                     & 2.06      & 0.58              & 30.69                    \\
            400        & 79.1              & 56.82           & 21       & 93.21          & 469.76                     & 2.26      & 1.01              & 28.31                    \\
            600        & 70.37             & 79.94           & 24       & 96.52          & 502.42                     & 2.32      & 1.12              & 27.15                    \\
            800        & 73.33             & 113.68          & 27       & 97.05          & 670.53                     & 2.47      & 1.31              & 26.86                    \\
            1000       & 68.64             & 166.74          & 33       & 97.84          & 777.57                     & 2.62      & 1.54              & 27.25                    \\
            1200       & 58.33             & 180.97          & 27       & 98.15          & 684.71                     & 2.55      & 1.57              & 26.25                    \\
            1400       & 57.14             & 235.76          & 33       & 98.61          & 782.30                     & 2.54      & 1.82              & 25.31                    \\
            \bottomrule
        \end{tabular}
    }
    \captionsetup{justification=centering}
    \caption{Aanval volgens het model van Dhondt et al.~\cite{Dhondt}}\label{tab:aanval_karel}
\end{table}

\subsection{Gegeven outer distance}
Het eerste scenario dat we testen is het model waarbij de outer distance
rechtstreeks af te lezen valt uit de data. We bespraken dit geval al
kortstondig in Sectie~\ref{sec:berekeningen}. Dit model komt voor wanneer de
cumulatieve afstanden gegeven zijn. Het voordeel die dit model heeft is dat er
geen \ac{gps}-data nodig is.

Hierbij zien we gelijkaardige trend als bij het model van~\citeauthor{Dhondt},
met op de meeste scores een kleine afname ten opzichte ervan. Er slechts één
tussenstap is ten opzichte van het model van~\citeauthor{Dhondt}, namelijk het
omreken van snelheid en de tijd tot de totale afstand. Dit verklaart dan ook
meteen de kleine afnames en toenames in de resultaten. Deze omrekening zal een
kleine afwijking met zich meebrengen, waarschijnlijk door afrondingen en
mogelijke additionele berekeningen van Strava, bij het berekenen van de
snelheid.

\begin{table}[h]
    \centering
    \scalebox{0.55}{
        \begin{tabular}{lrrrrrrrr}
            \toprule
            {}         & Success Rate (\%) & Correctness (m) & Accuracy & Reduction (\%) & Uncertainty Region ($m^2$) & Certainty & Spatial Certainty & Degree of Anonymity (\%) \\
            Radius (m) &                   &                 &          &                &                            &           &                   &                          \\
            \midrule
            200        & 81.43             & 35.96           & 15       & 86.01          & 322.32                     & 1.91      & 0.68              & 28.33                    \\
            400        & 79.71             & 51.38           & 21       & 93.78          & 445.30                     & 2.26      & 0.92              & 27.80                    \\
            600        & 70.77             & 96.94           & 23       & 95.78          & 542.48                     & 2.33      & 1.18              & 27.34                    \\
            800        & 65.83             & 113.18          & 30       & 97.28          & 703.00                     & 2.48      & 1.41              & 27.38                    \\
            1000       & 62.39             & 191.47          & 31       & 97.60          & 698.69                     & 2.62      & 1.62              & 27.31                    \\
            1200       & 57.98             & 212.06          & 36       & 97.86          & 850.01                     & 2.62      & 1.76              & 27.13                    \\
            1400       & 49.15             & 270.35          & 29       & 98.54          & 648.70                     & 2.51      & 1.72              & 24.90                    \\
            \bottomrule
        \end{tabular}
    }
    \captionsetup{justification=centering}
    \caption{Aanval op basis van gegeven \textit{outer distance}, en snelheid}\label{tab:outerDistance}
\end{table}

\subsection{Ruwe gps-data}
De volgende aanval is deze zonder gegeven outer distance, maar ook zonder
smoothing. Dit zorgt ervoor dat de aanvaller de ruwe \ac{gps}-data gebruikt
voor het berekenen van de outer distance. In dit geval zit zowel de afwijking
die afkomstig is van de snelheidsomrekening, die besproken werd in het vorige
model (waarbij de outer distance gegeven is) alsook de afwijkingen die
afkomstig zijn van de \ac{gps}-data zelf.

De afwijkingen veroorzaakt wegen zoals verwacht relatief sterk door. Zeker bij
grotere radiussen heeft dit een grote impact op de resultaten. Vanaf een radius
van 1000 meter is de success rate zelfs 0\%. Maar ook bij de rest van de
metrieken zien we een aanzienlijk slechtere score die erger wordt bij een
hogere \ac{EPZ}-radius. Dit is te wijten aan de grote afwijkingen die de
\ac{gps}-data in zijn geheel met zich mee brengt, zeker bij grotere radiussen
weegt dit sterk door. Hoe groter de af te leggen afstand, in dit geval de inner
distance, hoe groter de fout.
\begin{table}[h]
    \centering
    \scalebox{0.55}{
        \begin{tabular}{lrrrrrrrr}
            \toprule
            {}         & Success Rate (\%) & Correctness (m) & Accuracy & Reduction (\%) & Uncertainty Region ($m^2$) & Certainty & Spatial Certainty & Degree of Anonymity (\%) \\
            Radius (m) &                   &                 &          &                &                            &           &                   &                          \\
            \midrule
            200        & 72.06             & 59.92           & 21       & 81.89          & 473.05                     & 2.22      & 1.01              & 33.43                    \\
            400        & 2.08              & 351.85          & 17       & 90.71          & 446.35                     & 2.15      & 1.67              & 27.80                    \\
            600        & 4.55              & 473.15          & 27       & 92.46          & 734.62                     & 2.57      & 2.17              & 30.67                    \\
            800        & 2.13              & 651.38          & 42       & 95.06          & 1161.95                    & 2.87      & 2.32              & 30.84                    \\
            1000       & 0.00              & 737.93          & 37       & 96.84          & 994.80                     & 2.76      & 2.22              & 29.69                    \\
            1200       & 0.00              & 955.79          & 22       & 97.63          & 592.09                     & 2.54      & 2.28              & 25.16                    \\
            1400       & 0.00              & 986.46          & 25       & 98.08          & 697.50                     & 2.44      & 2.21              & 23.70                    \\
            \bottomrule
        \end{tabular}
    }
    \captionsetup{justification=centering}
    \caption{Aanval op basis van ruwe \ac{gps}-locaties (geen smoothing) en snelheid}\label{tab:noSmoothing}
\end{table}

\subsection{Smoothing}
Het laatste model is hetzelfde als het voorgaande, maar nu wordt wel smoothing
gebruikt op de routes in een poging om de afwijkingen van de \ac{gps}-data te
verminderen. In Sectie~\ref{sec:berekeningen} bescreven we het mechanisme.
Zeker de extremen zullen worden afgevlakt, waardoor de fouten van een kleinere
orde zouden moeten hebben. Dit toont zich ook in de resultaten. Het gebruikte
smoothing-algoritme heeft één parameter die wij kunnen aanpassen, namelijk de
smoothing window. Deze parameter bepaalt hoeveel punten er per window worden
gecombineerd. Het optimale smoothing window wordt hier bepaald door het
uitvoeren van een aantal experimenten, en hiervan de resultaten met elkaar
vergelijken. Deze resultaten zijn terug te vinden in
Bijlage~\ref{ch:smoothing_results} op Tabel~\ref{tab:full_smoothing}. Deze zijn
ietwat wisselvallig, en er is geen duidelijke trend in terug te vinden.
% De hypothese is dat dit komt door de
% factor van willekeur die meespeelt bij het manueel opzetten van de \ac{EPZ}.
% Wanneer afwijkende punten net wel of net niet afgesneden worden door de
% \ac{EPZ}. 
Maar wanneer we zoeken naar het best scorende smoothing window, bekomen we uit
op een smoothing window van 100.

De success rate toont een relatief kleine verbetering ten opzichte van het
model met de ruwe \ac{gps}-data bij kleine \acp{EPZ}, maar bij grotere
\acp{EPZ} zwakt deze veel minder zwaar af. Ook de andere metrieken vertonen een
gelijkaardig patroon, waarbij de afzwakking veel minder sterk is dan bij het
gebruik van ruwe \ac{gps}-punten. Dit toont dat het gebruik van smoothing zeker
een significante toegevoegde waarde heeft. Let wel dit smoothing window
empirisch bepaald en optimaal gekozen voor deze dataset. Voor een andere
dataset kan het optimale window een verschillende waarde aannemen.
\begin{table}[h]
    \centering
    \scalebox{0.5}{
        \begin{tabular}{lrrrrrrrrr}
            \toprule
            {}         &                      & Success Rate (\%) & Correctness (m) & Accuracy & Reduction (\%) & Uncertainty Region ($m^2$) & Certainty & Spatial Certainty & Degree of Anonymity (\%) \\
            Radius (m) & Smoothing Window (n) &                   &                 &          &                &                            &           &                   &                          \\
            \midrule
            200        & 100                  & 75.0              & 61.37           & 20       & 82.22          & 450.15                     & 2.15      & 1.04              & 32.57                    \\
            600        & 100                  & 58.97             & 141.04          & 29       & 94.89          & 692.52                     & 2.47      & 1.61              & 29.51                    \\
            800        & 100                  & 56.34             & 217.13          & 36       & 96.30          & 773.61                     & 2.80      & 1.94              & 30.30                    \\
            1000       & 100                  & 41.27             & 234.27          & 35       & 97.43          & 802.93                     & 2.69      & 1.87              & 29.13                    \\
            1200       & 100                  & 44.12             & 278.00          & 39       & 98.06          & 953.93                     & 2.73      & 1.92              & 27.86                    \\
            1400       & 100                  & 32.81             & 294.24          & 34       & 98.28          & 841.94                     & 2.82      & 2.06              & 27.51                    \\
            \bottomrule
        \end{tabular}
    }
    \captionsetup{justification=centering}
    \caption{Aanval op basis van gesmoothe \ac{gps}-data en snelheid, met een empirisch bepaald optimaal smoothing window $n=100$}\label{tab:optimal_smoothing}
\end{table}

% Figuur: 3 stappen hoe een EPZ wordt opgezet
%%%%%%%%%%%%%%%%%%%%%%%%%%%%%%%%%%%%%%%%%%%%%%%%%%%%%%%%%%%%%%%%%%% 
%                                                                 %
%                            CHAPTER                              %
%                                                                 %
%%%%%%%%%%%%%%%%%%%%%%%%%%%%%%%%%%%%%%%%%%%%%%%%%%%%%%%%%%%%%%%%%%% 
\chapter{Conclusies en toekomstig werk}
\section{Conclusies}
Deze thesis toont aan dat de inferentie-aanvallen mogelijk zijn op basis van
snelheden en \ac{gps}-data. Bepaalde afstand data, meer bepaald de cumulatieve
afstand en de totale afstand zijn dus niet levensnoodzakelijk om deze aanval
succesvol uit te voeren. 

\section{Mogelijke extra privacymaatregelen}

\section{Toekomstig werk}
% Te kleine dataset
% Niet genoeg punten in verschillende dichtbebouwde zones

% Mogelijke countermeasures
% Beweegtijd niet weergeven
%  Beide punten in alle gevallen cloaken

% Heeft het nut om niet home locations te cloaken

% Experimenteren met smoothing
% Experitmenteren met map matching
% Ideale gevallen - toekomstig werk
% Meer variatie in de dataset - meer dichtbebouwde zones, verbebouwde zones
%  AFhankelijk van bebouwingsdichtheid succesrate vaststellen
% Kijken in hoeverre is een aaanval succesvol in functie van het aantal activiteiten die hij ter beschikking heeft

% Verder experimenteren met filter op gps-punten
%  Dynamische jump filter  (indien sprong groter dan 50m - gewoon vervangen door 50m)

% Fileren op dense locations, dus op de manier die ik d8 voor die LAD

%%%%%%%%%%%%%%%%% EXTRA TOEVOEGINGEN IN ZOMER %%%%%%%%%%%%%%%%%%%%%%%%
% IN H4: studie naar de stilstaande gebruiker
% \section{Stilstaande gebruiker}
% In Sectie~\ref{sec:definitie-aanvaller} wordt de aanvaller assumptie gemaakt
% dat een gebruiker niet mag stilstaan binnen de~\ac{EPZ}. Deze is enkel van
% toepassing indien de berekening gebeurt op basis van de totale verstreken tijd. Op Figuur~\ref{fig:time_diff} zien we dat de verschillen
%
%  
% Implementatie smoothing herwerken
%       Visualiseren wat gebeurt met de route als ik aan smoothing doe
% Map snapping implementeren
% 
% Verder experimenteren met filter op gps-punten
%       Dynamische jump filter  (indien sprong groter dan 50m - gewoon vervangen door 50m)
% 
% 
% Verschil moving time vs total time in de dataset + effect op de aanval
% 
% 
% Sensitiviteitsanalyse
% Bibliografie: referenties. De items zitten in bibliografie.bib
%%%%%%%%%%%%%%%%%%%%%%%%%%%%%%%%%%%%%%%%%%%%%%%%%%%%%%%%%%%%%%%%%
% Indien je ook de niet geciteerde werken in je bibliografie wil opnemen, commentarieer dan onderstaande regel uit!
\nocite{*}
\bibliographystyle{apalike} % Gebruik ieeetr voor nummers
\bibliography{bibliografie_nl}

% Eventueel enkele appendices
%%%%%%%%%%%%%%%%%%%%%%%%%%%%%%
\appendix
\input{bijlage1}
\chapter{Scientific Article}

\includepdf{article.pdf}{}

\chapter{Poster}

\includepdf{Poster.pdf}{}

\AtEndDocument{\includepdf{private/back_fiiw_gent.pdf}}
\end{document}