In a society where social media is so vastly present, the privacyconcerns
around them are more relevant than ever. While developing applications, privacy
laws and concerns must be taken into account. But this does not mean all these
platforms are bulletproof. In a lot of applications is it still possible to
find loopholes in the system, with the possibility of rather unpleasant
consequences. During this thesis, the main focus will be on the privacy
policies of fitnesstrackers. Fitnesstrackers are platforms which store and
display data related to sport activities. These can be shared with other users,
to show your achievements, and possibly motivating others to exercise as well.
This data may include heart rate, GPS-locations, \ldots. Some pieces could
potentially be more harmful than others. The focus of this thesis will be on
the harmful abilities gotten from GPS-locations and GPS-related data (like
speed, distance, \ldots). The main concern about sharing these locations, is
potentially sharing locations you would rather keep private. For example your
home location. Sharing all the GPS-locations of your activities could leak this
location.

Most of these kind of social media platforms are aware of this danger and
implement a series of countermeasures to prevent leaking this sensitive
information. This comes however with a cost, namely the user experience. In the
perspective of the developers of the fitnesstrackers, a trade-off is
consistently being made between privacy and user experience. On most platforms
like Strava, Garmin, \ldots, similar basic privacy features are implemented.
These are features like hiding activities, or only sharing activities with your
followers. But another much-used countermeasure is a mechanism known as an
\textit{EPZ} (Endpoint Privacy Zone).

An \textit{EPZ} is a circle or polygon drawn around a certain sensitive
location. The main focus will be on circular zones, because they are used the
most and they are the most straightforward to bypass. These circles will de
drawn using a radius chosen by the user, and a center which is a random point
in the area of around the sensitive location. This center can't be further than
70\% of the radius away from this sensitive location. When this zone is
generated, the end and beginning of the trajectory followed will be hidden
which pass through this zone will be hidden for other users.

This implementation of hiding parts of the track, is not a bulletproof system.
While hiding these parts, other useful information is not being hidden or
adapted to this sort of cloaking. During this thesis, the goal is to retrieve
sensitive location. This can be achieved by using the total times and distances
of the activities. Previous research showed that it is possible to retrieve
sensitive locations using the total distance combined with the street map of
the area. These attacks are called \textit{inference attacks}. The distance
traveled inside the EPZ can be inferred using the total distance given by the
activity, and the distance traveled outside of the EPZ (this is the visible
distance on the map). Using the distance traveled inside of the EPZ, a route
can be constructed and mapped onto the streetplan. If all the possible routes
are considered, multiple possible locations are found. If this is repeated for
different activities, with different points where the EPZ is being entered,
there will (in the best case) only one point remain. This would then be the
sensitive location.

During this thesis, a research is being held on the possibilities of such
inference attacks using other data as a base rather than distance. The main
focus will be on the speed and tempo of the activities, in combination with the
GPS-locations. This method will consist of three parts. First, the average
speed and the total duration will be used to calculate the total distance.
Second, the GPS-points will be used to calculate the distance traveled outside
of the \ac{EPZ}. In order to do this effectively, smoothing and map snapping
strategies need to be tested out to get the best possible results. These two
values can be used in the third step to execute the interference attack. The
results of this attack will be compared with the results of the previous
implementations of this sort of attack.

This attack is successful in some cases. With the correct tuning of the
parameters of the smoothing algorithm, a succes rate of 75\% can be achieved.
This is lower than previous implementations of this attack, which is logical
because of the type of data that is being used. Especially the GPS-locations
which are not always as accurate. And because there are so many points needed
small deviations on every point result in a large deviation on the calculated
distance. But the main conclusion is that this attack is possible, with a
significant succes rate.

\textbf{Keywords}: fitness-trackers, privacy, endpoint privacy zone, gps-locations, inference attack, speed