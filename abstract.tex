In a society where social media is so ubiquitous present, the privacy concerns
around them are more relevant than ever. While developing applications, privacy
laws and concerns must be taken into account. But this does not mean all these
platforms that where built whit those in mind are bulletproof. In a lot of
applications it is still possible to find vulnerabilities in the system, with
the possibility of rather unpleasant consequences. During this thesis, the main
focus will be on the privacy policies of fitness trackers. Fitness trackers are
platforms which store and display data related to sport activities. These can
be shared with other users, to show your achievements, and possibly motivating
others to exercise as well. This data may include heart rate, GPS-locations,
etc. Some pieces could potentially be more privacy-sensitive than others. The
relevant data to study in this thesis are GPS-locations and GPS-related data
(like speed, distance, \ldots). A great concern about sharing GPS-data, is
potentially sharing locations you would rather keep private, for example your
home location. Sharing full GPS data of your activities could leak this
location.

Most fitness tracking networks are aware of this danger and implement a series
of countermeasures to prevent leaking this sensitive information.
Countermeasures are coming however with a cost, namely a (slightly) worse user
experience. From the perspective of the developers of the fitness trackers, a
trade-off is consistently being made between privacy and user experience. On
most platforms like Strava, Garmin, similar basic privacy features are
implemented. These are features like hiding activities, or only sharing
activities with your followers. Another commonly used countermeasure is a
mechanism known as an \textit{EPZ} (Endpoint Privacy Zone).

An \textit{EPZ} is a circle or polygon drawn around a certain sensitive
location. The circular EPZ's will be drawn using a radius chosen by the user,
and a center which is a random point in the area of around the sensitive
location. This center can't be further than 70\% of the radius away from this
sensitive location. When this zone is generated, the end and beginning of the
trajectory followed which pass through this zone will be hidden for other
users.

Most EPZ implementation are not perfect in assuring privacy. While hiding these
parts, other useful information is not being hidden or adapted to this sort of
cloaking. During this thesis, the goal is to retrieve sensitive locations. This
can be achieved by using the total times and distances of the activities.
Previous research showed that it is possible to retrieve sensitive locations
using the total distance combined with the street map of the area. These
attacks are called \textit{inference attacks}. The distance travelled inside
the EPZ can be inferred using the total distance given by the API, and the
distance travelled outside of the EPZ (this is the visible distance on the
map). Using the distance travelled inside of the EPZ, a route can be
constructed and mapped onto the street plan. If all the possible routes are
considered, multiple possible locations are found. If this is repeated for
different activities, with different points where the EPZ is being entered,
only one point will remain (in the best case). This would then be the sensitive
location.

This thesis investigates the possibilities of such inference attacks using data
other than the distance as a base. In our implementation, the speed and tempo
of the activities will be used, in combination with the GPS-locations. This
method will consist of three parts. First, the average speed and the total
duration will be used to calculate the total distance. Second, the GPS-points
will be used to calculate the distance travelled outside of the \ac{EPZ}. In
order to do this effectively, smoothing and map snapping strategies need to be
tested out to get the best possible results. These two values can be used in
the third step to execute the interference attack. The results of this attack
will be compared with the results of the previous implementations of this sort
of attack.

This attack is successful in some cases. With the correct tuning of the
parameters of the smoothing algorithm, a succes rate of 75\% can be achieved.
This is lower than previous implementations of this attack, which was as
expected considering the type of data that is being used. The GPS locations in
particular are not always accurate. And because there are so many GPS-points
needed for these calculations, small deviations on every point result in a
large deviation on the calculated distance. But the main conclusion is that
this attack is possible, with a reasonable succes rate.

\textbf{Keywords}: fitness-trackers, privacy, gps-locations, endpoint privacy zone, inference attack