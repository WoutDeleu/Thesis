\documentclass[conference]{IEEEtran}
\IEEEoverridecommandlockouts{}
% The preceding line is only needed to identify funding in the first footnote. If that is unneeded, please comment it out.
\usepackage{cite}
\usepackage{amsmath,amssymb,amsfonts}
\usepackage{algorithmic}
\usepackage{graphicx}
\usepackage{textcomp}
\usepackage{xcolor}
\def\BibTeX{{\rm B\kern-.05em{\sc i\kern-.025em b}\kern-.08em
\kern-.1667em\lower.7ex\hbox{E}\kern-.125emX}}

%lijst van afkortingen toevoegen?

% \setlength{\parskip}{8pt}
\begin{document}

\title{\textbf{\LARGE Time Is Running Out\\
        \large Assessing Temporal Privacy of Privacy Zones in Fitness Tracking Social Networks}
}

\author{
    \IEEEauthorblockN{Deleu, Wout}
    \IEEEauthorblockA{
        \textit{KU Leuven, Campus Rabot} \\
        Ghent, Belgium
    }
}
\maketitle

\begin{abstract}
    In a society where social media is so ubiquitous, the privacy concerns
    around them are more relevant than ever.  During this article, the main
    focus will be on the privacy policies of fitness trackers. Fitness trackers are
    platforms which store and display data related to sport activities. These can
    be shared with other users. This data may include heart rate, GPS-locations,
    etc. This type of data sharing can however cause unintentionally sharing of
    sensitive information, like home addresses.

    Most fitness tracking networks are aware of this danger and implement a series
    of countermeasures to prevent this. One of these countermeasures is the use of
    Endpoint Privacy Zones (EPZs) which is a zone around a sensitive location,
    which hides the part of the trajectory which ends or begins in this zone.
    Previous research has shown that it is possible to retrieve the sensitive
    location using the available data from the activity. Dhondt et al.\ showed that
    based on the total distance travelled, the sensitive location can be retrieved
    using an `inference attack'~\cite{Dhondt}. This study will investigate the
    possibilities of such inference attacks using other data than the distance. We
    want to recreate the results as good as possible using the speed and tempo of
    the activity, together with GPS-locations. This can result in an attack model
    with a success rate up to 75\%. This is lower than the previous implementation
    of Dhondt et al., but this shows that the attack is still possible under
    circumstances where the distance is rendered unusable. This also includes some
    countermeasure described by Dhondt et al. But countermeasures like enlarging
    the EPZ or shifting endpoints still have effect.\vspace{10pt}

    \textbf{Keywords}: fitness-trackers, privacy, gps-locations, endpoint privacy
    zone, inference attack
\end{abstract}

\section{Introduction}

\bibliographystyle{IEEEtran}
\bibliography{bibliografie_eng}
\end{document}