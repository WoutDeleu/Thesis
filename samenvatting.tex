In een maatschappij waar sociale media alomtegenwoordig is, zijn de
privacybezorgdheden hierrond evenzeer erg actueel. Bij het ontwikkelen van
applicaties moeten privacywetgevingen en -bezorgdheden in acht genomen worden.
Dit neemt echter niet weg dat in heel wat applicaties nog gebreken te vinden
zijn in de uitvoering van het privacybeleid. In deze scriptie wordt de focus
gelegd op de uitvoering van het beleid binnen de fitnesstrackers. Dit zijn
platformen met als doel gegevens (die betrekking hebben op sportactiviteiten)
op te slaan en te delen met andere gebruikers. Dit zijn gegevens zoals
hartslag, gps-locaties, etc. Sommige van deze gegevens kunnen mogelijk
gevoelige informatie bevatten of vrijgeven. Gedurende deze thesis wordt
getracht om deze gevoelige informatie uit te buiten, met de nadruk op
gps-gerelateerde data. Het grootste gevaar bij het delen van deze locaties is
het vrijgeven van plekken die je liever niet deelt met de buitenwereld, zoals
bv.\ een woonplaats.

Heel wat van deze fitnesstrackers zijn zich bewust van de mogelijke gevaren en
gaan op gelijkaardige manieren te werk om de privacy van de gebruiker te
garanderen. Dit gaat echter ten koste van gebruiksvriendelijkheid. Vanuit het
perspectief van ontwikkelaars wordt de trade-off tussen privacy en
gebruiksvriendelijkheid constant gemaakt. Op de meeste platformen zoals
Strava\footnote{\url{https://www.strava.com/}} en Garmin worden gelijkaardige
privacy features geïmplementeerd. Bijvoorbeeld het verbergen van activiteiten
voor andere gebruikers, of enkel activiteiten weergeven voor je volgers. Een
andere veelgebruikte techniek is gekend als het gebruik van \textit{EPZs}
(Endpoint Privacy Zones). Een \textit{EPZ} is een cirkel, of bij uitzondering
een polygoon, opgezet rond een gevoelige locatie. Deze cirkels worden opgesteld
met een radius gekozen door de gebruiker. Het centrum van de EPZ zal een
willekeurig punt zijn in de buurt van de locatie in kwestie. Deze kan niet
verder dan 70\% van de radius verwijderd zijn van dete verbergen locatie in het
geval van Strava. Elk stuk van het afgelegde traject dat binnen deze zone ligt
zal worden verborgen voor de andere gebruikers.

Het verbergen van delen van de route is echter geen waterdicht
beveiligingsmechanisme, want hierbij worden bijhorende gegevens niet aangepast
of mee verborgen. De bijhorende afgelegde afstand wordt bijvoorbeeld niet
aangepast. Voorafgaand onderzoek toonde aan dat het mogelijk is om gevoelige
locaties te achterhalen door het gebruik van de totale duur en totale afgelegde
afstand van de activiteiten, in combinatie met het stratenplan van het gebied.
Dit soort aanvallen worden \textit{inferentieaanvallen} genoemd. Het traject
afgelegd binnenin de EPZ kan worden afgeleid met behulp van de totale afstand
van de activiteit en de zichtbare afstand, afgelegd buiten de EPZ.\ De afstand
binnenin de EPZ kan dan worden gemapt op het stratennetwerk, om zo alle
mogelijke routes te bekomen die de gebruiker kan afgelegd hebben binnenin de
EPZ.\ Door dit mechanisme toe te passen op alle activiteiten en geleidelijk aan
punten te schrappen die niet voor alle activiteiten een mogelijk eindpunt zijn,
kan een intersectie gevonden worden die uiteindelijk de gevoelige locatie
oplevert. Dit punt is dan de gevoelige locatie.

Deze thesis onderzoekt mogelijke implementaties van dergelijke
inferentieaanvallen in een situatie waarbij de afstand niet gekend is of
onbruikbaar is. Als alternatieve gegevens worden de snelheid en het tempo van
de activiteiten gebruikt, in combinatie met gps-punten. Deze gevolgde methode
bestaat uit drie delen. In de eerste stap beschouwen we de gemiddelde snelheid
en de totale duur om de totale afstand te berekenen. Ten tweede worden de
gps-punten gebruikt om de afgelegde afstand buiten de EPZ te berekenen. Om dit
zo accuraat mogelijk uit te kunnen voeren, bestuderen we smoothing- en
map-matchingstrategieën om de best mogelijke resultaten te verkrijgen. Deze
twee berekende waarden gebruiken we in de derde stap worden gebruikt om de
interferentieaanval uit te voeren. De resultaten van deze aanval zullen worden
vergeleken met de resultaten van eerdere implementaties van dit soort
aanval.the average or total travel distance gotten from running the simulation.
When the sample variance based on the k values is within an acceptable range,
the calculation is stopped and the value of k is chosen for the amount of
simulations that needs to be run. The acceptable deviation differs for the
average and total travel distance. For the average travel distance the
deviation chosen is 0.1. This is a low value, but given the speed of execution
and the and magnitude of this value, it was a achievable. For the total
distance the accepted deviation is 1000. If we look at the magnitude of the
total travel distance, we see it surpasses a million. A deviation of a
thousands seems in that case reasonable. The total distance travelled is a much
larger value than the average distance, because of this the accepted deviation
is larger. The deviation of the average distance travelled is 0.082 after 100
runs. The deviation of the total distance travelled is 997.08 after 120 runs.
The evolution of the deviation for both features is shown in figure ??. In the
graph of the average distance, the beginning has more fluctuations than the
graph of the total travel distance. This is because the deviation is far
smaller than the total distance deviation. From this information we can
conclude that at least 120 simulations must be done to get consistent results.

Met de juiste afstemming van de parameters van het smoothing-algoritme kan een
succespercentage tot 75\% worden bereikt. Dit is lager dan eerdere
implementaties van deze aanval, wat te verwachten is vanwege het type gegevens
dat wordt gebruikt. Voornamelijk doordat gps-data soms fouten bevat zoals
gps-drift, signaalverlies, gps-bounce, zal de afstand niet altijd even
nauwkeurig berekend kunnen worden. Doordat er zoveel punten nodig zijn,
resulteren kleine afwijkingen op elk punt in een grote afwijking op de
berekende afstand. Maar met dit onderzoek kunnen we aantonen dat een dergelijke
aanval mogelijk is en een aanzienlijke nauwkeurigheidsscore behaalt, ondanks
het ontbreken van de totale afstand. Dit houdt ook in dat een selectie van de
countermeasures die beschreven werden door~\citeauthor{Dhondt} niet meer
voldoende zijn om de privacy van een gebruiker te garanderen, en moeten worden
uitgebreid.

\textbf{Kernwoorden}: gps-locaties, privacy, endpoint privacy zone,
inferentieaanval, snelheid