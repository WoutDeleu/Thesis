
In een maatschappij waar sociale media alom aanwezig is, zijn de
privacybezorgdheden hierrond evenzeer erg actueel. Bij het ontwikkelen van
applicaties moeten privacywetgevingen en -bezorgdheden in acht genomen worden.
Maar dit neemt niet weg dat in heel wat applicaties nog gaten te vinden zijn in
het privacybeleid. In deze scriptie wordt de focus gelegd op het beleid binnen
de fitness-trackers. Dit zijn platformen met als doel gegevens (die betrekking
hebben op sportactiviteiten) op te slaan en weer te geven voor andere
gebruikers. Dit zijn gegevens zoals hartslag, \ac{gps}-locaties, \ldots.
Sommige van deze gegevens kunnen mogelijks ongewenste informatie bevatten of
vrijgeven. De focus van deze thesis ligt op het uitbuiten van deze mogelijks
schadelijke informatie, met de nadruk op gps-locaties en andere
gps-gerelateerde data. Het grootste gevaar van het delen van deze locaties is
het vrijgeven van locaties die je liever niet deelt met de buitenwereld, zoals
bv.\ een woonplaats.

Heel wat van deze platformen zijn zich bewust van de mogelijke gevaren gaan op
gelijkaardige manieren te werk om de privacy van de gebruiker proberen te
garanderen. Dit komt echter met een prijs, namelijk van
gebruiksvriendelijkheid. Volgens het perspectief van developers wordt de
trade-off tussen privacy en gebruiksvriendelijkheid constant gemaakt. Op de
meest platformen zoals Strava, Garmin, \ldots worden gelijkaardige privacy
features geïmplementeerd. Dit zijn features zoals het verbergen van
activiteiten voor andere gebruikers, of enkel activiteiten weergeven voor je
volgers. Maar een ander veelgebruikte techniek is gekend als het implementeren
van \textit{EPZ's} (Endpoint Privacy Zone).

Een \textit{EPZ} is een cirkel of polygoon opgezet rond een gevoelig
coördinaat. De focus zal hier gelegd worden op het gebruik van cirkels, omdat
deze het meest gebruikt worden. Deze cirkels worden opgesteld rond de gevoelige
locatie, met een radius gekozen door de gebruiker. Het centrum van de EPZ zal
een willekeurig punt zijn in de buurt van de locatie in kwestie. Deze kan niet
verder dan 70\% van de radius verwijderd zijn van de gevoelige locatie. Elk
stuk van het afgelegde traject dat binnen deze zone ligt zal worden verborgen
worden voor de andere gebruikers.

Het verbergen van delen van de route is echter geen waterdichte implementatie.
Want bij het verbergen van deze gps-locaties, worden bijhorende gegevens niet
aangepast of mee verborgen. Bijvoorbeeld wordt de totale afgelegde afstand
hieraan niet aangepast. Gedurende deze implementatie is het doel om de
gevoelige locatie te achterhalen. Voorafgaand onderzoek toonde aan dat dit
mogelijk is door het gebruik van de totale duur en totale afgelegde afstanden
van de activiteiten, in combinatie met het stratenplan van het gebied. Dit
soort aanvallen worden \textit{inference attacks} genoemd. De afstand afgelegd
binnenin de EPZ kan worden afgeleid met behulp van de totale afstand en de
afstand afgelegd buiten de EPZ (de zichtbare afstand). De afstand binnenin de
EPZ kan worden gemapt op het stratennetwerk, om zo alle mogelijke routes te
bekomen die de gebruiker kan afgelegd hebben binnenin de EPZ.\ Wanneer voor
alle activiteiten die voor deze gebruiker ter beschikking zijn dit mechanisme
wordt toegepast, zal in het slechtste geval slechts één punt overblijven,
doordat geleidelijk aan punten zullen worden kunnen geschrapt omdat deze niet
voor alle activiteiten een mogelijk eindpunt zullen zijn. Dit punt is dan de
gevoelige locatie.

In deze thesis wordt onderzoek gedaan naar mogelijke implementaties van
dergelijke inferentie-aanvallen, met andere gegevens dan de totale afstand als
basis. De focus zal hier liggen op snelheid en tempo van de activiteiten, in
combinatie met gps-punten. Deze gevolgde methode bestaat uit drie delen. In de
eerste stap zullen de gemiddelde snelheid en de totale duur worden gebruikt om
de totale afstand te berekenen. Ten tweede zullen de GPS-punten gebruikt worden
om de afgelegde afstand buiten de \ac{EPZ} te berekenen. Om dit effectief te
doen, moeten smoothing- en map-matchingstrategieën uitgetest worden om de best
mogelijke resultaten te verkrijgen. Deze twee berekende waarden kunnen in de
derde stap worden gebruikt om de interferentie-aanval uit te voeren. De
resultaten van deze aanval zullen worden vergeleken met de resultaten van
eerdere implementaties van dit soort aanval.

Deze methodiek kan in vele gevallen succesvol worden uitgevoerd. Met de juiste
afstemming van de parameters van het smoothing-algoritme kan een
succespercentage van 75\% worden bereikt. Dit is lager dan eerdere
implementaties van deze aanval, wat logisch is vanwege het type gegevens dat
wordt gebruikt. Voornamelijk de GPS-locaties die niet altijd even nauwkeurig
zijn. Doordat er zoveel punten nodig zijn resulteren kleine afwijkingen op elk
punt in een grote afwijking op de berekende afstand. Maar de belangrijkste
conclusie is dat deze aanval mogelijk is, met een aanzienlijke kans op succes.

\textbf{Keywords}: fitness-trackers, privacy, endpoint privacy zone,
\ac{gps}-locaties, inference attack, snelheid