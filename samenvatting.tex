% De (korte) samenvatting, toegankelijk voor een breed publiek, wordt in het Nederlands geschreven en bevat {\bf maximum 3500 tekens}. Deze samenvatting moet ook verplicht opgeladen worden in KU Loket.

\section{Situering}
In een maatschappij waar sociale media alom aanwezig is, zijn de
privacybezorgdheden hierrond evenzeer erg actueel. Bij het ontwikkelingen van
applicaties moeten privacywetgevingen en -bezorgdheden in acht genomen worden.
Maar dit neemt niet weg dat in heel wat applicaties nog gaten te vinden zijn in
het privacybeleid. Gedurende deze thesis worden gekende fitness-trackers onder
de loep genomen, waaronder Strava. Er is op te merken dat heel wat van deze
platformen op gelijkaardige manieren proberen privacy te garanderen. In de
meeste gevallen gaat dit over het verbergen van een stuk van de activiteit, en
zo de start- en/of eindpositie niet weer te geven op de kaart. Het verbergen
van deze activiteit gebeurd door het opstellen van een \textit{Endpoint Privacy
    Zone}. Hierbij wordt een cirkel opgesteld waarbinnen de afgelegde weg zal
worden verborgen. In de paper van
\citeauthor{Dhondt_Pochat_Voulimeneas_Joosen_Volckaert_2022} is een manier
terug te vinden om door het combineren van verschillende gegeven afstanden
(bijvoorbeeld totale afstand, afstand tussen 2 punten, \ldots) en het bepalen
van deze EPZ, de effectieve startpositie van de activiteit te achterhalen en zo
gevoelige informatie bloot te leggen.

\section{Doel}
In deze thesis zal getracht worden om alternatieve manieren te vinden om deze
gevoelige locatie te bepalen aan de hand van andere metadata zoals snelheid.
Hierbij zal ook de effectiviteit ervan geanalyseerd worden, en zal getracht
worden om enkele manieren te vinden om de privacy van het platform te verhogen.
Er zal getracht worden deze werkwijze toe te passen aan de hand van
verscheidene beschikbare metadata.