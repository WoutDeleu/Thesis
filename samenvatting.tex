% De (korte) samenvatting, toegankelijk voor een breed publiek, wordt in het Nederlands geschreven en bevat {\bf maximum 3500 tekens}. Deze samenvatting moet ook verplicht opgeladen worden in KU Loket.

In een maatschappij waar sociale media alom aanwezig is, zijn de
privacybezorgdheden hierrond evenzeer erg actueel. Bij het ontwikkelingen van
applicaties moeten privacywetgevingen en -bezorgdheden in acht genomen worden.
Maar dit neemt niet weg dat in heel wat applicaties nog gaten te vinden zijn in
het privacybeleid. In deze scriptie wordt de focus gelegd op het beleid binnen
de fitness-trackers. Dit zijn platformen met als doel gegevens (die betrekking
hebben op sportactiviteiten) op te slaan en weer te geven voor andere
gebruikers. Dit zijn gegevens zoals hartslag, gps-locaties, \ldots. Heel wat
van deze platformen gaan op gelijkaardige manieren te werk om de privacy van de
gebruiker proberen te garanderen. In de meeste gevallen gaat dit over het
verbergen van een stuk van de activiteit, en zo de start- en/of eindpositie
niet weer te geven op de kaart. Het achterhouden van dit routesegment gebeurd
door het opstellen van een \textit{Endpoint Privacy Zone}. Hierbij wordt een
cirkel opgesteld waarbinnen de afgelegde weg zal worden verborgen. In het
verleden werd reeds aangetoond dat dit zeker geen waterdicht systeem is. Een
tekortkoming van dit systeem die eerder beschreven is, maakt gebruik van
gegeven afstanden die terug te vinden zijn in metadata die het desbetreffende
platform vrijgeeft. Hiermee wordt de EPZ bepaalt, en zo kan uiteindelijk ook de
effectieve startpositie van de activiteit achterhaald worden en zo deze
gevoelige informatie bloot leggen
\cite{Dhondt_Pochat_Voulimeneas_Joosen_Volckaert_2022}.

OMSCHRIJVEN WAT HIER BERIJKT WERD