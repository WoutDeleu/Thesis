% De (korte) samenvatting, toegankelijk voor een breed publiek, wordt in het Nederlands geschreven en bevat {\bf maximum 3500 tekens}. Deze samenvatting moet ook verplicht opgeladen worden in KU Loket.

In een maatschappij waar sociale media alom aanwezig is, zijn de
privacybezorgdheden hierrond evenzeer erg actueel. Bij het ontwikkelingen van
applicaties moeten privacywetgevingen en -bezorgdheden in acht genomen worden.
Maar dit neemt niet weg dat in heel wat applicaties nog gaten te vinden zijn in
het privacybeleid. In deze scriptie wordt de focus gelegd op het beleid binnen
de fitness-trackers. Dit zijn platformen met als doel gegevens (die betrekking
hebben op sportactiviteiten) op te slaan en weer te geven voor andere
gebruikers. Dit zijn gegevens zoals hartslag, gps-locaties, \ldots. Heel wat
van deze platformen gaan op gelijkaardige manieren te werk om de privacy van de
gebruiker proberen te garanderen. In de meeste gevallen gaat dit over het
verbergen van een stuk van de activiteit, en zo de start- en/of eindpositie
niet weer te geven op de kaart. Het achterhouden van dit routesegment gebeurd
door het opstellen van een \textit{Endpoint Privacy Zone}. Hierbij wordt een
cirkel opgesteld waarbinnen de afgelegde weg wordt verborgen. In het verleden
werd reeds aangetoond dat dit zeker geen waterdicht systeem is. Een
tekortkoming van dit systeem, is het feit dat gebruik gemaakt wordt van gegeven
afstanden die terug te vinden zijn in metadata die het desbetreffende platform
vrijgeeft. Hiermee wordt de \textit{Endpoint Privacy Zone} \textbf{EPZ}
bepaalt, en zo kan uiteindelijk ook de effectieve startpositie van de
activiteit achterhaald worden en zo deze gevoelige informatie bloot
leggen~\cite{Dhondt_Pochat_Voulimeneas_Joosen_Volckaert_2022}.

Gedurende deze thesis wordt onderzoek gedaan naar manieren om deze EPZ te
omzeilen, via metadata die terug te vinden is in de activiteiten in kwestie. De
focus gedurende deze thesis ligt op snelheidsdata. Er wordt voornamelijk
gezocht naar de omstandigheden waaronder deze aanval succesvol zou kunnen zijn,
bij welke eigenschappen van de activiteit. En er wordt ook gezocht naar
manieren om deze aanval in de best mogelijke omstandigheden uit te voeren. De
studie focust zich voornamelijk op de berekeningen van afstanden en snelheden
a.d.h.v.\ coördinaten. Het doel is om de EPZ te omzeilen, ervan uitgaande dat
de desbetreffende fitnesstrackers afstandsinformatie achterhouden. Een grote
focus ligt dus op het berekenen van afstanden via gps punten met een zo hoog
mogelijke precisie. Een belangrijke techniek hierbij is gps-smoothing.

% Aanvullen met data, resultaten van wat ik gedaan heb... zie tibo verdonck

\textbf{Keywords}: fitness-trackers, privacy, endpoint privacy zone,
gps-locaties, inference attack, snelheid