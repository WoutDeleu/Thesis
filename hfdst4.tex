%%%%%%%%%%%%%%%%%%%%%%%%%%%%%%%%%%%%%%%%%%%%%%%%%%%%%%%%%%%%%%%%%%% 
%                                                                 %
%                            CHAPTER                              %
%                                                                 %
%%%%%%%%%%%%%%%%%%%%%%%%%%%%%%%%%%%%%%%%%%%%%%%%%%%%%%%%%%%%%%%%%%% 
\chapter{Gebruikte data en afwijkingen}
%  Beschrijven hoe zo'n aanval werkt - Karels methodiek uitwerken
% Afstanden wijken af
% Fouten in GPS data
% Fouten bij mappen GPS op het Strava routeplan
% Bij het aan en uitzetten van Strava - route verspringt!
% Afwijkende punten - zie grafiek

% Afstanden mogelijks berekenen op 2 manieren 
% 1. Coordinaten 
% 2. Cumulatieve afstanden van de punten

% Werkwijze
%     1. Bereken tijd in de EPZ
%           Wanneer ge stil staat... in de EPZ
%           Eerste - laatste tijd
%           Snelheid
%               Zelf berekenen (Maar komt overeen met degene gegeven door Strava - Strava geeft sommige weer als NULL, dus kan er niet mee werken)
%               Strava: m/h
%               Zelf: m/s
%     2. Berkenen outerdistance op verschillende manieren
%     2. Berekenen inner distance
%           Voor complete dataset => visualize verschillen tss de berekende en de gegeven inner distance
%           opm: Wanneer je stil staat in den EPZ => aanval niet mogelijk (Gemiddelde snelheid obv moving time <-> Mijn methode gebruikt elapsed_time)
%           haversine => Vectorized
%       Methodiek volledig uitschrijven + analyse van deze waarden

% Bepalen bij welke omstandigheden het werkt/niet werkt

% Basseren op totale verstreken tijd of niet?

% Welke gps afwijkingne zijn er\