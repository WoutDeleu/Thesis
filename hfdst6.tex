%%%%%%%%%%%%%%%%%%%%%%%%%%%%%%%%%%%%%%%%%%%%%%%%%%%%%%%%%%%%%%%%%%% 
%                                                                 %
%                            CHAPTER                              %
%                                                                 %
%%%%%%%%%%%%%%%%%%%%%%%%%%%%%%%%%%%%%%%%%%%%%%%%%%%%%%%%%%%%%%%%%%% 
\chapter{Conclusies en toekomstig werk}
\section{Conclusies}
Deze thesis toont aan dat de inferentie-aanvallen mogelijk zijn op basis van
snelheden en \ac{gps}-data, weliswaar met een lagere success rate en grotere
onzekerheid. Indien de cumulatieve afstand ter beschikking is, kan via een
eenvoudige omrekening ($inner\ distance = total\ time \times average\ pace -
      outer\ distance$) de aanval nog steeds worden uitgevoerd met een success rate
van 81.43\%, 79.71\%, 70.77\%, 65.83\%, 62.39\%, 57.98\%, 49.15\% voor de
desbetreffende radii: $200m$, $400m$, $600m$, $800m$, $1000m$, $1200m $ en
$1400m$, wat absoluut aanvaardbaar is. Indien de cumulatieve afstandsdata op
zijn beurt niet voor handen is, kan via \ac{gps}-locaties een additionele
omrekening gebeuren om de aanval alsnog te doen slagen. Door het gebruik van
smoothing algoritmes kan de aanval ook in precisie winnen, en komen we tot een
success rate van 75.0\%, 58.97\%, 56.34\%, 41.27\%, 44.12\%, 32.81\% voor de
desbetreffende radii en een smoothing window van 100. Dit is op zijn beurt een
acceptabele success rate, maar is wel een stuk lager dan de success rate
bekomen door~\citeauthor{Dhondt}~\cite{Dhondt}.

Bepaalde afstandsdata, meer bepaald de cumulatieve afstand en de totale afstand
zijn dus niet van cruciaal belang voor het succesvol uitvoeren van dit
aanvalsmodel. Let wel, de dataset die ter beschikking werd gesteld in deze
thesis is echter aan de kleine kant. De getrokken conclusies in deze thesis
mogen dus niet zomaar worden geëxtrapoleerd naar de volledige groep van
Stravagebruikers.

Dhondt et al.\ beschreven enkele maatregelen om de privacy van gebruikers te
verzekeren, zoals het afronden van afstandsdata of het toevoegen van ruis aan
deze data om deze zo onbruikbaar te maken~\cite{Dhondt}. Dit is dus niet meer
van toepassing, alle \textit{`Distance-Focused Countermeasures'} beschreven
door~\citeauthor{Dhondt} kunnen worden omzeild door ons aanvalsmodel. De
aanvaller kan de afstandsdata in theorie herberekenen, en ultiem nog steeds de
aanval tot een goed einde brengen. Alle distance based maatregelen die
\citeauthor{Dhondt} beschreven zijn dus in principe niet meer bruikbaar. We
zouden deze countermeasures in principe kunnen uitbreiden door hierbij de
snelheid en/of tijd in rekening te brengen. Bijvoorbeeld bij generalisation,
wat neerkomt op afronding van een afstand die zichtbaar is naar de
buitenwereld, kan de snelheid op dezelfde manier een manipulatie ondergaan.
Voor alle countermeasures in deze categorie, behalve \textit{Shifting
      Distances} geldt dit principe, namelijk dat deze teniet worden gedaan ten
opzichte van ons aanvalsmodel, maar indien deze uitgebreid worden naar de tijd
en/of snelheid zouden deze alsnog nuttig kunnen blijken. Alhoewel niet
expliciet getest, is de hypothese dat alle countermeasures, indien uitgebreid
naar de snelheid en/of tijd, een gelijkaardig effect zullen hebben. Het is en
blijft natuurlijk een afweging tussen privacy en gebruiksvriendelijkheid. De
besproken countermeasures zijn:
\begin{itemize}
      \item \textit{Generalisation} houdt in dat de afstand wordt afgerond tot een bepaalde
            precisie.~\citeauthor{Dhondt} stelt een precisie van 500 meter
            voor~\cite{Dhondt}.
      \item \textit{Noisy Distances} is een techniek waarbij een willekeurige waarde wordt
            toegevoegd of afgetrokken van de totale afstand~\cite{Dhondt}.
      \item \textit{Shifting Distances} verschuift de zichtbare begin- of eindpunten van een activiteit met een willekeurige afstand, in een willekeurige richting,
            zodat het vertrekpunt onzeker wordt~\cite{Dhondt}. Dit zorgt voor problemen bij het reconstrueren van de route.
      \item \textit{Truncation} stelt dat het verborgen traject niet mee in rekening wordt gebracht in de
            totale afstand~\cite{Dhondt}. Dit verwijdert eigenlijk het verborgen traject volledig uit de activiteit.
\end{itemize}
De countermeasure \textit{Shifting Distances} is de enige die niet kan worden
omzeild door onze implementatie, doordat voor ons model ook de
zichtbare begin- en eindpunten worden genomen, en een vitaal onderdeel van de
aanval zijn, net zoals bij het model door~\citeauthor{Dhondt}.

In het geval van de andere categorie van countermeasures, namelijk de
`EPZ-Focused Countermeasures', is de verwachting dat deze wel nog steeds gelden
als nuttige countermeasures. Deze countermeasures zijn:
\begin{itemize}
      \item \textit{Increasing \ac{EPZ} radii} wat, zoals de naam doet vermoeden inhoudt dat de \ac{EPZ} wordt vergroot~\cite{Dhondt}. We merken op dat de aanval
            aanzienlijk minder presteert bij grote \ac{EPZ} radiussen.
      \item \textit{Complex \ac{EPZ} shapes} houdt in dat de \ac{EPZ} niet langer een cirkel is, maar een
            complexere vorm aanneemt~\cite{Dhondt}. Dit kan bijvoorbeeld een veelhoek zijn.
\end{itemize}

Als laatste merken we wel op dat onze aanval enkel mogelijk is bij activiteiten
waarvan slechts het begin- of eindpunt wordt gecloaked, nooit beide. Indien
beide punten worden gecloaked, is het onmogelijk om de route te reconstrueren
volgens ons model. Dit is een beperking van ons model, maar resulteert wel in
een mogelijke countermeasure. Indien de snelheid en tijd van een activiteit
onaangetast blijven, maar een \textit{Distance-Focused Countermeasure} wordt
toegepast, zou het volstaan om beide punten te cloaken om de aanval die we
beschrijven te voorkomen.

\section{Toekomstig werk}
Naast de volledige uiteenzetting van dit onderzoek, zijn er nog enkele zaken
die misschien niet in detail genoeg werden onderzocht, en die ook interessante
resultaten zouden kunnen naar voor brengen. Ook zijn er bepaalde onderwerpen
die hierop verder bouwen en ook het onderzoeken waard kunnen zijn. Daaronder
valt het implementeren van het beschreven map matching, wat eventueel zou
kunnen leiden tot een hogere success rate bij het uitvoeren van de aanval.
Zeker de combinatie met het smoothing mechanisme kan een interessant onderwerp
zijn. Hierbij zou ook nog kunnen geëxperimenteerd met een dynamisch smoothing
window~\cite{shmoothing}. Dit is een smoothing window dat zich aanpast aan de
eigenschappen van het signaal, zoals de ruis of veranderingssnelheid van de
grafiek. Bijvoorbeeld dat de breedte van het venster verhoogt bij grote
hoeveelheden ruis is een mogelijke implementatie.

Ook zou er een additionele analyse kunnen plaatsvinden om te zien onder welke
externe omstandigheden de aanval het meest succesvol is. Er zou bijvoorbeeld
kunnen worden getest bij welke bebouwingsdichtheid de aanval het meest
succesvol is. Een belangrijk eigenschap die hierbij zeker ook aan bod moet
komen is de hoeveelheid activiteiten die zorgen voor een succesvolle aanval.

Daarnaast zou in een bijkomstig onderzoek dat een wat bredere invloed heeft op
het onderzoeksdomein kunnen worden onderzocht in hoeverre omgevingsfactoren
zoals bebouwingsdichtheid een invloed hebben op de success rate en dergelijke
en of de success rate niet dynamisch kan worden gesteld op basis van deze
dichtheid. Een straat die slechts één huis bevat is immers al te onderscheiden
met een veel lagere precisie dan een met meer inwoners in een stad. De $22.5$
meter die nu empirisch werd bepaald zou kunnen aangepast worden op basis van de
bebouwingsdichtheid.

Als laatste is er een idee om een ander implementatie van de \ac{LAD}-regressie
te gebruiken. Nu wordt voor elke activiteit de afwijking berekend ten opzichte
van de nodes in de graaf, en wordt de node gekozen die over de volledige lijn
de laagste afwijking bevat. Een alternatief zou kunnen zijn om voor elke
activiteit een node te kiezen die de laagste afwijking heeft ten opzichte van
die activiteit. En uiteindelijk de node te verkiezen die in totaal de laagste
totale som van alle onderlinge afstanden heeft. Met andere woorden, degene die
het meest gecentreerd ligt. De \ac{LAD}-methode die beschreven werd in deze
thesis gaat op zoek naar een zo laag mogelijke afwijking voor elke activiteit,
uitgaande van het ideale scenario waarbij elke activiteit stopt of begint op
dezelfde locatie. In de hier alternatief beschreven \ac{LAD}-methode vertrekken
we eerder vanuit het feit dat activiteiten alternatieve locaties aanwijzen
volgens hun afstand, door het feit dat ze effectief niet op dezelfde locatie
eindigen. Gebruikers stoppen eens wat verder, eens wat vroeger, enzovoort. De
alternatieve implementatie van de \ac{LAD}-methode zou dit dus voor een stuk
kunnen opvangen.

% Een laatste idee is een nieuw soort \ac{EPZ}-mechanisme uit te werken. We
% hebben het idee om een \ac{EPZ}-mechanisme te ontwerpen die werkt met een
% onzekerheidsband.

% Verder experimenteren met filter op gps-punten
%  Dynamische jump filter  (indien sprong groter dan 50m - gewoon vervangen door 50m)

% 
% 
% 
%%%%%%%%%%%%%%%%% EXTRA TOEVOEGINGEN IN ZOMER %%%%%%%%%%%%%%%%%%%%%%%%
% IN H4: studie naar de stilstaande gebruiker
% \section{Stilstaande gebruiker}
% In Sectie~\ref{sec:definitie-aanvaller} wordt de aanvaller assumptie gemaakt
% dat een gebruiker niet mag stilstaan binnen de~\ac{EPZ}. Deze is enkel van
% toepassing indien de berekening gebeurt op basis van de totale verstreken tijd. Op Figuur~\ref{fig:time_diff} zien we dat de verschillen
%
%  
% Implementatie smoothing herwerken
%       Visualiseren wat gebeurt met de route als ik aan smoothing doe
% Map snapping implementeren
% ==>> Voor beide: vergelijken met hetgeen strava doet, en die figuur plotten
% 
% Verder experimenteren met filter op gps-punten
%       Dynamische jump filter  (indien sprong groter dan 50m - gewoon vervangen door 50m)
% 
% 
% Verschil moving time vs total time in de dataset + effect op de aanval
%   => Effect onderzoeken; 
%   => In de conclusie dit meegeven als mogelijke countermeasure, geen beweegtijd meegeven
% 
% 
% Sensitiviteitsanalyse