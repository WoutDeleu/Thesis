%%%%%%%%%%%%%%%%%%%%%%%%%%%%%%%%%%%%%%%%%%%%%%%%%%%%%%%%%%%%%%%%%%% 
%                                                                 %
%                            CHAPTER                              %
%                                                                 %
%%%%%%%%%%%%%%%%%%%%%%%%%%%%%%%%%%%%%%%%%%%%%%%%%%%%%%%%%%%%%%%%%%% 

\chapter{Achtergrond}

\section{Fitnesstrackers}
Zoals al enkele malen werd aangehaald, ligt de focus van deze scriptie op
mogelijke tekortkomingen/vulnerabilities betreffende privacybeleid in
fitnesstrackers. Maar voordat een aanval op basis van deze kwetsbaarheden kan
opgezet worden, is het noodzakelijk om een te vat te krijgen op welke manier
een fitnesstracker info verzamelt en weergeeft. En meer precies, hoe de
mechanismen die de privacy voorzien voor de gebruikers in detail werken.

De data waarmee de aanval wordt opgezet en waarmee wordt geëxperimenteerd, is
afkomstig van de populaire fitnesstracker
\textit{Strava\footnote{\url{https://www.strava.com/}}}. Dit is een sociaal
netwerk, waarbij alle soorten sporters hun activiteiten kunnen delen. Dit gaat
over lopen, wandelen, fietsen, zwemmen, \ldots, maar ook sporten als fitnessen,
voetballen, \ldots De verzamelde data wordt volgens het perspectief van een
mogelijke aanval gefilterd. Enkel data die gevoelige informatie met betrekking
tot woonplaats zou kunnen vrijgeven wordt behouden. Dit zal er dus op neerkomen
dat enkel activiteiten die relevante gps-informatie bevatten in beschouwing
worden genomen. Dit gaat dan meer specifiek over \textit{runs, hikes, walks,
    and rides}.

\subsection{Activiteiten}\label{data}
Een Strava activiteit bevat erg veel informatie. Echter is niet alles even
bruikbaar. Een correcte abstractie van de onnodige data is dus nodig.
Figuur~\ref{fig:activityData} geeft een voorbeeld van een gedetailleerde
activiteit weer. Een gebruiker is in staat om de activiteit een titel te geven,
en er een korte beschrijving aan toe te voegen. Ook een foto kan optioneel
toegevoegd worden. De exacte datum en tijd van de start van de activiteit wordt
hierbij ook weergegeven.

Rechts daarvan zijn de algemene basisstatistieken te zien. Deze zijn de totale
afgelegde afstand, de totale bewegingstijd, de gemiddelde snelheid, het totale
hoogteverschil, de totale verstreken tijden, en het aantal calorieën verbrand.
Als extra kunnen hier enkele statistieken m.b.t.\ het gebruikte materiaal,
zoals type fiets, loopschoenen, hartslagmeter, enzovoort worden weergegeven.
Een belangrijk onderscheid in deze context is het verschil tussen de beweegtijd
en de verstreken tijd. Deze twee lijken in definitie gelijk, maar dit zijn ze
niet. Strava, en vaak fitnessplatformen in het algemeen werken met twee
verschillende soorten tijdsberekeningen voor het bekomen van een accuratere
gemiddelde snelheid. De verstreken tijd is simpelweg het tijdsinterval tussen
het vertrek van de activiteit en de aankomsttijd ervan. De bewegingstijd is de
tijd waarbij de gebruiker zich effectief bewoog. Met andere woorden worden de
tijden waarbij de gebruiker stilstond uit de verstreken tijd gefilterd. Dit kan
gaan over bijvoorbeeld een pauze, of het wachten voor een verkeerslicht. De
snelheid wordt berekend aan de hand van de bewegingstijd. Dit kan simpel worden
geverifieerd via een manuele berekening volgens de formule voor het berekenen
van gemiddelde snelheid\footnote{$ \quad v(\frac{min}{km}) =
        \frac{t(\min)}{d(km)}$}, met de data die terug te vinden is op
Figuur~\ref{fig:activityData} ($\frac{(39:17)\min}{7.44 km} = 5:16
\frac{\min}{km}$). Een kanttekening hierbij is dat dit enkel geldt voor
activiteiten die niet gelabeld zijn als \textit{race}, dan wordt de snelheid
berekend in functie van de totaal verstreken tijd~\cite{MovingTi80:online}.

Onder de basisstatistieken zijn de \textit{Strava-segmenten} te zien. Een
Strava-segment is een specifiek deel van een bepaalde route dat door gebruikers
van de sport-app kan worden gemarkeerd, gedeeld en vergeleken met andere
gebruikers. Het segment is een bepaalde afstand en route, bijvoorbeeld een klim
of afdaling, die vaak wordt beschouwd als een uitdagende of iconische sectie
van een bepaalde fiets- of hardlooproute. Gebruikers van Strava kunnen een
segment maken door de begin- en eindpunten op een kaart aan te geven en een
naam en beschrijving toe te voegen. Zodra het segment is gemaakt, kunnen andere
gebruikers het segment vinden en deelnemen aan een leaderboard, waarop de
snelste tijden worden bijgehouden en vergeleken met andere gebruikers.
Segmenten worden vaak gebruikt om prestaties te meten en te vergelijken.

Centraal op de figuur is ook de kaart duidelijk zichtbaar. Daarbij horen ook de
tussentijden en de grafiek van snelheid. Optioneel kan hierbij ook nog een
visualisatie van de afgelegde hoogte en de hartslag worden weergegeven, indien
de gebruiker hiervoor met de juiste meetinstrumenten zijn sportactiviteit
opneemt. De tussentijden en de grafiek van snelheid zijn qua inhoud
gelijkaardig, met als verschil dat deze erg precies kan worden bestudeerd. Op
de grafiek is voor elk afstandspunt de ogenblikkelijke snelheid zichtbaar. Bij
de tussentijden wordt de gemiddelde snelheid over een kilometer weergegeven. De
kaart die de route weergeeft is zeker ook belangrijk om even te bestuderen.
Deze bevat namelijk alle gps-geregistreerde punten, en verbindt deze ook om zo
één aaneensluitende route te vormen. Wanneer deze echter in detail bestudeerd
wordt, samen met de legende die aanwezig is, is te zien dat de route uit twee
delen bestaat, een zichtbaar deel en een onzichtbaar deel. Een andere gebruiker
zal enkel zicht hebben op de het zichtbare deel, het onzichtbare deel zal dus
voor een andere gebruiker niet zichtbaar zijn. Anders geformuleerd, de
activiteit zal voor deze persoon dus als het ware afgekapt zijn, en zal in zijn
zichtbare versie op een andere plek starten en eindigen. In de volgende
Secties~\ref{Algemene Privacy} \&~\ref{EPZ} wordt meer in detail ingegaan op de
werking van deze methodiek.

Een laatste kanttekening die hierbij gemaakt moet worden, is dat voor een
gebruiker verschillende eenheden mogelijk zijn om uit te kiezen. Er is keuze
mogelijk tussen de mijl en pond, en kilometer en kilogram. Gebruikers kiezen in
welke eenheid ze de applicatie wensen te gebruiken. Voor de gebruiker in
kwestie zal dus de volledige applicatie worden weergegeven in de gekozen
eenheden.
\begin{figure}
    \centering
    \includegraphics[width=0.8\textwidth]{fig/VoorbeeldActiviteiten/VoorbeeldActiviteit_Personal.png}
    \caption{Data van een activiteit}\label{fig:activityData}
\end{figure}

\subsection{Berekening Afstanden}
Fitnesstrackers krijgen vanuit de buitenwereld ruwe data binnen. Deze data moet
dus verwerkt worden vooraleer ze bruikbaar is voor de gebruiker. Er werd al
kort ingegaan in Sectie~\ref{data} op de berekening die Strava gebruikt voor de
snelheid. Echter is het ook interessant om de berekening van Strava eens onder
de loep te nemen voor de afgelegde afstand. Strava maakt gebruik van twee
verschillende methodieken voor het berekenen van deze afstand. De eerste is de
\textit{GPS-calculated Distance}. Dit bestaat eruit om de afstand tussen
opeenvolgende gps-punten te berekenen, en deze op te tellen. Precisie is hier
afhankelijk van de precisie van de gps-punten, aangezien de afstand wordt
berekend door de punten met rechte lijnen te verbinden. Dit kan gebeuren in
real time, via de gsm, smartwatch of ander toestel die gebruikt wordt om de
activiteit op te nemen. Er zal dan ook mogelijkheid zijn om real time info te
zien. Op elk punt zal de afstand vanaf het startpunt gekend zijn, en het is
deze afstand die gedeeld zal worden op het platform. Het grote nadeel hierbij
is het real-time aspect. Fouten kunnen moeilijker on the fly worden
gecorrigeerd. Een tweede aanpak is om gps-data pas bij het uploaden te
verwerken. De gps-data wordt dan geanalyseerd, en de nodige berekeningen worden
uitgevoerd.

Het alternatief voor de GPS-calculated distance is de \textit{Ground Speed
    Distance} methodiek. Deze afstand kan enkel worden bepaald in het geval van een
fietsactiviteit. Deze afstand wordt berekend door het aantal omwentelingen te
vermenigvuldigen met de omtrek van het fietswiel~\cite{HowDista47:online}.

De bovenstaande afstandsberekeningen zijn de 2 technieken die officiële support
documentatie van Strava beschrijft~\cite{HowDista47:online}. Echter blijkt
wanneer de afstand op deze manier manueel berekent worden, afwijkende
resultaten bekomen worden. Dit is zeer waarschijnlijk te wijten aan de
preprocessing van de data die gebeurd bij het uploaden van een activiteit.
Alhoewel dit niet expliciet gedocumenteerd staat doen de resultaten dit wel
sterk vermoeden. De hypothese is dat tijdens het uploaden, de afstand
herberekend wordt. De gps-punten zullen worden geanalyseerd, en er zullen
technieken worden gebruikt om de resultaten hiervan te verbeteren. De twee
meest waarschijnlijke technieken zijn \textit{Map Snapping} en
\textit{Smoothing}.

Map Snapping of Snap to Roads is een techniek waarbij gps-punten worden
verschoven naar de dichtstbijzijnde weg. Per gps-punt wordt dan gezocht naar de
dichtste node op de desbetreffende \textit{roadgraph}\footnote{De roadgraph is
    afhankelijk van welke implementatie gebruikt wordt voor het snappen. Het is een
    wegennetwerk, omgezet in een graaf, bestaande uit edges en nodes. Elke weg of
    pad, bevat een of meerdere nodes, zodat een skeletstructuur ontstaat, die een
    abstractie van het wegennetwerk voorstelt~\cite{seiler2022haul}.} (op
Figuur~\ref{fig:MapSnapping} is de werking ervan te
zien)\cite{Snapping96:online}.
\begin{figure}[h]
    \centering
    \begin{subfigure}[b]{.5\textwidth}
        \centering
        \caption{Voorbeeldroute zonder map snapping}
        \includegraphics[width=0.5\textwidth]{fig/Map Snapping/before.png}\label{fig:before_MapSnapping}
    \end{subfigure}\hfill
    \begin{subfigure}[b]{.5\textwidth}
        \centering
        \caption{Voorbeeldroute met map snapping}
        \includegraphics[width=0.5\textwidth]{fig/Map Snapping/after.png}\label{fig:after_MapSnapping}
    \end{subfigure}
    \caption{Voorbeeld van de werking van een EPZ}\label{fig:MapSnapping}
\end{figure}

Daarnaast bestaat de kans dat er gebruik gemaakt wordt van smoothening.
Smoothening is een proces dat ruwe gps-punten (of datapunten in het algemeen)
op een traject probeert te optimaliseren opdat ze een vloeiend `curve' vormen.
Dit wordt bekomen door ruis, schommelingen en onnauwkeurigheden te filteren uit
het traject. Hiervoor bestaan verschillende implementaties. Aangezien Strava
geen openbare informatie verstrekt over het gebruik van GPS-smoothing, is het
niet bekend of ze deze techniek effectief toepassen. Het is dus gissen naar,
indien ze deze zouden gebruiken, welke implementatie dan wel gebruikt wordt. De
makkelijkste en meest modulaire methode om aan smoothening te doen, is
\textit{Smoothing met Moving Average}. Deze methode bestaat eruit om van een
aantal punten in een bepaalde range (ook `window' genoemd) het gemiddelde te
nemen, en vervolgens op te schuiven. Het gemiddelde wordt berekend met volgende
formule: $\overline{y_x} = \frac{y_x + y_{x+1} + \ldots + y_{n}}{x+n}$, voor
punt x, met n als window-grootte. Zo kan voor elk punt een evenwichtige waarde
op de nieuwe grafiek bekomen worden, en krijgt de grafiek een meer vloeiende
vorm. Merk wel op dat de precisie van de route daalt wordt op deze manier. Bij
het smoothen van een traject wordt het aantal gebruikte punten namelijk
vermindert volgens de grote van de window. Afhankelijk van de grote, worden
meer (resp.\ minder) punten samengenomen, en zo minder/meer punten weergegeven
op de grafiek. Een voorbeeld is terug te vinden op
figuur~\ref{fig:SmoothingExample}
~\cite{Smoothin16:online}\cite{SmoothingandInterpolatingNoisyGPSDatawithSmoothingSplines}\cite{Smoothin86:online}.
\begin{figure}[h]
    \centering
    \includegraphics[width=0.6\linewidth]{fig/SmoothingExample.png}
    \caption{Voorbeeld Data smoothing with moving average}\label{fig:SmoothingExample}
\end{figure}
% - https://support.strava.com/hc/en-us/articles/216917707-Bad-GPS-Data

\subsection{Algemeen Privacybeleid}\label{Algemene Privacy}
Het delen van alle data die vervat zit in zo'n activitei met alle andere
gebruikers op het platform, is zeker niet altijd wenselijk. De ontwikkelaars
kiezen er dan ook voor om gebruikers de mogelijkheid te geven om hun privacy te
bewaren. In deze sectie wordt de focus gelegd op de mechanismen gebruikt door
\textit{Strava}. Als opmerking valt te melden dat in heel wat andere
sport-applicaties worden vergelijkbare, zo niet dezelfde methodieken gebruikt.
Een eerste algemeen mechanisme bestaat eruit om de gebruiker de keuze te geven
om alle activiteiten en alle gegevens over het profiel heen te laten voldoen
aan bepaalde privacy regels. Deze regels kunnen ook per activiteit worden
ingesteld. Onder de keuzes staan meestal drie opties, \textit{zichtbaar voor
    iedereen}, \textit{zichtbaar voor volgers} en \textit{zichtbaar voor niemand}.
Er kan ook zelf een keuze gemaakt worden om specifieke elementen van een
activiteit niet te delen met de buitenwereld, zoals bijvoorbeeld de
zichtbaarheid van de kaart die de route weergeeft.\cite{Activity24:online}

\section{Endpoint Privacy Zones}\label{EPZ}
Een tweede belangrijke maatregel is het gebruik van de de \textbf{Strava
    Endpoint Privacy Zones (EPZ)}. Een EPZ is een cirkelzone met een bepaalde
straal rond een gps-punt. Het punt in kwestie zal dus de betreffende
\textit{gevoelige locatie} zijn. De straal van deze cirkel\footnote{Op Strava
    heeft de EPZ de vorm van een cirkel, maar op andere platformen kunnen andere
    vormen de norm zijn, bv.\ polygonen.} kan worden gekozen door de gebruiker, en
in het geval van Strava hebben gebruikers keuze uit waarden van 0 tot 1600m, in
stappen van 200m. Wanneer een gebruiker binnen deze zone zijn activiteit
beëindigt of begint, dan zal dat deel van de route binnen de EPZ niet zichtbaar
zijn voor anderen. Vanuit het perspectief van een andere gebruiker zal de
activiteit dus starten/eindigen op de rand van deze cirkel (die natuurlijk niet
zichtbaar is). Merk op dat een sporter ook andere gevoelige locaties kan
verbergen op de kaart. Bijvoorbeeld een frequent bezocht café, of een huis van
een partner waar regelmatig een tussenstop plaatsvindt. Een tweede opmerking is
dat wanneer een gebruiker de EPZ doorkruist, maar er niet in stopt, de route
onaangepast blijft. Op Figuur~\ref{fig:EPZ_Voorbeeld} zijn de verschillende
perspectieven te zien, hoe het er als uploader uitziet, en hoe het eruit ziet
voor een andere gebruiker. Het traject die de buitenstaander te zien krijgt,
zijn alle punten die zich buiten de EPZ bevinden. Merk ook op dat de eigenaar
van de activiteit zicht heeft op de EPZ, en wat zal verborgen worden die zich
buiten de EPZ bevinden. Dit onderscheid wordt gemaakt door het verschil in
kleur (oranje voor de publiek zichtbare punten en grijs voor de onzichtbare).
\begin{figure}[h]
    \centering
    \begin{subfigure}[b]{.7\textwidth}
        \centering
        \caption{Perspectief eigenaar}
        \includegraphics[width=1\textwidth]{fig/EPZ-mechanisme/Example_EPZ_InternalView.png}\label{fig:EPZ_internal}
    \end{subfigure}\hfill
    \begin{subfigure}[b]{.7\textwidth}
        \centering
        \caption{Perspectief externe gebruiker}
        \includegraphics[width=1\textwidth]{fig/EPZ-mechanisme/Example_EPZ_ExternalView.png}\label{fig:EPZ_external}
    \end{subfigure}
    \caption{Voorbeeld van de werking van een EPZ}\label{fig:EPZ_Voorbeeld}
\end{figure}

Het opzetten van een EPZ is dus een belangrijk onderdeel bij het blootleggen
van mogelijke zwakheden van dit systeem. Bij dit proces zal de gevoelige
locatie worden genomen als beginlocatie. Hieruit zal a.d.h.v.\ de op voorhand
vastgelegde EPZ-straal een cirkel worden opgesteld. Het centrum van deze cirkel
zal hierna een translatie ondervinden in een willekeurige richting. Dit kan een
verschuiving zijn met een afstand die maximaal 70\% van de straal van de EPZ
bedraagt (Figuur~\ref{fig:translation}). Het transleren van deze cirkel wordt
ook \textit{spatial cloaking} genoemd.
\begin{figure}[h]
    \centering
    \includegraphics[width=0.4\linewidth]{fig/EPZ-mechanisme/Translation_Center.png}
    \caption{Voorbeeld translatie EPZ}\label{fig:translation}
\end{figure}

Daarna worden alle punten vertrekkende vanaf de gevoelige locatie tot aan de
rand van de EPZ, en vanaf de rand van de EPZ tot aan de gevoelige locatie
verwijdert van het zichtbare traject. Merk op dat punten die de EPZ
doorkruisen, maar niet vertrekken/aankomen bij de gevoelige locatie niet worden
gefilterd (Figuur~\ref{fig:drop points}).
\begin{figure}[h]
    \centering
    \includegraphics[width=0.7\linewidth]{fig/EPZ-mechanisme/DropEPZPoints.png}
    \caption{Voorbeeld filtering van punten binnen EPZ}\label{fig:drop points}
\end{figure}

\section{Gerelateerd werken}
In het verleden is al wat onderzoek verricht in de richting van de
doeltreffendheid van EPZ's bij fitnesstrackers.~\citeauthor{sec18has3:online}
beschreef een implementatie van EPZ's waarbij het centrum van de zone de
gevoelige locatie is. M.a.w.\ het identificeren van deze zone is dus voldoende
om de gevoelige locatie te achterhalen\cite{sec18has3:online}. In tegenstelling
tot dit onderzoek, wordt ervan uitgegaan dat het centrum geen translatie
ondervindt, en er dus geen spatial cloaking wordt toegepast. In deze paper
wordt gefocust op de reconstructie van de cirkel op basis van 3 punten op de
rand (Figuur~\ref{fig:Hassan_EPZ}). Deze 3 randpunten worden dus bekomen door
begin/eindpunten te nemen van activiteiten, volgens het perspectief van
gebruiker die geen eigenaar is. Deze begin/eindpunten zullen zich altijd op de
rand van de cirkel begeven. In deze paper wordt spatial cloaking wel aangehaald
als mogelijke countermeasure tegen dit soort aanvallen.
\begin{figure}[h]
    \centering
    \includegraphics[width=0.6\textwidth]{fig/EPZ-mechanisme/Hassan.png}
    \caption{Mechanisme EPZ beschreven door \citeauthor{sec18has3:online}}\label{fig:Hassan_EPZ}
\end{figure}

Een onderzoek door~\citeauthor{10.1145/3491102.3502136} toonde ook aan dat
intuïtief heel wat mensen in staat zijn om de gevoelige locatie te achterhalen.
Dit gebeurde op basis van enquêtes die werden afgenomen bij gebruikers van het
platform. Uit het onderzoek bleek dat 68\% van de ondervraagden bij een
EPZ-radius van 200m de beschermde locatie tot op 50m nauwkeurig konden
voorspellen. Deze resultaten op zich zijn alarmerend, en tonen aan dat EPZ's
verre van perfect zijn.

\citeauthor{Dhondt_Pochat_Voulimeneas_Joosen_Volckaert_2022} voerde ook een studie naar de lekken aanwezig in het principe van EPZ's. Er
wordt in deze paper een nadruk gelegd op de translatie van de EPZ, en hoe deze
de privacy van een gebruiker verhoogt. Een inferentie aanval wordt er
beschreven die gebruikmaakt van de totale afstand, terug te vinden bij de
activiteit. Aan de hand van deze totale afstand in combinatie met het
wegennetwerk, wordt een poging gedaan om alle mogelijke routes die de sporter binnenin de EPZ zou kunnen afgelegd hebben, voor elk traject te
reconstrueren. Wanneer dit gedaan wordt voor verschillende trajecten, kan een
locatie voorspeld worden die het meest waarschijnlijk wordt geacht om de
gevoelige locatie te zijn.

Een laatste onderzoek die zeker ook het vermelden waard is, is de thesis
van~\citeauthor{Verdonck_2022} et al. Deze thesis bouwt in grote mate verder op
de paper van \citeauthor{Dhondt_Pochat_Voulimeneas_Joosen_Volckaert_2022}, maar
er wordt alternatieve data gebruikt. Er wordt gewerkt met hoogtedata i.p.v.\
totale afstanden, en zo wordt ook een inferentie aanval geconstrueerd.
% https://labs.strava.com/slide/
% https://www.jamesrcroft.com/2015/06/snapping-gps-tracks-to-roads/

% Beter uitwerken EPZ + literatuur achtergrond
% Beter uitwerken attack model
% 2.2 heruitwerken